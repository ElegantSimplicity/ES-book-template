\setcounter{chapter}{-1}
\chapter{Preliminaries}
\section{Transversality}

\todo{Interesting source appendix of Schultens\sidecite{schultens2014introduction}}


% TODO: spell check in brackets in vim!
\begin{definition}[Transversality]
    Let $M$ be a manifold and $N_1, N_2$ be two submanifolds.
    Then $N_1$ intersects $N_2$ \emph{transversely} if and only if for all points of intersection $p \in N_1 \cap N_2$, we have
    $T_pM = T_pN_1 + T_p N_2$.
    We denote this by $ N_1 \tcap N_2$.\sidenotemark[1]
\end{definition}
\sidenotetext[1][]{The notation is suggestive: the line $\cap$ intersects $|$ transversally: $\tcap$.}
Note in particular that this depends on the ambient manifold $M$, and that if $N_1$ and $N_2$ do not intersect, their intersection is vacuously transverse.

A useful result is that when two manifolds intersect transversely, the codimensions of their intersection is the sum of their codimensions.
\begin{prop}
    Let $M$ be a manifold and $N_1, N_2$ be two submanifold. If $ N_1 \tcap N_2$, then 
    \[
        \codim (N_1 \tcap N_2) = \codim N_1 + \codim N_2
    ,\] 
    where the codimensions relative to $M$.
    \label{prop:transverse-codimensions-add}
\end{prop}
\begin{proof}[Proof. TODO]
    As this is a local statement, we prove it for $M = \R^{m}$.
    We may assume that $N_1 = f_1^{-1}(0)$ and $ N_2 = f_2^{-1}(0)$ where $f_i$ are submersions from $\R^{p}$ to $\R^{n_i}$, where $n_i$ is the codimension of $N_i$.
    % i.e. $(f_i)_*$ has full rank everywhere.
    \begin{marginfigure}
        \centering
        \incfig{codimensions-transverse}
        \caption{When $N_1 \tcap N_2$, the codimension of the intersection is the sum of their codimensions. By using the implicit function theorem, we straighten the situation.}
        \label{fig:codimensions-transverse}
    \end{marginfigure}
    Then we can also consider $F= (f_1, f_2): \R^{m} \to  \R^{n_1} \times \R^{n_2}$.
    Notice that $ N_1 \cap N_2 = F^{-1}(0)$.
    Then
    \begin{align*}
        \dim N_1 \cap  N_2 &= \dim \Ker d_0F\\
                           &= \dim(\Ker d_0 f_1 \cap \Ker d_0 f_2)\\
                           &= \dim \Ker d_0 f_1 + \dim \Ker d_0 f_2 - \dim (\Ker d_0 f_1 + \Ker d_0 f_2)\\
                           &= (m - n_1) + (m - n_2) - m\\
                           &= m - (n_1 + n_2)
    .\end{align*} 
\end{proof}
Note that this proof also shows that $T_p(N_1 \tcap N_2) = T_pN_1 \tcap T_pN_2$ and that $ N_1 \tcap N_2$ is a manifold, which cannot be guaranteed when the intersection is not transverse.

\begin{marginfigure}
    \centering
    \incfig{intersection-of-manifolds-is-not-always-a-manifold}
    \caption{Let $M = \R^2$ and let $N_1$ and $N_2$ be submanifolds as in the figure. Then $N_1$ and $N_2$ do not intersect transversely and their intersection is not a manifold: it is the union of a point and a closed interval.}
    \label{fig:intersection-of-manifolds-is-not-always-a-manifold}
\end{marginfigure}


\todo{Give examples of transverse / non-transverse intersection and show that it is dependent of the ambient manifold}


\begin{theorem}[Sard's theorem]
    Let $f: M \to  N$ be a $\Cinfty$ map. Then the set of critical values of  $f$ has measure zero in $N$.
\end{theorem}
\begin{proof}
    Omitted.
\end{proof}


\sidecite[23]{schultens2014introduction}

\begin{definition}[Isotopy]
    Two embeddings $f_0, f_1: M \to  N$ are \emph{isotopic} if there is a continuous map $H: M \times  [0, 1] \to  N$ such that $H(x, 0) = f_0$ and $H(x, 1) = f_1(x)$ for all $x \in M$ and such that for all $t \in [0, 1]$, the map $f_t$ defined by $H(\cdot , t)$ is an embedding. The map $H$ is called an \emph{isotopy between} $f_0$ and $f_1$.

    Two submanifolds $S_0, S_1$ of $M$ are \emph{isotopic} if their inclusion maps are isotopic.
\end{definition}
