\chapter*{Summary}
\label{ch:summary}
\addcontentsline{toc}{chapter}{\nameref{ch:summary}}

Morse theory is the study of smooth functions from manifolds to $\R$. While these functions are some of the simplest structures that live on manifolds, it turns out that they can provide great insight in the structure of manifolds.

In Chapter \ref{chap:morse-theory}, we talk about the basics of Morse theory.
We give several equivalent definitions of a Morse function and give some examples.
We also show that Morse functions give rise to so-called handle decompositions and that any manifold admits (infinitely many) Morse functions.
Lastly, we give some examples of handle decompositions in dimensions one, two and three.

In Chapter \ref{chap:stable-and-unstable-manifolds}, we answer two important questions concerning reordering and cancelling critical points.
For this, we introduce the concept of stable and unstable manifolds and show that intersections of such manifolds are of particular interest.
We prove the reordering theorem and give a sketch for the theorem about cancellations of critical points.
Combining these results, we prove the existence of self-indexing Morse functions which give rise to Heegaard splittings, which we briefly discuss.
