\usepackage{enumitem}
\newlist{abbrv}{itemize}{1}
\setlist[abbrv,1]{label=,labelwidth=1in,align=parleft,itemsep=0.1\baselineskip,leftmargin=!}

\DeclareMathOperator{\Crit}{Crit}
\newcommand{\Cinfty}{C^\infty}

\newcommand{\stable}[1]{W^s(#1)}
\newcommand{\unstable}[1]{W^u(#1)}
\newcommand{\unstableb}[1]{\overline{W}^u(#1)}

\def\symbolentry#1#2#3{\item[#2] #3}
\def\sort#1{}

\makeatletter
\newcommand{\superimpose}[2]{%
  {\ooalign{$#1\@firstoftwo#2$\cr\hfil$#1\@secondoftwo#2$\hfil\cr}}}
\makeatother

% https://tex.stackexchange.com/questions/134863/command-for-transverse-and-not-pitchfork-as-used-in-guillemin-and-pollack
% \newcommand{\tcap}{\mathrel{\mathpalette\superimpose{{\raise0.15ex\hbox{$\top$}}{\cap}}}}
\newcommand{\tcap}{\pitchfork}

\newcommand\traj[2]{\mathcal M(#1, #2)}
\renewcommand\L[2]{\mathcal L(#1, #2)}
\newcommand\Lb[2]{\overline{\mathcal L}(#1, #2)}

\newcommand\nX[3]{n_{#1}(#2, #3)}
\newcommand\NX[3]{N_{#1}(#2, #3)}
\newcommand\HM[3][]{HM_{#1}(C_\bul(#2), \partial_#3)}
\newcommand\HMf[2][]{HM_{#1}(#2)}

\DeclareMathOperator{\Ind}{Ind}
\DeclareMathOperator{\Rank}{Rank}

\DeclareMathOperator{\codim}{codim}
\DeclareMathOperator{\grad}{grad}

\DeclareMathOperator{\Ker}{Ker}
\renewcommand{\Im}{\operatorname{Im}}

\newcommand\sphere[1]{S^{#1}}
\newcommand\cdisk[1]{B^{#1}}
\newcommand\odisk[1]{D^{#1}}
\newcommand\bul{\bullet}

 % \newcommand{\bigstar}{\mathop{\Huge \mathlarger{\mathlarger{*}}}}

\DeclareMathOperator{\Hom}{Hom}

\newcommand{\listofsymbols}{
    \chapter*{List of symbols}
    \begin{abbrv}


        % \symbolentry{U}{$U(\epsilon, \eta)$}{Morse chart}
        \symbolentry{0}{$M \tcap N$}{Transverse intersection}
        \symbolentry{1}{$\left<\cdot ,\cdot  \right>$}{Riemannian metric on a manifold or,\\ Inner product on space of critical points $\left<c, d \right> = \delta_{cd}$}
        \symbolentry{2}{$N \cdot N'$}{Intersection number of two manifolds}
        \symbolentry{A}{$A$}{A $\Z$-module, i.e.\ an abelian group}
        \symbolentry{B}{$\cdisk{n}$}{Closed disk of dimension $n$}
        \symbolentry{Ck}{$C_k(f, \Z)$}{
        Free module over $\Z$ generated by index $k$ critical points of $f$, i.e.\ the space of formal sums of index $k$ critical points
    }
        \symbolentry{Ck}{$C_k(f, \Z_2)$}{Vector space over $\Z_2$ generated by the index $k$ critical points of the Morse function $f$}
        \symbolentry{Codim}{$\codim N$}{Codimension of $N$}
        \symbolentry{Codim}{$\dim N$}{Dimension of $N$}
        \symbolentry{Critk}{$\Crit_k f$}{Critical points of $f$ of index $k$}
        \symbolentry{Crit}{$\Crit f$}{Critical points of $f$}
        \symbolentry{C}{$\Cinfty(M, N)$}{Smooth maps from $M$ to  $N$}
        \symbolentry{DpartialA}{$\partial_{X, k}$}{Morse differential associated to pseudo-gradient $X$}
    \symbolentry{DpartialM}{\mbox{$[\partial_k]$}}{Matrix of the Morse differential $\partial_k: C_k \to  C_{k-1}$}
        \symbolentry{D}{$\odisk{n}$}{Open disk of dimension $n$}
        \symbolentry{Grad}{$\grad f$}{Gradient of  $f$, i.e. $(df)^{\sharp}$}
        \symbolentry{HM2}{$\HMf{M, \Z_2}$}{Morse homology of a manifold $M$ with coefficients in $\Z_2$}
        \symbolentry{HM3}{$\HMf{M, \Z}$}{Morse homology of a manifold $M$ with coefficients in $\Z$}
        \symbolentry{HM}{$\HM{f}{X}$}{\mbox{Morse homology of Morse function $f$ and pseudo-gradient $X$.}}
        \symbolentry{H}{$H_k(M, N)$}{Singular homology of $M$ relative to $N$}
        \symbolentry{H}{$H_k(M, \Z)$}{Singular homology of $M$ over $\Z$}
        \symbolentry{H}{$H_k(M, \Z_2)$}{Singular homology $M$ over $\Z_2$}
        \symbolentry{Ind}{$\Ind a$}{Index of critical point $a$}
        \symbolentry{La}{$\L{p}{q}$}{Moduli space of unbroken trajectories between $p$ and $q$, i.e.\ $\traj{p}{q} / \R$, where $\R$ acts by time translations}
        \symbolentry{Lb}{$\Lb{p}{q}$}{Space of broken and unbroken trajectories between $p$ and $q$, the compactification of $\L pq$.}
        \symbolentry{M}{$M$}{A smooth manifold}
        \symbolentry{M}{$\traj{p}{q}$}{Set of all points on trajectories following a pseudo-gradient from $p$ to $q$, $\unstable p \tcap \stable q$.}
        \symbolentry{NX}{$\NX{X}{p}{q}$}{Signed number of trajectories of $X$ connecting  $p$ to $q$}
        \symbolentry{NX}{$\nX{X}{p}{q}$}{Number of trajectories of $X$ connecting  $p$ to $q$}
        \symbolentry{Pi}{$\pi_k(M)$}{Homotopy group of a manifold}
        \symbolentry{R0}{$r_0(A)$}{Free rank of a $\Z$-module, i.e.\ $\dim_{\Q} A \otimes \Q$}
        \symbolentry{Rp}{$r_p(A)$}{$p$-torsion rank of a $\Z$-module, i.e.\ cardinality of a maximal set of independent elements of order $p^{k}$ for some $k$}
        \symbolentry{Rt}{$r_t(A)$}{Total torsion rank of a $\Z$-module, i.e.\ $\sum r_t$}
        \symbolentry{Ru}{$r(A)$}{Total rank of a $\Z$-module, i.e.\ $r_t(A) + r_0(A)$}
        \symbolentry{S1}{$S^{s}(p)$}{Stable sphere associated to a critical point $p$, alternatively called the belt sphere}
        \symbolentry{S2}{$S^{u}(p)$}{Unstable sphere associated to a critical point $p$, alternatively called the attachment sphere}
        \symbolentry{S}{$\sphere{n}$}{Sphere of dimension $n$}
        \symbolentry{W1}{$\stable{p}$}{Stable manifold of a critical point $p$}
        \symbolentry{W2}{$\unstable{p}$}{Unstable manifold of a critical point $p$}
        \symbolentry{W3}{$\unstableb{p}$}{Compactification of the unstable manifold associated to a critical point $p$}
        \symbolentry{X}{$X$}{Pseudo-gradient vector field}
    \end{abbrv}
}

\newcommand{\tpoinc}[1]{\ensuremath{\mathrm P_{\text{top}}^{#1}}}
\newcommand{\spoinc}[1]{\ensuremath{\mathrm P_{\infty}^{#1}}}
\newcommand{\ppoinc}[1]{\ensuremath{\mathrm P_{\text{PL}}^{#1}}}
\newcommand{\cpoinc}[1]{\ensuremath{\mathrm P_{C}^{#1}}}

\newcommand{\tcob}[1]{\ensuremath{\mathrm H_{\text{top}}^{#1}}}
\newcommand{\scob}[1]{\ensuremath{\mathrm H_{\infty}^{#1}}}
\newcommand{\pcob}[1]{\ensuremath{\mathrm H_{\text{PL}}^{#1}}}
\newcommand{\ccob}[1]{\ensuremath{\mathrm H_{C}^{#1}}}

\newcommand{\mant}{\ensuremath{\textsf{Man}_{\text{top}}}}
\newcommand{\mans}{\ensuremath{\textsf{Man}_{\infty}}}
\newcommand{\manp}{\ensuremath{\textsf{Man}_{\text{PL}}}}

\usepackage{booktabs}
\usepackage{array}
\newcommand{\overview}[7]{
    \smallskip
    \begin{center}
        \begin{tabular}{
                >{\centering\arraybackslash}p{1.3cm}%
                >{\centering\arraybackslash}p{1.3cm}%
                >{\centering\arraybackslash}p{1.3cm}%
                >{}p{0.3cm}%
                >{\centering\arraybackslash}p{1.3cm}%
                >{\centering\arraybackslash}p{1.3cm}%
                >{\centering\arraybackslash}p{1.3cm}}
                \tpoinc{#1} & \ppoinc{#1} & \spoinc{#1} && \tcob{#1} & \pcob{#1} & \scob{#1}\\
                #2   & #3 & #4 & & #5 & #6 & #7
        \end{tabular}
    \end{center}
}
