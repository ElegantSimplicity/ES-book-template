\setcounter{chapter}{-1}
\chapter{Preliminaries}

\lipsum[1-10]

\newthought{Lorem ipsum dolor sit amet}, consetetur sadipscing elitr, sed diam nonumy eirmod
tempor invidunt ut labore et dolore magna aliquyam \sidenote{This is a side note!} erat, sed diam voluptua. At
vero eos et accusam et justo\sidecite{crandall2006prime}.
duo dolores et ea rebum. Stet clita kasd gubergren,
no sea takimata sanctus est Lorem ipsum dolor sit amet.

\begin{marginfigure}
    \centering
    \incfig[1]{test-figure}
    \caption{This is a test figure}
    \label{fig:test-figure}
\end{marginfigure}

\lipsum[1-5]

\begin{marginfigure}
    \centering
    \incfig[1]{test-figure}
    \caption{This is a test figure}
    \label{fig:test-figure}
\end{marginfigure}


\lipsum[6-10]

Bigger figure:


\begin{SCfigure}[b]
    \caption{This is a bigger figure. It shows some ellipses.}
    \incfig{bigger-figure}
    \label{fig:bigger-figure}
\end{SCfigure}
\lipsum[1-10]

This is a list:

\begin{itemize}
    \item Item 1
    \item Item 2
    \item Item 3
\end{itemize}

Large figure
\begin{figure*}
    \centering
    \incfig[1]{test-figure}
    \caption{This is a test figure. This is a really long caption!}
    \label{fig:test2-figure}
\end{figure*}

Theorem: 

\begin{theorem}[Sard's theorem]
    \lipsum[60]
\end{theorem}
