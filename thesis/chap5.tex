\chapter{$h$-cobordism theorem}
\section{Introduction}

In this chapter, we will prove three important theorems that are all closely related.
Let us first state them.
The first one states that the Morse inequalities are in fact attainable under certain conditions.
In other words, under these assumptions, the bound on $\# \Crit_k f$ given by the (weak) Morse inequalities is in fact the best bound possible. This was first proven by Smale\sidecite{smale1960generalized} in 1960.
\begin{theorem}[Smale's theorem]
    If $M$ is a simply connected closed manifold of dimension  $m \ge  6$ and $H_\bul(M)$ is free, then there exists a Morse function on  $M$ such that  $\# \Crit_k f = r_0(H_k(M, \Z))$ for all $k = 0, \ldots, n$.
\end{theorem}
Remember that we use $r_0(A)$ to denote the free rank of a $\Z$-module $A$.
A closely related theorem is the $h$-cobordism theorem, stating
\begin{theorem}[h-cobordism theorem]
    If $M$ is a cobordism from $M_0$ to $M_1$ that is simply connected, $\dim M \ge 6$ and $H_{\bul}(M,M_0) = 0$, then $M$ is a trivial cobordism, i.e.\ diffeomorphic to  $ M_0 \times I$.
\end{theorem}
\todo{Introduce relative homology}
The last theorem---and perhaps most famous one---we will prove is generalized Poincaré conjecture for dimensions $n \ge 5$.
This conjecture (now a theorem) states that any homotopy sphere is a topological sphere. By this, we mean the following:
% TODO spell check in [] of theorems1
\begin{theorem}[Higher dimensional Poincaré conjecture]
    If $M$ is a homotopy sphere of dimension  $n \ge 5$, then $M$ is homeomorphic to  $S^{n}$.
\end{theorem}
This theorem has a great historical significance, and the proofs for dimension $n = 3, 4$ are notoriously difficult and very subtle\sidecite{stallings2016not}. We will briefly 

These theorems are all closely related, and we will prove them using a handle cancelling approach which roughly goes as follows:
\begin{enumerate}[(1)]
    \item Let $f: M \to  \R$ be an arbitrary Morse function.
    \item Under certain assumptions, we can cancel pairs of critical points.
    \item Repeatedly cancelling critical points allows us to conclude that (depending on the assumptions):
        \begin{itemize}
            \item The resulting Morse function has no critical points (in the case of $M$ being a cobordism);
            \item The resulting Morse function attains the Morse inequalities; or
            \item The resulting Morse function has 2 critical points: a minimum and a maximum, which implies that $M \cong S^{n}$, as we will show.
        \end{itemize}
\end{enumerate}

\todo{To this purpose, this chapter will mostly contain cancellation results.
Because the proof in total is very lengthy,
we will first give an overview, and we will later fill in details where needed.
We follow Milnor, and also Antonio Kosinski}



\section{Stronger cancellation result}
Let us first recall the cancellation result we stated previously:

\begin{theorem}[Francois Laudenbach\sidenotemark]
    Let $f: M \to  \R$ be a Morse function and $X$ be a pseudo-gradient field.
    Let $p$ and $q$ be two critical points such that
    \begin{enumerate}[(i)]
        \item $p$ and $q$ are connected by a unique gradient line  $\ell := \traj{p}{q}$.
        \item For some $\epsilon>0$, each gradient flow line starting in $\unstable{p}$ distinct from $ \ell$ crosses the level set $f^{-1}(f(q) - \epsilon)$
    \end{enumerate}
    Then these two points are cancelable.
More precisely, there exists an isotopy of functions $f_t$ starting at $f_0=f$ supported in an open neighbourhood of the closure of  $\unstable{p} \cap  \{f \ge  f(q) - \epsilon\} $ such that
\begin{itemize}
    \item $f_t|_U$ is Morse with two critical points when  $0 \le t\le \frac{1}{2}$;
    \item $f_t|_U$ has a cubic singularity when $t=\frac{1}{2}$
    \item $f_t|_U$ has no critical points when $\frac{1}{2}<t\le 1$.
\end{itemize}
\end{theorem}
\sidenotetext[][0.0cm]{\fullcite{laudenbach2013proof}}

In this chapter, we will mostly consider cobordisms, and in that context, the statement of this theorem can be expressed shortly as
\begin{theorem}
    Let $f: M \to  [0,1]$ be a Morse function on a cobordism with two critical points $p, q$.
    If $|\L pq| = 1$, then $p$ and  $q$ are cancelable.
\end{theorem}

Phrased in this way, this theorem begs for a generalization.
In the chapter on Morse homology, we have seen that we can orient $\L pq$.
If $\L pq$ is a discrete space (as is the case when  $\Ind p - \Ind q = 1$), then this orientation assigns a sign to each of the points in  $\L pq$.
Therefore, instead of considering the cardinality of $\L pq$, it would be interesting if the theorem still holds if we consider the signed count of the points in $\L pq$, i.e.\ if $\NX Xpq = \pm 1$.
So instead of requiring that there is a single gradient flow line connecting $p$ and  $q$, we would have that the signed count of flow lines connecting  $p$ and  $q$ is  $\pm 1$.
It turns out that this is indeed the case if we add some additional assumptions, giving the first strengthening of our cancellation result.

\begin{theorem}[Second Cancellation Theorem]
    Let $f: M \to  [0,1]$ be a Morse function on a cobordism $M$ (from $M_0$ to $M_1$) with two critical points $p, q$.
    Suppose furthermore that $M, M_0, M_1$ are simply connected and that
    \[
    2 \le \Ind q = k  \qquad \Ind p = k+1 \le  n-3
    .\] 
    % Let $X$ be a pseudo-gradient vector field adapted to $f$ satisfying the Smale condition.
    If $\NX X p q = \pm 1$, then  $p$ and  $q$ are cancelable.
    \label{thm:cancel-second}
\end{theorem}
\begin{remark}
    By changing $f$ to $-f$, the theorem is also true if we replace the conditions with  $k \ge  3$ and $k + 1 \le  n-2$.
    This means that the theorem is true for 
    \[
    2 \le  \Ind q = k \quad \Ind p = k + 1 \le n-2
    .\] 
\end{remark}

Before giving an idea of how we will prove this result, let us first give some examples.
\begin{eg}
    The following examples should be taken with a grain of salt, because the conditions of the theorem are actually not satisfied (in particular the dimension is too low).
    However, they are still helpful as they express the idea of the theorem.
    We will consider a $3$-dimensional manifold $M$ with a Morse function $f$ and let $\alpha$ be a regular value such that  $f^{-1}((-\infty, \alpha])$ consists of a $0$, $1$ and  $2$ handle as in the figures below.
    Let us the four situations from left to right.
    \begin{enumerate}[(i)]
        \item The first figure shows the $2$- and  $1$- handle and their corresponding critical points $p$ and  $q$.
            It is visually clear that these two critical are cancelable.
        \item The second figure shows the same situation, but we have hidden the $2$-handle and indicated the so-called belt sphere $S^{s}(q)$ and the attaching sphere $S^{u}(p)$. Their intersection consists of a single point, corresponding to a single flow line from $p$ to $q$. 
            By the first cancellation theorem, we can confirm that $p$ and  $q$ are indeed cancelable.
        \item Next, we have drawn a situation where the number of intersection points is $3$, but the intersection number (which corresponds to $\NX Xpq$, as we will expand upon later) is $1$. It is clear that we can isotope the attaching sphere such that it only intersects once with the belt sphere.
            This reduces the situation to (ii), so $p$ and  $q$ are cancellable, as is also clear from the figure. 
        \item The last figure shows a situation where the number of intersections points is $2$ and the intersection number is $0$. In this situation, we cannot cancel the two critical points.
    \end{enumerate}
    \begin{figure}[H]
    \centering
    \incfig{second-cancellation-theorem-example}
    \caption{TODO second cancellation theorem example}
    \label{fig:second-cancellation-theorem-example}
\end{figure}
\end{eg}

\begin{marginfigure}
    \centering
    \incfig{second-cancellation-theorem-setup}
    \caption{Setup of the second cancellation theorem in dimension two and three. Note the figure is somewhat misleading because of dimensionality reasons.}
    \label{fig:second-cancellation-theorem-setup}
\end{marginfigure}
As already hinted at before in the example, the proof of this stronger cancellation result will reduce the given situation where $\NX X pq = \pm 1$ to one that satisfies the condition of the first cancellation theorem, i.e. $| \L pq | = 1$.
In order to do this we want to `cancel' flow lines of opposite signs.
We will do this by using a theorem of Whitney which allows to cancel intersection points of opposite intersection numbers.\sidenote{For a primer on intersection numbers, see Chapter 0.}

The correspondence goes as follows: we can interpret $\NX X pq$ as the intersection number of two submanifolds.
Indeed, let $\alpha$ be a regular value between  $f(p)$ and  $f(q)$ and let $V = f^{-1}(\alpha)$.
Let $S^{u}(p) = \unstable p \cap V$ and 
$S^{s}(q) = \stable q \cap V$ be spheres inside $V$.
Then we have noted before that $\L pq \cong S^{u}(p) \cap S^{s}(q)$ so that \[
    | \L pq| = |S^{u}(p) \cap S^{s}(q)|.
\]
Moreover, we have that
\[
    \NX X pq = S^{u}(p) \cdot S^{s}(q)
,\] 
where we denote the intersection number of two manifolds $N$ and $N'$ by $N \cdot N'$.
For this to be defined, we assign arbitrary orientations of the unstable manifolds in order to define $\NX X pq$, and we also use these orientations for calculating the intersection number$S^{u}(p)$ and $S^{s}(q)$.
The orientation issues here are the same as those we discussed in the chapter on Morse homology.

We now state the theorem of Whitney, prove the second cancellation theorem and end with the proof of Whitney's theorem.
\begin{theorem}[Whitney]
    Let $N$ and  $N'$ be smooth, closed, tranversely intersecting submanifolds of dimensions  $r$ and $s$ in a smooth  $(r+s)$-dimensional manifold $V$.
    Suppose  $N$ is oriented and  $N'$ is cooriented.
    Suppose $r+s \ge 5$, $s\ge 3$ and if $r<3$ suppose that $\pi_1(V - N') \hookrightarrow \pi_1(N)$ is an isomorphism.

    Let $p, q \in  M \tcap N'$ points with opposite intersection number as in Figure~\ref{fig:intersection-points-cancellation} such that there exists a loop $L$ contractible in  $V$ connecting  $p$ smoothly to $q$ in  $N$  and then $q$ smoothly to  $p$ in $N'$ where both arcs miss other intersection points.

    Then there exists an isotopy $h_t$ of the identity $V \to V$ such that
    \begin{enumerate}[(i)]
        \item The isotopy is locally the identity around other intersection points;
        \item At time  $t = 1$,  $N'$ and  $N$ do not intersect in $p$ and $q$  any more.  In other words, $h_1(N) \cap N' = N \cap  N' \setminus \{ p, q\} $.
    \end{enumerate}
\end{theorem}
\begin{marginfigure}[-6.5cm]
    \centering
    \incfig{intersection-points-cancellation}
    \caption{
    Under certain conditions, we can `cancel' intersection points of opposite intersection number by deforming the manifold $M$ by an isotopy.}
    \label{fig:intersection-points-cancellation}
\end{marginfigure}
\begin{proof}[Proof of the second cancellation theorem]
    Let $X$ be an adopted pseudo-gradient vector field adopted to  $f$ and satisfying the Smale condition.
    Let $N = S^{u}(q) = \unstable q \cap V$ and $N' = S^{s}(p) = \stable p \cap  V$.

    We now that $S^{u}(q) \cdot S^{s}(p) = \pm 1$, which means that either there is only one flow line connecting $p$ and  $q$, in which case the proof reduces to the first cancellation theorem, or there are multiple flow lines with opposite signs.
    If we can show that the conditions of the theorem of Whitney are satisfied, we can cancel these intersection points pair by pair (by altering $X$) until we have reached the situation of the first cancellation theorem.

    Let $V = f^{-1}(\alpha)$.  If $M$ is simply connected, then $V$ is also simply connected.\sidenote{This follows from applying Seifert--Van Kampen theorem: $\pi_1(V) \cong \pi_1(\stable{p} \cup  V \cup \unstable{q})$ (here we use that $k > 2, n - k \ge 3$).
    Then notice that $\stable{p} \cup V \cup \unstable{q}$ is homotopic to $M$ showing that $\pi_1(V) \cong \pi_1(M) = 1$.}
    If $k \ge 3$, then all the conditions of the theorem are satisfied and we are done.
    If $k = 2$, we need to show that  $\pi_1(V - S^{s}(p)) \cong \pi_1(V) = 1$,
    or equivalently we need to show that $\pi_1(V - S^{s}(p))$ is trivial.
    Let $S = \unstable q \cap f^{-1}(0)$.
    Flowing $M_0 \setminus S$ via $X$ gives  $V - S^{s}(p)$, so we need to show that $M_0 \setminus S$ has trivial fundamental group.
    For this we use Seifert--Van Kampen. Let $N$ be a product neighbourhood\sidenote{If the normal bundle of a submanifold is trivial, a tubular neighbourhood is called a product neighbourhood. TODO why does this exists?} of $S$ in $M_0$.
    Note that $S$ is diffeomorphic to $S^{1}$, and $\dim M_0 = n-1$, so the product neighbourhood is diffeomorphic to $ S^1 \times \R^{n-2}$.
    As $k=2$, $n - 2 \ge 4$, so $\R^{n-2} \setminus \{ 0 \} $ has trivial fundamental group. Therefore $N \setminus S \cong S^{1} \times (\R^{n-2}\setminus \{0\})$ has the same fundamental group as $S^{1}$, which is $\Z$.
    This allows us to use Seifert--Van Kampen in the following way:
    $(M_0 \setminus S) \cup N = M_0$, and $\pi_1(M_0) = 1$ $\pi_1(N) = \Z$ $\pi_1(N \setminus S) = \Z$. This implies that $\pi_1(M_0 \setminus S) = 1$, completing the proof.
\end{proof}

Let us finally prove the theorem of Whitney.

\begin{proof}[Proof of Whitney's theorem]
    We will construct the isotopy of $N$ in a local model.
    \paragraph{Plane model}
    Let $C, C'$ be the arcs in  $N$ and $N'$ connecting $p$ and  $q$ and extend them a little bit either way, as in Figure~\ref{fig:whitneys-theorem-proof-model}.
    \begin{marginfigure}
        \center
        \incfig{whitneys-theorem-proof-model}
        \caption{On the left: the plane model, on the right: the higher dimensional model.}
        \label{fig:whitneys-theorem-proof-model}
    \end{marginfigure}
    For the plane model, let $C_0$ and $ C_1$ be open curves in the plane intersecting transversely in two points, call them $a$ and $b$.
    Let $D$ be the disk with two corners enclosed by $C_0$ and $C_0'$.
    Then there exists an embedding $\phi_1$ of these curves into $N \cup N'$ such that the following holds:
     \begin{itemize}
         \item $\phi_1(C_0) = C$, $\phi_1(C_0') = C'$,
         \item $\phi_1(a) = p$, $\phi_1(b) = q$.
    \end{itemize}
    \paragraph{Model}
    Now we claim that we can extend this embedding by adding extra dimensions such that the following conditions are satisfied:
    \begin{itemize}
        \item The new embedding $\phi: U \times \R^{r-1} \times \R^{s-1}$ is an extension of $\phi_1|_{U \cap (C_0 \cup C_0')}$,
        \item $\phi^{-1}(N) = (U \cap C_0) \times \R^{r-1} \times 0$,
            \item $\phi^{-1}(N') = (U \cap C_0) \times 0\times \R^{s-1}$.
    \end{itemize}
    This is actually quite subtle, and for a detailed proof of this claim, we refer the reader to the notes of Milnor\sidecite[p.~75]{hcobord}.
    \begin{marginfigure}
        \centering
        \incfig{whitneys-theorem-model-isotopy}
        \caption{The isotopy $G_t$ in the plane model moves $C_0$ below $ C_0'$, i.e. $ G_1(U \cap C_0) \cap C_0' = \O$.}
        \label{fig:whitneys-theorem-model-isotopy}
    \end{marginfigure}
    With this model, the proof of the theorem follows quickly.
    \paragraph{Isotopy in the plane model}
    Let $G_t: U \to U$ be an isotopy in the plane model that when applied to $C_0$, moves it under $C_0'$ as in Figure~\ref{fig:whitneys-theorem-proof-model}.
    More specifically, we require the following:
    \begin{itemize}
        \item $G_0$ is the identity map,
        \item $G_t$ is the identity near the boundary of  $U$ for all $t$,
        \item  $ G_1(U \cap C_0) \cap C_0' = \O$.
    \end{itemize}
    \paragraph{Isotopy in the model}
    To extend this isotopy to one on $U \times \R^{r-1} \times \R^{s-1}$, define a bump function $\rho: \R^{r-1}\times \R^{s-1} \to  [0,1]$ supported in $\{(x, y)  \mid |x|^2 + |y|^2 \le  1\}$ and set
    \begin{align*}
        H_t: U \times \R^{r-1}\times \R^{s-1} &\longrightarrow  U \times \R^{r-1} \times \R^{s-1}\\
        (u, x, y) &\longmapsto (G_{t \rho(x, y)}, x, y)
    .\end{align*}
    \paragraph{Isotopy of $V$}
    To finally find an isotopy of $V$, define $F_t: V\to V$ such that
    $F_0$ is the identity, $F_t$ is the identity everywhere except away from  $\Im \phi$, and on  $\Im \phi$, define  $F_t = \phi  \circ  H_t  \circ  \phi^{-1}$.
    This finishes the proof.
\end{proof}

\section{Cancellation of points in the middle dimensions and diagonalizing $[\partial_k]$}

The second cancellation theorem discussed in the previous section allows us to cancel two critical points if there is (counting with signs) a single flow line connecting them.
In this section, we will show how we can reach that situation.

We can interpret the condition of the second cancellation theorem in another way, namely that the matrix of the differential $\partial_k$, whose entries are \[
    [\partial_k]_{p, q} = \NX X p q \qquad (p,q) \in C_k(f, \Z) \times  C_{k-1}(f, \Z)
\] has a row and a column that intersect in $\NX X p q = 1$ whose other entries are zeros. Then the theorem allows us remove this row and column, by which we mean that we can alter $X$ and $f$ such that $[\partial_{X', k}]$ is $[\partial_{X, k}]$ with this one row and column removed.
For example, in the case of $|C_k| = 4, |C_{k-1}| = 5$, the reduction could look as follows:
\[
    \begin{pmatrix}
        * & *&0 & *\\
        * & *&0 & *\\
        0 & 0&\pm 1 & 0 \\
        * & *&0 & *\\
        * & *&0 & *
    \end{pmatrix}
\leadsto
\begin{pmatrix}
    * & *& *\\
    * & *& *\\
    * & *& *\\
    * & *& *
\end{pmatrix}
.\] 
This restatement of the theorem leads us to our next goal: diagonalizing $[\partial_k]$ and showing that under certain conditions,  $[\partial_k]$ is a diagonal matrix with only $\pm 1$ on the diagonal. Then we can apply the second cancellation theorem multiple times, cancelling all the critical points (that lie in the middle dimensions, per assumption of the theorem).

Let us make clear what we mean by diagonalizing.
Starting off with a matrix $[\partial_k]$ based on $f$ and $X$, we want to change it step by step such that the resulting $[\partial_k']$ is diagonal (note that it does not need to be square). These changes are done by altering $f$ and $X$.

We will consider the following elementary operations:
\todo{In order to diagonalize a matrix over $\Z$, it suffices to be able to do three elementary row/column operations\sidecite[p.~682]{rotman2010advanced}:}
\begin{enumerate}[E1]
    \item Interchange two rows or columns
    \item Multiplication of a row or column by $-1$
    \item Addition of a row (resp.~column) to another row (resp.~column)
\end{enumerate}

It is clear that E1 and E2 can be done: we can just relabel the critical points for E1 and change the (arbitrary) orientation of the unstable manifolds for E2.
For the third operation, consider the following figure:

\begin{figure}[H]
    \centering
    \sidecaption{TODO addition of columns geometrically\label{fig:addition-of-columns-geometrically}}
    \incfig{addition-of-columns-geometrically}
\end{figure}
This shows how such an addition operation can be performed geometrically: isotope the attachment sphere of the second $2$-handle$, i.e.\ S^{u}(p_2)$ to the sphere that is `$S^{u}(p_1)$ connected to $S^{u}(p_2)$ via a tube'. Note that this isotopy depends on the existence of the first $2$-handle.
We have that $\partial'(p_2) = q_1 + q_2 = \partial(p_1) + \partial(p_2)$.

To show that this works in general, we will use the following lemma:
\begin{lemma}
    Let $N$ be a connected closed manifold of dimension $m-1$ containing two embedded $(k-1)$-spheres $ S_1$, $S_2$ ($1 < k < m$).
    Assume that $S_1$ bounds a $k$-disk $D$  disjoint from $ S_2$. Then there is an isotopy in $N$ of  $ S_2$ to a sphere $S$ that is `$S_1$ connected to $ S_2$' with a tube.
    More explicitly, $S$ consists of $S_1$ and $ S_2$ with small discs removed and of a tube connecting these openings.

    If $V$ is a submanifold of dimension $m-k$ that does not disconnect $N$,  then we can assume that the tube does not intersect $V$ such that
     \[
         V \cap S = (V \cap S_1) \cup (V \cap S_2)
    .\] 
\end{lemma}
\begin{figure}[H]
    \centering
    \incfig{lemma-sphere-connect-with-tube}
    \caption{TODO lemma sphere connect with tube}
    \label{fig:lemma-sphere-connect-with-tube}
\end{figure}
\begin{proof}
    \begin{marginfigure}
        \centering
        \incfig{lemma-sphere-connect-with-tube-proof}
        \caption{TODO lemma sphere connect with tube proof}
        \label{fig:lemma-sphere-connect-with-tube-proof}
    \end{marginfigure}
    Let $L$ be an arc in $N$ that is disjoint from  the interior of $K$ connecting  $s_1$ to $s_2$ points in $S_1$ and $S_2$.
    Let $D_1, D_2$ be disks around $s_1$ and $ s_2$ in $ S_1$ and $S_2$.
    First move $s_2$ along $L$ to  $ s_1$ and extend this to an isotopy that moves a smaller disk $D \subset D_2$ to $D_1$, and keeps $ S_2 - D_2$ fixed.
    Then move $D_1$ `in' $D$ keeping its boundary fixed. Again extend to an isotopy of $N$.
    Composing these isotopies, we get the result.
\end{proof}

\begin{remark}
    It is also clear that $S\cdot V = S_1\cdot V \pm S_2 \cdot V$, because the tube misses the submanifold $V$.
    If $k < m-1$ we can actually choose this sign freely by changing the first isotopy in the proof: we can either move $D$ to $ D_2$ with the same or the opposite orientation. If $k = m-1$, we do not have this freedom.
\end{remark}

This shows that we can perform E3 geometrically:
\begin{theorem}
    Let $1 < k < m$. Then we can perform E1, E2, E3 geometrically on $[\partial_k]$, only affecting critical points of index $k$ and $k+1$.
\end{theorem}
\begin{proof}
    Operations E1 and E2 are clear.
    Assume the Morse function $f$ is self-indexing.
    Let  $M^{t} = f^{-1}([-\infty, t))$ and let $V$ be the union of all belt spheres of critical points of index $k-1$,i.e. \[
        V = \bigcup_{p \in \Crit_{k-1} f} S^{s}(p)
    \]
    Let $N = f^{-1}(k - \frac{1}{2})$.
    For dimensional reasons, $V$ can only disconnect $N$ if  $k = 2$.
    However, the $N \setminus V \cong f^{-1}(k - \frac{3}{2})$ with a finite set of points removed, namely the attaching spheres of the one-handles, as illustrated in Figure~\ref{fig:proof-e1-e2-e3-disconnect-case}.
    Therefore, as $f^{-1}(k- \frac{3}{2})$ is connected\sidenote{\todo{Why}}, so is $N \setminus V$.
    This means we can apply the previous lemma and we conclude that we can add/subtract one column to the other one.
    If $k < m-1$, we have control over the sign (determining addition or subtraction), and if $k = m-1$, we may need to change the orientation of the attaching sphere of  $k$-handle first.
    Notice that this does not only affect $[\partial_k]$ and also  $[\partial_{k+1}]$.\sidenote{\todo{How?}}
\end{proof}

In order to also diagonalize $[\partial_1]$ and  $[\partial_m]$, we will use the following fact:
\begin{theorem}
    Let $M$ be a connected cobordism from  $ M_0$ to $ M_1$.
    If $ M_0 \neq \O$, there exists a handlebody decomposition with exactly one $0$-handle. In the other case, there exists one without $0$-handles.
    In both cases, $[\partial_1]$ is trivial.
    \label{thm:no-zero}
\end{theorem}
\begin{proof}
    Suppose $M_0 \neq \O$, i.e.\ the cobordism has a bottom border and suppose $\# \Crit_0 f =1$, i.e.\ there is a single $0$-handle.
    Then because $M$ is connected, there must exist an $1$-handle connecting the $0$-handle to another connected component of $f^{-1}((-\infty, \frac{1}{2}])$.
    The first cancellation theorem allows us to cancel the $0$-and  $1$-handle.
    It is clear how to extend this to multiple $0$-handles.

    Next, suppose $M_0 = \O$ and $\# \Crit_0 f  = 2$. Then because $M$ is  connected, there must be a $1$-handle connecting the two components in $f^{-1}((-\infty, \frac{1}{2}])$. Again using the first cancellation theorem , we can cancel the $0$- and $1$-handle. A similar reasoning works if $\# \Crit_0 f > 2$.

    For the last part of the theorem, stating that $[\partial_1]$ is trivial, consider the following. 
    If there are no  $0$-handles, $\partial_1: C_1 \to  C_0$ is clearly trivial.
    If there is a single $0$-handle, all $1-$ handles have both ends attached to the same sphere, so the intersection number is $0$, so $[\partial_1] = (0 \ \cdots \ 0)$.
\begin{marginfigure}
    \centering
    \incfig{without-zero-handles}
    \caption{TODO without zero handles}
    \label{fig:without-zero-handles}
\end{marginfigure}
\end{proof}
\begin{remark}
    Again by turning the cobordism upside down, we can conclude the same for $n$-handles and $[\partial_n]$.
\end{remark}

The two previous theorems allow us diagonalize all $[\partial_k]$ \emph{simultaneously}.
\begin{theorem}
    Let $M$ be a cobordism from  $ M_0$ to $M_{1}$.
    Assume $M, M_0, M_1$ are connected and oriented.
    Then there exists a Morse function such that $[\partial_k]$ is diagonal for all $k$.
\end{theorem}


\begin{proof}
    Proof by induction.
    Note that $[\partial_1]$ is diagonal by the previous theorem.
    Suppose $[\partial_i]$ is diagonal for  $1 \le  i < k < n$.
    We have shown that we can do operations E1--E3 on columns geometrically.
    By turning the cobordism upside down, we can also do this for rows.
    These six operations are all that is needed for the diagonalizing $[\partial_k]$ by using the algorithm of Smith\sidecite{rotman2010advanced}.

    In order to see that this does not change the already diagonalized matrices, we use the fact that $\partial^2 = \partial_{k-1} \circ \partial_{k} = 0$.

    Lastly $[\partial_n]$ is already diagonal by the previous theorem.
    \todo{Finish proof}
\end{proof}
\todo{Here we care if we change $k-1$ when diagonalizing $k$, but in the previous theorem, it is diagonalizing $k$ and changing $k+1$?}
\todo{Check if they use rows/columns differently. }
\todo{Check prop, theorem, lemma throughout}

\begin{marginfigure}
    \centering
    \incfig{proof-e1-e2-e3-disconnect-case}
    \caption{In the case $k = 2$, the union of the belt spheres $V$ does disconnect $M^{3 / 2}$. However, it is diffeomorphic to $M^{ 1 / 2}$ with a finite set of points removed, and since $M^{ 1 / 2}$ is connected, so is  $M^{ 1 / 2} \setminus V$.}
    \label{fig:proof-e1-e2-e3-disconnect-case}
\end{marginfigure}


\section{Cancelling $1$-handles}
The previous theorems allow us to diagonalize $[\partial_k]$ for all $k$.
Moreover, the second cancellation theorem allows us to remove columns and rows of $[\partial_k]$ under certain conditions.

These conditions do not cover the case $[\partial_2]$ and it requires simply connectedness of $M$ for  $[\partial_3]$, which could be problematic if the handlebody decomposition contains  $1$-handles.
Similar problems occur for $[\partial_{n-2}]$ and $[\partial_{n-3}]$.
In this section, we will show that we can eliminate $1$-handles by replacing them with $3$-handles, solving these issues.

\todo{Here simply connectedness is still required?}
\begin{prop}
    Let $M$ be a cobordism from $M_0$ to $M_1$.
    Assume that $M$ and  $M_0$ are connected, $M$ is simply connected and $\dim M \ge 5$.
    Then there there exists a handlebody decomposition with $0$ (or  $1$) $0$-handles if $M_0 \neq \O$ (or $M_0 = \O$), without $1$-handles.
    Moreover, starting from another handlebody decomposition, we can alter it such that it satisfies these conditions, without changing the number of handles of index greater than $3$.
    \label{thm:one-to-three}
\end{prop}
\begin{marginfigure}
    \centering
    \incfig{changing-one-handles-in-three-handles}
    \caption{
        To change a $1$-handle into a $3$-handle, we first introduce a pair of cancelling auxiliary handles of index $2$ and $3$. 
        Then we cancel the $1$- and $2$-handle, leaving us with a  $3$-handle.
    }
    \label{fig:changing-one-handles-in-three-handles}
\end{marginfigure}
\begin{proof}
    The idea of the proof goes as follows and is illustrated in Figure~\ref{fig:changing-one-handles-in-three-handles}.
    In order to change a $1$-handle into a $3$-handle, we introduce an auxiliary cancelling pair of $2$- and $3$-handles.
    Then we cancel the $1$- and $2$-handle, leaving us with a $3$-handle.
    \paragraph{Adding a cancelling pair of $2$- and $3$-handles}
    By the first cancellation theorem, we can decompose an $m$-dimensional disk $D^{m}$ as a $2$- and  $3$-handle. (The attachment sphere of the $3$-handle intersects the belt sphere of the $2$-handle once transversely.)
    Therefore, we can remove a small disk from $M$ and fill it up with a $2$- and $3$-handle.
    The attachment sphere $L$ of the $2$-handle bounds a $2$-dimensional disk in $f^{-1}(\frac{3}{2})$ and we can make sure that it does not intersect belt spheres of $1$-handles or attachment regions of other $2$-handles.

    Our next goal is to isotope this $2$- and $3$-handle such that the attachment region $L$ of the $2$-handle crosses a $1$-handle exactly once. Then we will be able to cancel the $2$-handle against the $1$-handle.

    \paragraph{Constructing the desired attachment region}
    Let $L'$ a loop on top of a $1$-handle that intersects the belt sphere once transversely.
    By Theorem~\ref{thm:no-zero}, we can assume that there are a minimum number of $0$-handles, implying that $f^{-1}(\frac{1}{2})$ is connected.
    Hence we can connect the endpoints of $L'$ with a curve that lies in $f^{-1}(\frac{1}{2})$,
    moreover missing the other attachment spheres of $1$- and $2$-handles.\sidenote{
        The attachment spheres of the $1$-handles are a finite number of disjoint points, so it is easy to avoid them.
        For the second claim, we can assume that the loop is smooth and transversal to the attachment region of the $2$-handles. Now note that transversal under these conditions means disjoint because of dimensional reasons.
    }
    Because $M$ is simply connected, $f^{-1}(\frac{3}{2})$ is as well, so this loop is null-homotopic.
    \paragraph{Isotoping $L$ to $L'$}
    Both $L$ and $L'$ are null-homotopic in  $f^{-1}(\frac{3}{2})$.
    Because $\dim f^{-1}(\frac{3}{2}) \ge 4$, they are in fact isotopic, by a theorem of Withney.\sidecite{whitney1936differentiable}
    It states that if $f, g: M \to N$ be two \emph{homotopic} embeddings of a compact manifold. If $\dim N \ge  2 \dim M + 2$, then $f$ and  $g$ are \emph{isotopic.}
    Both embeddings of $L$ and  $L'$ are null homotopic, and  $\dim f^{-1}(\frac{3}{2}) \ge 4 \ge  2 \dim S^{1} +  2$, so the conditions are satisfied.
    Hence, we can isotope $L$ to $L'$, and assume that the $2$- and  $3$-handle combination is actually attached along $L'$, i.e.\ the attachment region of the $2$-handle is $L'$.

    Because $L'$ crosses the  $1$-handle exactly one time, we can cancel the $1$- and $2$-handle as claimed before, ending the proof.
\end{proof}
\begin{remark}
    This idea can be extended to higher index critical points as well, but we will not need that here.
\end{remark}
\begin{remark}
    By reversing the cobordism ($f \leadsto -f$), we can do the same for $n-1$-handles.
\end{remark}

\section{Minimality of the Morse inequalities}
\begin{theorem}
    Let $M$ be a cobordism from $M_0$ to $M_1$.
    Assume $M, M_0, M_1$ are connected and simply connected and $\dim M \ge 6$.
    Assume moreover that the homology of $M$ is free, i.e.  $H_{\bul}(M, M_0)$ and $H_{\bul}(M, M_1)$ are free.
    Then there exists a Morse function such that
    \[
        \# \Crit_k f = r_0(H_k(M, M_0, \Z))
    .\] 
    In other words, under these conditions, the Morse inequalities are attainable.
    \label{thm:minimal-cob}
\end{theorem}
\begin{proof}
    Let $f$ be an arbitrary Morse function on $M$.
    We show inductively that $[\partial_k]$ is trivial.
    Using Theorem~\ref{thm:no-zero}, we can alter $f$ such that the number of $0$-handles is minimized.
    Theorem~\ref{thm:one-to-three} allows us to change $1$-handles into $3$-handles.
    \begin{enumerate}
        \item[$H_0$]
            We have $C_{-1} = 0$, so $\Ker \partial_0 = C_0$.
            As there are no $1$-handles, $\Im \partial_1 = 0$.
            Hence  $H_0(M, M_0) = C_0$.
        \item[$H_1$]
            There are no $1$-handles.
            Hence $\Ker \partial_1 = 0 = C_1$, and $\Im \partial_2 = 0$. Hence  $H_1(M, M_0) = C_1$.
        \item[$ H_2$]
            By definition, $H_2(M, M_0) = \frac{\Ker \partial_2}{\Im \partial_3}$.
            However, there are no $1$-handles,
            so $\Ker \partial_2 = C_2$.
            By altering the Morse function and gradient, we can assume that $[\partial_3]$ is diagonal.
            This combined with the fact that the homology is free allows us to conclude that $\Im \partial_3$ is matrix with as entries $\pm 1$.
            (If it would contain e.g.\ a $2$, then the resulting quotient could be  $\Z / 2 \Z$, which has torsion.)
            This allows us to cancel pairs of critical points removing rows and columns until $[\partial_3]$ is trivial.
            We can do so without altering the triviality for $k = 1,2$.
            We conclude that $H_2(M, M_0) = C_2 / 0 = C_2$.
        \item [$H_k$]
            By definition $H_k(M, M_0) = \frac{\Ker \partial_k}{\Im \partial_{k+1}}$.
            We know that $[\partial_k]$ is trivial, hence  $\Ker \partial_k = C_k$.
            Because $H_k$ does not have torsion, this does mean that $[\partial_{k+1}]$ only contains $\pm 1$'s, so we can cancel critical points until  $[\partial_{k+1}]$ is trivial. We can do this without changing the triviality of $[\partial_\ell]$ for $\ell < k$.
            We conclude that $H_k(M, M_0) = C_k$.
    \end{enumerate}
    While the proof now seems finished, we should be careful when $k$ gets close to  $n$, because then the second cancellation does not apply any more.
    We can fix this by first doing the previous process for $k = 1, \ldots, n-2$. 
    Then we turn the cobordism upside down ($f \leadsto -f$), and repeat the procedure, eventually making all matrices trivial resulting in $H_k(M, M_0) \cong C_k$.
\end{proof}
\begin{remark}
    If we do not assume that the homology is free, the weak Morse inequalities are not attainable.
    However, we can prove that the Morse inequalities including torsion rank in fact are.
\end{remark}
\begin{corollary}
    If $M$ is a simply connected closed manifold of dimension  $n \ge 6$ with free homology, then the Morse inequalities are attainable, i.e.\ there exists a Morse function such that
    \[
        \# \Crit_kf = r_0 (H_k(M, \Z))
    .\] 
\end{corollary}

\begin{corollary}[$h$-cobordism theorem]
    Let $M$ be a cobordism from  $M_0$ to $M_1$.
    If $M, M_0, M_1$ are connected and simply connected, $\dim M \ge  6$ and $H_{\bul}(W, V_0) = 0$, then $M$ is a trivial cobordism, i.e.  $M \cong M_0 \times [0,1]$.
\end{corollary}

TODO: talk about name $h$-cobordism

\begin{corollary}[Generalized Poincaré conjecture]
    If $M$ is a homotopy sphere of dimension  $n \ge  6$, then $M$ is homeomorphic to  $S^{n}$.
\end{corollary}
\begin{proof}
    By the previous theorem, there exists a Morse function with exactly one $0$-handle and one $n$-handle, i.e. $M$ consists of two disks glued along their boundary. This is homeomorphic to a sphere, as the following explicit homeomorphism shows:
    \begin{align*}
        h: S^{n} = D_1^{n} \cup_\text{Id}  D_2^{n} &\longrightarrow D_1^{n} \cup_\phi D_2^{n} \\
         x &\longmapsto 
         \begin{cases}
             x & \text{if $x \in D_1^{n}$}\\
             \|x\| \phi\Big(\frac{x}{\|x\|}\Big) & \text{if $x \in D_2 ^{n} \setminus \{0\} $}\\
             0 & \text{if $x = 0 \in D_2^{n}$.}
         \end{cases}
    \end{align*}
\end{proof}
\begin{remark}
    We cannot conclude that $M$ is \emph{diffeomorphic} to $S^{n}$.
    Indeed, there are so-called exotic spheres, which are topological spheres with a differential structure that is not equivalent to the standard differential structure on $S^{n}$.
    We will expand on this in the following section.
\end{remark}

The next result shows that there is only one differential structure on disks:
\begin{theorem}
    If $M$ is contractible with a simply connected boundary and of dimension $n \ge  6$, then $M$ is diffeomorphic to $B^{n}$
\end{theorem}
\begin{proof}
    Consider $M$ as a cobordism from  $\O$ to  $\partial M$.
    We have that  $H_\bul(M) \cong H_\bul(M, \O)$.
    By Poincaré duality, this is also isomorphic to  $H^\bul(M, \partial M)$.
    Because  $H^{\bul}(M, \partial M)$ is finitely generated, it is isomorphic to $H_\bul(M, \partial M)$.\sidecite[Chapter 5, Section 5, Corollary 4]{spanier1989algebraic}
    In conclusion, $H_\bul(M) \cong H_\bul(M, \O) \cong H_\bul(M, \partial M)$.
    This means that $H_\bul(M, \O)$ and  $H_\bul(M, \partial M)$ are free.
    Applying Theorem~\ref{thm:minimal-cob} gives, together with the fact that $M$ is contractible, a handlebody decomposition with exactly one $0$-handle.
    Hence $M$ is diffeomorphic to a disk.
\end{proof}

This allows us to strengthen the generalized Poincaré theorem:
\begin{theorem}[Generalized Poincaré conjecture for $n \ge  5$]
    Let $M$ be a homotopy sphere of dimension  $\ge  5$. Then $M \cong S^{n}$.
\end{theorem}
\begin{proof}
    We claim that $M \# (-M)$ bounds a contractible manifold $W$ of dimension greater than or equal to $6$.
    Indeed, consider the following figure.
\begin{figure}[H]
    \centering
    \sidecaption{The connected sum of a manifold with itself bounds a manifold $W$ which deformation retracts on  $M \setminus D^{n}$.}
    \incfig{connected-sum-homotopy-spheres}
    \label{fig:connected-sum-homotopy-spheres}
\end{figure}
The two top rows show a local model of a connected sum, considering the connected sum of two copies of $\R^{n}$ with reversion orientation on one of the copies.
The resulting manifold bounds a manifold $W$ and squashing in the vertical direction gives a deformation retract from $W$ to  $ \R^{n} \setminus D^{n}$.
On the bottom, we use this local model for $ M \# (-M)$. This shows that $M \#(-M)$ bounds a manifold that is homotopy equivalent with  $M \setminus D^{n}$.
Because $M$ is a homotopy sphere, we can easily use Mayer--Vietoris to compute that the homology of $M \setminus D^{n}$ is that of a contractible manifold.

By the previous theorem, this implies that $W$ is diffeomorphic to $D^{n+1}$.
Hence, $M \#(-M)$ is homeomorphic to $S^{n}$.
If we can prove that the sphere $S^{n}$ is irreducible, this implies that $M$ is homeomorphic to $S^{n}$.
\end{proof}

\begin{lemma}[Irreducibility of the sphere]
    If $S^{n} \cong M \# N$, then $M, N \cong S^{d}$.
\end{lemma}
\begin{proof}
    We use a trick called Mazur's swindle.\sidecite{mazur1959embeddings}
    It is based on the fact that the infinite connected sum $\#_{n=1}^{\infty} M_n$ is well defined and associative.
    Suppose $M \# N \cong S^{n}$.
    Consider
    \[
       C =  (M \# N) \# 
        (M \# N) \#  \cdots
    .\] 
    This is an infinite connected sums of spheres, hence a half-open cylinder.
    Pinching its boundary results in $S^{n}$.
    Using associativity, this is also equal to
    \[
        C= M
        \# (N \# M)
        \# (N \# M)
        \# \cdots
    ,\] 
    which is homeomorphic to $M$ with a small ball removed. Pinching its boundary, we get $M$.
    This proves that $M$ is homeomorphic to $S^{n}$.
\end{proof}
\url{https://terrytao.wordpress.com/2009/10/05/mazurs-swindle/#irred}

% TODO Firstly note that Σ is in particular a homology sphere by Hurewicz theorem (i.e. has homology isomorphic to that of a sphere) . 
% TODO Next, it follows from the Hurewicz theorem (plus the fact that π1(Σ′)≅π1(Σ)={1}) that all of the homotopy groups πi(Σ′) are trivial. Finally, by Whiteheads theorem the inclusion of a point in Σ′ induces an isomorphism on all homotopy groups hence is a homotopy equivelance, i.e. Σ′ is a contractible space.
% TODO https://math.stackexchange.com/questions/2987911/if-sigma-is-a-homotopy-sphere-then-sigma-sigma-bounds-a-contractible


\section{Historical aspects of the Poincaré conjecture}

table of results, categories, \ldots

\todo{
\begin{itemize}
    \item Relative Morse homology
        \item Fix notation for cardinality
        \item Spell check in brackets theorem
        \item Check $H_1 = 0$ does not imply simply connectedness \url{https://mathoverflow.net/questions/110158/what-if-i-want-to-look-for-a-space-with-vanishing-first-homology-but-nonzero-fun}
\end{itemize}
}
