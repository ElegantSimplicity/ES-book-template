\chapter{Morse homology}
\label{chap:morse-homology}

% \startcontents[chapters]
% \printcontents[chapters]{}{1}{}

\section{Morse complex}
In order to define the Morse complex, we need a sequence of modules over a certain ring and maps between these modules.
Most of the time, we will be working over $\Z/2\Z$, which we will denote by $\Z_2$, but sometimes the ring $\Z$ will be used instead. As will turn out, working over $\Z_2$ will allow us not to worry about orientation and it being a field also has some advantages.

The modules we will considering depend on a Morse function $f$ and consist of formal sums of critical points of a certain index:
\[
    C_k(f) = \Big\{ \sum_{p \in \Crit_k f} n_p p  \mid n_p \in \Z_2\Big\}  = \bigoplus_{p \in \Crit_k f} \Z_2 p
.\] 
Note that this implies that $C_{-1} = C_{-2} = \cdots = 0$ and $C_{n+1} = C_{n+2} = \cdots = 0$, where $n$ is the dimension of the manifold.


\begin{marginfigure}
    \centering
    \incfig{other-sphere-morse-complex-modules}
    \caption{The critical points of the height function can be split up depending on their index and form the generators of the modules in the Morse complex.}
    \label{fig:other-sphere-morse-complex-modules}
\end{marginfigure}

\begin{eg}
    Consider the other sphere in Figure~\ref{fig:other-sphere-morse-complex-modules}. We have
    \[
    C_0 = \{0, a\}  \qquad C_1 = \{0, b\}  \qquad C_2 = \{0, c, d, c+d\} 
    .\] 
\end{eg}

The definition of the differential is based on a pseudo-gradient $X$: it counts trajectories connecting critical points of lower index.
Because critical points of index $k$ generate $C_k$, it suffices to define $\partial_{X,k}$ on these critical points and extend linearly:
\begin{align*}
    \partial_{X, k}: C_k &\longrightarrow C_{k-1} \\
    p &\longmapsto \sum_{q \in \Crit_{k-1} f} \nX Xpq q
,\end{align*}
where $\nX Xpq$ is the number of trajectories of $X$ connecting $p$ and $q$, modulo $2$.
If we require that $X$ satisfies the Smale condition, we will later show that this is well defined, by which we mean that $n_X(p,q)$ is an integer (see also Remark~\ref{remark:trajectories-finite}).
If it is clear from the context, we will often drop $X$, $k$ or both from the notation.
\begin{eg}
    Consider again the other sphere with its height function.
    Let $X$ be the gradient induced from the standard gradient in $\R^3$.
    Then $\partial_X$ is defined as follows, keeping in mind that we are working over $\Z_2$ and that $C_{-1} = 0$:
\begin{figure}[H]
    \centering
    \incfig{morse-homology-other-sphere-differential}
    \caption{Definition of the differential $\partial_X$}
    \label{fig:morse-homology-other-sphere-differential}
\end{figure}
\end{eg}

With this information, we are ready to define the Morse complex and Morse homology.
\begin{definition}[Morse complex]
    Let $f:M \to \R$ be a Morse function and $X$ a pseudo-gradient with the Smale property.
    Then the Morse complex is 
    \[
        \cdots \xrightarrow{\partial} C_2(f) \xrightarrow{\partial}  C_1(f) \xrightarrow{\partial} C_0(f) \textcolor{gray}{{} \xrightarrow{\partial}  0 \xrightarrow{\partial }0 \xrightarrow{\partial}  \cdots}
    \] 
    The Morse Homology $\HM{f}{X}$ is the homology of this chain complex.
\end{definition}
The attentive reader will have noticed that for this to be a chain complex, we need $\partial^2 = 0$, which is not obvious at all.
Apart from this, we also would to like to prove that  this homology does not depend on the choice of the Morse function $f$ and the pseudo-gradient field $X$.
We will address these topics in the two following sections, but let us first compute the Morse homology of some examples.


\begin{marginfigure}
    \centering
    \incfig{other-sphere-and-normal-sphere-morse-complex}
    \caption{Two embeddings of the sphere in $\R^3$.
        The corresponding height functions are Morse functions and give rise to a different Morse complex.
        However, the resulting Morse homology is the same.
    }
    \label{fig:other-sphere-and-normal-sphere-morse-complex}
\end{marginfigure}

\begin{eg}[Homology of the (other) sphere]
    We have already computed the differential in the previous example,
    so computing the homology is just a matter of applying the definition.
    \begin{itemize}
        \item $\HM[0]{f}{X} = \dfrac{\Ker \partial: C_0 \to  C_{-1}}{\Im \partial: C_1 \to  C_0} = \dfrac{\{0, a\} }{\{0\} } \cong \Z_2$
        \item $\HM[1]{f}{X} = \dfrac{\Ker \partial: C_1 \to  C_{0}}{\Im \partial: C_2 \to  C_1} = \dfrac{\{0, b\} }{\{0, b\} } \cong 0$
        \item $\HM[2]{f}{X} = \dfrac{\Ker \partial: C_2 \to  C_{1}}{\Im \partial: C_3 \to  C_2} = \dfrac{\{0, c+d\}}{\{ 0\}} \cong \Z_2$
    \end{itemize}
    In summary, we have
    \[
        H_0 = \Z_2 \qquad H_1 = 0 \qquad H_2 = \Z_2
    .\] 

    Instead of embedding $S^2$ in this strange way, we can also repeat the same calculation with its standard embedding, illustrated in Figure~\ref{fig:other-sphere-and-normal-sphere-morse-complex}.
    We have $ C_2 = \{0, b\} $, $ C_1 = \{0\} $  and $C_0 = \{0, a\}$, and $\partial(b) = 0$,  $\partial(a) = 0$.
    This way, we obtain
    \begin{itemize}
        \item $\HM[0]{g}{Y} = \dfrac{\Ker \partial: C_0 \to  C_{-1}}{\Im \partial: C_1 \to  C_0} = \dfrac{\{0, a\} }{\{0\} } \cong \Z_2$  
        \item $\HM[1]{g}{Y} = \dfrac{\Ker \partial: C_1 \to  C_{0}}{\Im \partial: C_2 \to  C_1} = \dfrac{\{0\} }{\{0\} } \cong 0$
        \item $\HM[2]{g}{Y} = \dfrac{\Ker \partial: C_2 \to  C_{1}}{\Im \partial: C_3 \to  C_2} = \dfrac{\{0, b\} }{\{0\} } \cong \Z_2$
    \end{itemize}
    In summary, we have 
    \[
    H_0 = \Z_2 \qquad H_1 = 0 \qquad H_2 = \Z_2
    ,\] 
    exactly the same homology as with the other Morse function and other gradient.
    As mentioned earlier, we will prove that this is in general the case.
\end{eg}
\begin{eg}[Homology of the three-torus]
    \label{eg:homology-of-the-three-torus}
    Let us recall Example~\ref{eg:handle-decomposition-three-torus}, which discusses the three-torus $T^{3} = S^{1} \times S^{1} \times S^{1} = \R^3 / \Z^3$ and the following Morse function
    \begin{marginfigure}
        \centering
        \incfig{morse-homology-three-torus-trajectories}
        \caption{
        Trajectories connecting critical points whose index differ by exactly $1$.
        Here $T^{3} = \R^{3} / \Z^{3}$ and we have drawn $\big(-\frac{1}{2}, \frac{1}{2}\big]^3$ as representative cube.  }
        \label{fig:morse-homology-three-torus-trajectories}
    \end{marginfigure}
    \begin{marginfigure}
        \centering
        \incfig{mores-homology-three-torus-critical-points-graph}
        \caption{Graph of the critical points of $f$ on $T^{3}$.
            Each edge represents a flow line between points whose indices differ by one.
        }
        \label{fig:mores-homology-three-torus-critical-points-graph}
    \end{marginfigure}
    \begin{align*}
        f: T^3 &\longrightarrow \R \\
        (x,y,z) &\longmapsto 2 \cos(2 \pi x)+ 3 \cos(2 \pi y) + 4 \cos (2 \pi z),
    \end{align*}
    with critical values repeated here for convenience:
    \[\begin{array}{cccccc}
  & x & y  & z & f(x, y, z) & \text{Index}\\ \hline
  & 0 & 0 & 0 & 9 & 3 \\
& \frac{1}{2} & 0 & 0 & 5 & 2 \\
& 0 & \frac{1}{2} & 0 & 3 & 2 \\
& 0 & 0 & \frac{1}{2} & 1 & 2 \\
& \frac{1}{2} & \frac{1}{2} & 0 & -1 & 1 \\
& \frac{1}{2} & 0 & \frac{1}{2} & -3 & 1 \\
& 0 & \frac{1}{2} & \frac{1}{2} & -5 & 1 \\
& \frac{1}{2} & \frac{1}{2} & \frac{1}{2} & -9 & 0 \\
    \end{array}\]
    The differential equation for trajectories along $\grad f$ is 
    \begin{align*}
        \dot x &= -4 \pi \sin(2 \pi x)\\
        \dot y &= -6 \pi \sin(2 \pi y)\\
        \dot z &= -8 \pi \sin(2 \pi z)
    .\end{align*} 


    Note that $\dot x$ only depends on  $x$,  $\dot y$ on  $y$ and  $\dot z$ on $z$, so this is a decoupled system that is easy to solve.
    Requiring that for $t \to  \pm\infty$ we end up in critical points, we find the trajectories in Figure~\ref{fig:morse-homology-three-torus-trajectories}.
    To have a better overview, we can also make a graph of critical points and trajectories between them, as is done in Figure~\ref{fig:mores-homology-three-torus-critical-points-graph}.
    This graph provides all the information we need to compute the Morse homology of $T^{3}$.
    Note that all critical points are connected with two gradient lines, which means that each differential $\partial_3, \partial_{2}, \partial_1, \partial_0$ is the zero map, making it very easy to compute the homology. In the end, we get
    \[
    H_0 = \Z_2 \qquad 
    H_1 = \Z_2^3 \qquad 
    H_2 = \Z_2^3 \qquad 
    H_3 = \Z_2
    ,\] 
    which corresponds to the usual homology.
\end{eg}

\section{The Morse complex is a complex: $\partial^2 = 0$}
In this section, we will prove that the Morse complex is actually a complex, by which we mean that $\partial^2 = 0$.
The core idea of the proof is very geometrical and depends on the fact that compact one-dimensional manifolds with boundary have an even number of boundary points.\sidenote{
    Compact one-dimensional manifolds consist of disjoint union of copies of $S^{1}$ and closed intervals, as proven in \fullcite[p.55]{milnor1997topology}
}
We will first make this core idea clear, and then make this argument precise.

Let us first restate what we have to prove. Let $a \in \Crit_k f$ be a critical point of a Morse function $f: M \to  \R$.
We need to prove that $\partial^2(a) = 0$, so\begin{align*}
    \partial^2(a) &= \partial \Big(\sum_{c \in \Crit_{k-1} f} n_X(a, c) c\Big)\\
                  &= \sum_{b \in \Crit_{k-1} f} \sum_{c \in \Crit_k f} n_X(a, c) n_X(c, b) b
.\end{align*} 
We see that $\partial^2(a)$ counts trajectories from $a$ to points of index $k-2$ that are once broken in a critical point of index $k-1$.
Remember that we are working over $\Z_2$, so if we can prove that these once broken trajectories always occur in pairs, we are done.
\charlotte{REWORD: The core idea of the proof is to notice that these once broken trajectories together with the space of unbroken trajectories $\L ab$ form a compact 1-dimensional manifold with boundary exactly the broken trajectories.}
    We have illustrated this in Figure~\ref{fig:morse-complex-is-a-complex-idea-of-proof}.
In other words, we can compactify $\L ab$ by adding the once broken trajectories between $a$ and $c$, resulting in a space we will denote by $\Lb ab$.
As previously remarked, this is sufficient to prove that $\partial^2(a) = 0$, because compact 1-dimensional manifolds always have an even number of boundary points.

\begin{marginfigure}
    \centering
    \incfig{morse-complex-is-a-complex-idea-of-proof}
    \caption{
        Core idea of the proof stating $\partial^2 = 0$.
        There are two unbroken trajectories from $a$ to $b$, one passing through $c_1$ and one through $c_2$.
        There is a one parameter family of unbroken trajectories interpolating between the two broken ones.
        Together, they form a compact 1-dimensional manifold with boundary, which has an even number of boundary points.
    }
    \label{fig:morse-complex-is-a-complex-idea-of-proof}
\end{marginfigure}

Before we proceed with this idea, let us give an overview of the steps that need to be taken

\begin{enumerate}
    \item Define $\Lb ab$, the space of all (broken and unbroken) trajectories.
    \item Define a topology on $\Lb ab$ so that we are able to talk about compactness
    \item Prove that  $\Lb ab$ is the compactification of $\L ab$
    \item Prove that if $\Ind a - \Ind b = 2$, then $\Lb ab$ is a 1-dimensional manifold with as boundary the once broken trajectories between $a$ and  $b$
\end{enumerate}

\subsection{The space of broken trajectories}
\begin{definition}
    The space of broken trajectories between $a$ and  $b$ is
     \[
         \Lb ab = \bigcup_{c_i \in \Crit f} \L a{c_1} \times \L {c_1}{c_2} \times  \cdots \times \L {c_\ell}b
    .\] 
\end{definition}
\charlotte{From this expression it is unclear that is ok eto have no points in the middle. Rewrite e.g. using power set.}

    Note that the number $\ell$ describes how many times the trajectory is broken.
    In other words, each broken trajectory in $\L a{c_1} \times \L {c_1}{c_2} \times  \cdots \times \L {c_\ell}b$ has $ \ell+1$ segments,  and we will denote such a trajectory by $(\lambda_1, \lambda_2, \cdots, \lambda_{\ell+1})$.
    In the case where $(\Ind a, \Ind c, \Ind b) = (k, k-1, k-2)$ \charlotte{Explain why you only consider this configuration of indices}, this definition results in
    \[
        \Lb ab = \L ab \cup  \bigcup_{c \in \Crit_{k-1} f}  \L a{c} \times \L {c} b
    ,\] 
    which makes it clear it is the union of unbroken and once broken trajectories.

     \todo{
         If manifold with boundary, 
         then $\partial \Lb ab = \bigcup_{c} \L a c \times  \L c b$ so $| \partial \Lb ab| = \sum n_X(a,c) n_X(c,b) b$.
     }
    


\subsection{Topology of $\Lb ab$}
% Also remember that $\L ab$ is homeomorphic to $\traj ab \cap f^{-1}(\alpha)$ for some regular value $\alpha$ between $f(a)$ and $f(b)$,


\begin{marginfigure}
    \centering
    \incfig{morse-homology-definition-of-topology}
    \caption{The topology on $\Lb ab $ is defined by looking at the entrance and exit points in the Morse charts.
        Here we have shown in black paths that lie in a neighborhood of the broken path $(\lambda_1, \lambda_2) \in \Lb ab$.
    }
    \label{fig:morse-homology-definition-of-topology}
\end{marginfigure}
To define the topology on $\Lb ab$,
we will describe a basis around a (possibly broken) trajectory.
Let $(\lambda_1, \lambda_2, \ldots, \lambda_q)$ be a trajectory in $\Lb ab$.
Consider Morse charts around each critical point  $a, c_1, c_2, \ldots, c_{q-1}, b$.
Consider opens around entry and exit points of the trajectories, lying in level sets of $f$, indicated in the figure with thick black lines.
We declare all trajectories passing through these opens to be in a open neighborhood of $(\lambda_1, \lambda_2, \ldots, \lambda_q)$.
Doing this for all possible `entrance and exit opens' defines a basis of the topology.

Remember that the topology of  $\L ab$ came from the one on $\traj ab \subset M$ using the projection map $\pi: \traj ab \to \L ab $.
We also mentioned that $\L ab$ is homeomorphic to $\traj ab \cap  f^{-1}(\alpha)$ for some regular value $\alpha$, making it clear that the topology on $\Lb ab$ corresponds to the one  $\L ab$ for unbroken trajectories.

So the essence of the topology on $\Lb ab$ is: `trajectories are nearby if entrance and exit points in Morse charts are nearby'.

\subsection{$\Lb ab$ is the compactification of $\L ab$}
Now we are ready to prove that $\Lb ab$ is compact, and in fact a compactification of  $\L ab$.\sidenote{By this, we mean that there are points in $\L ab$ arbitrary close to ones in  $\Lb ab$.\charlotte{Express this in terms of topology}}
\begin{theorem}
    $\Lb ab$ is compact
\end{theorem}

We will actually prove that $\Lb ab$ is sequentially compact, which will turn out to be enough for our needs, and is in this context actually equivalent with being compact because the topology on $M$ is second countable.

\begin{figure}[H]
    \centering
    \sidecaption{
        To show that $\Lb ab$ is compact, we consider a sequence of paths $\ell_n$ in $\L ab$ and find an accumulation trajectory, i.e.\ a limit of a subsequence.
        In case 1, the accumulation trajectory lies in $\L ab$.
        Case 2 shows the situation when accumulation trajectory is broken in  $c$ (and possibly in multiple other critical points).
    \label{fig:partial-squared-zero-space-compact}
    }
    \incfig{partial-squared-zero-space-compact}
\end{figure}



\begin{proof}
    Let us first consider a sequence of unbroken trajectories, i.e.\ a sequence $\ell_n$ in $\L ab$.
    Let $\Omega(a)$ be a Morse chart around $a$.
    Consider the points where $\ell_n$ exits the Morse chart $a$, call them $a_n^{-}$.
    Using compactness of the sphere, extract a subsequence of $\ell_n$ such that $a_{n}^{-}$ converges and call the point of convergence $a^{-}$.
    We will denote the extracted subsequence again by $\ell_n$ and will do so continuously in the proof.
    Similarly, extract a subsequence such that $b_n^{+}$, defined as the entry point in the Morse chart of $b$ converges to a point $b^{+}$.
    In conclusion, we have a subsequence whose exit point in $\Omega(a)$  converges to $a^{-}$ and whose entrance in $\Omega(b)$ converges to  $b^{+}$.

    Let us now try to construct a trajectory that is the `accumulation trajectory'\sidenote{By this, we simply mean an accumulation point of the sequence $\ell_n$, but this terminology can be confusing because it is not a point in the geometrical sense of the word.} of $\ell_n$.
    An obvious starting point would be to consider the unique trajectory $\gamma_1$ passing through $a^{-}$ going from $a$ to another critical point.
    Suppose for a moment that this other critical points is in fact $b$, so $\gamma_1$ connects  $a$ and  $b$, as in case 1 in Figure~\ref{fig:partial-squared-zero-space-compact}.
    Lemma~\ref{lemma:level-sets}, which we will prove later on\sidenote{
        For convenience, we repeat the statement here:\\
        \textbf{Lemma~\ref{lemma:level-sets}.}
        Let $x$ be a regular point of $f$ and $x_n \to x$.
        Let $y_n$ and  $y$ be points lying on the same trajectory of $X$ as  $x_n$ and  $x$.
        Suppose all $y_n$ lie on the same level as  $y$, i.e.  $f(y_n) = f(y)$.
        Then  $y_n \to  y$.
        }, implies that the entry points of $\ell_n$ in $\Omega(b)$ converge to the entry point of $\gamma_1$. This then proves that $\ell_n \to  \gamma_1$\sidenote{
        By definition of our topology, a sequence of unbroken trajectories connecting $a$ to  $b$ ($\ell_n$) converges to an unbroken trajectory ($\gamma_1$) if entry and exit points converge.
    }.

    In the other case , when $\gamma_1$ connects $a$ and another critical point $c \neq b$, the accumulation trajectory of $\ell_n$ will be a trajectory that is at least broken in $c$.
We have again that the entry points of $\ell_n$ in $\Omega(c)$ (call them $c_n^{+}$) converge to the entry point of $\gamma_1$, so the first segment of the accumulation trajectory will be $\gamma_1$.
To find the second segment of the accumulation trajectory, we would again want a starting point (like we had $a^{-}$ before) in the unstable manifold of $c$ to flow from in order to find $\gamma_{2}$.
For this, extract a subsequence of $\ell_n$ such that their exit points $c^{-}_n$ in $\Omega(c)$ converge to a point $c^{-}$.
We claim that $c^{-}$ lies in the unstable manifold of $c$.\sidenote{
    Suppose that it does not.
    Then we could flow back $c^{-}$ to a point $c^{\star}$ with $f(c^{\star}) = f(c^{+})$.
    Note that $c^{\star}$ is not in the stable manifold of $c$. (Otherwise, flowing forward again we would end up in  $c$)
    Applying the lemma again, $c_n^{+} \to  c^{\star}$ meaning that $c^{\star} = c^{+}$. This cannot be possibly right since $c^{\star}$ is not in the stable manifold of $c$, but $c^{+}$ is, by definition.
}
This means that when we flow back, we indeed get to $c$ like we wanted, and when we flow forward, we get to another critical point, which may be $b$, or $d$, yet another critical point. This way we go on and find a subsequence of $\ell_n$ that converges to $(\gamma_1, \gamma_2, \ldots, \gamma_k)$.

In order to complete the proof for a sequence $\ell_n$ in $ \Lb ab$ (instead of $\L ab$), note that for sufficiently large $n$ and after extracting a subsequence, the critical points where $ \ell_n$ is broken do not change. Then apply the proof above to the first segment, then to the second, \ldots
\end{proof}

% \todo{Introduce $\Omega(p)$ as notation for Morse chart around  $p$}


The lemma used in the above proof says that points $y_n$ on trajectories that pass through a convergent sequence of points $x_n \to x$ also converge, at least if $y_n$ all lie on the same level.
\begin{marginfigure}
    \centering
    \incfig{lemma-partial-squared-zero-proof}
    \caption{A convergent sequence $x_n \to  x$ defines a sequence of trajectories. If $y_n$ is a sequence of points that lie on these trajectories, then it also converges to a point $y$ lying on the trajectory that passes through $x$.}
    \label{fig:lemma-partial-squared-zero-proof}
\end{marginfigure}
\begin{lemma}
    Let $x$ be a regular point of $f$ and $x_n \to x$.
    Let $y_n$ and  $y$ be points lying on the same trajectory of $X$ as  $x_n$ and  $x$.
    Suppose all $y_n$ lie on the same level as  $y$, i.e.  $f(y_n) = f(y)$.
    Then  $y_n \to  y$.
    \label{lemma:level-sets}
\end{lemma}
\begin{proof}
    The idea of the proof is to flow $y_n$ to  $x_n$ and $y$ to $x$ so that convergence of $x_n$ implies convergence of $y_n$.
    Let  $\psi_t$ be the flow of  $-\frac{1}{df (X)} X$ on a subset of $M$ that contains $x_n, y_n, x, y$ for large enough  $n$ and does not contain critical points.
    Then $f(\psi_t(z)) = f(z) - t$, so
     \[
         y_n = \psi_{-f(y_n) + f(x_n)}(x_n) = \psi_{-f(y) + f(x_n)}(x_n) \xrightarrow{n \to \infty}   \psi_{-f(y) + f(x)}(x) = y
    .\] 
\end{proof}

Note that the proof of the previous theorem also shows that $\Lb ab $ is actually the compactification of  $\Lb ab$, in the sense that there are elements of  $\L ab$ that are arbitrarily close to a fixed element of $\Lb ab$.

\subsection{$\Lb ab$ is a 1-dimensional manifold with boundary}

The last thing we need to show is that the topological space $\Lb ab$ actually has the structure of a manifold:

\begin{theorem}
    Let $a, b$ be critical points of  $M$ such that  $\Ind a - \Ind b = 2$. Then  $\Lb ab$ is a compact  $1$-dimensional manifold with boundary.
\end{theorem}

\begin{remark}
    One can show that in general, when $\Ind a - \Ind b > 2$,  $\Lb ab$ is a compact manifold with corners.
\end{remark}

We already know that $\L ab$ is a 1-manifold, so the following proposition immediately implies the theorem.

\begin{prop}
    Let $M$ be a compact manifold and $f: M \to  \R$ a Morse function with adapted pseudo-gradient $X$ satisfying the Smale property. Let  $a,c,b$ be three critical points of indices  $k+1, k$ and $k-1$. Let $\lambda_1 \in \L ac$ and $\lambda_2 \in \L cb$.
    There exists a continuous embedding $\psi$ from $[0, \delta)$ to a neighborhood of $(\lambda_1, \lambda_2)$ in $\Lb ab$ such that
    \[
    \begin{cases}
        \psi(0) = (\lambda_1, \lambda_2) \in \Lb ab\\
        \psi(s) \in \L ab \quad \text{ for } s \neq 0.
    \end{cases}
    \] 
    Moreover if $(\ell_n)$ is a sequence in $\L ab$ that tends to $(\lambda_1, \lambda_2)$, then $ \ell_n$ is eventually contained in the image of $\psi$.
\end{prop}

The last part is important in order to show that broken trajectories actually form the boundary, as illustrated in Figure~\ref{fig:l-bar-last-condition-of-manifold-with-boundary}.
\begin{marginfigure}
    \centering
    \incfig{l-bar-last-condition-of-manifold-with-boundary}
    \caption{We can embed a half open interval $[a, b)$ in $\R$, but that does not mean that $a$ is a boundary point of $\R$. Clearly, there exists $x_n \to a$ that is not eventually contained in $[a,b)$.
        Requiring that this last condition always holds ensures that $a$ is actually a boundary point.
    }
    \label{fig:l-bar-last-condition-of-manifold-with-boundary}
\end{marginfigure}

\begin{proof}
    Let us first consider a $2$-dimensional manifold, where the only interesting case is $(\Ind a, \Ind c, \Ind b) = (2,1,0)$.

    \begin{marginfigure}
        \centering
        \incfig{l-bar-manifold-2-dimensional-case}
        \caption{The map $\chi$ is an embedding of a half-open interval $[0, \delta)$ in the level set $\alpha+ \epsilon$. Considering the trajectories passing through these points, we get an embdding $\psi: [0, \delta) \to \Lb ab$.}
        \label{fig:l-bar-manifold-2-dimensional-case}
    \end{marginfigure}
    Choose a Morse chart around $c$ that lies between level sets $\alpha \pm \epsilon$ where  $\alpha = f(c)$.
    The trajectories starting in $a$ meet $f^{-1}(\alpha + \epsilon)$ transversely.
    Let $P$ be their intersection, i.e. $\unstable{a} \cap  f^{-1}( \alpha + \epsilon)$.
    Let $c^{+}$ be the entry point of $\lambda_1$ in  the Morse chart of $c$, which is contained in the $1$-manifold $P$.

    The idea is that we will embed an interval $[0, \delta)$ as on Figure~\ref{fig:l-bar-manifold-2-dimensional-case} in the level set $\alpha + \epsilon$. Call this embedding $\chi$.
    The problem is however, we have to find a way to determine in what direction the second part of the broken trajectory ($\lambda_2$) exits $c$ so that we embed this interval on the `correct side' in order for the trajectories passing through these points to lie in a neighbourhood of $(\lambda_1, \lambda_2)$.
    In the figure, we need to embed it `on the right', because $\lambda_2$ exits on the right.

    The map $\psi:(0, \delta) \to  \L ab$ will then map $t$ to the trajectory passing through the point $\chi(t)$.
    Note that once again we have to be careful at the endpoint, but if we have found the `correct side', this will be no problem, and we will be able to extend $\psi$ to $[0, \delta)$ by mapping  $0 \mapsto (\lambda_1, \lambda_2)$.

    In order to solve the problem, we first consider `all directions', by embedding an open interval $D^{1}$ that is small enough \emph{centred} around $c^{+}$, i.e. a map $\Psi: (D^{1}, 0) \to  (P, c^{+})$.
    Now, flow $\psi(D^{1} \setminus \{0\} )$ along $X$ until it reaches the level set $\alpha - \epsilon$, as in Figure~\ref{fig:l-bar-manifold-2-dimensional-case-part-flow}.
    \begin{marginfigure}
        \centering
        \incfig{l-bar-manifold-2-dimensional-case-part-flow}
        \caption{An overview of the different submanifolds considered in the proof.}
        \label{fig:l-bar-manifold-2-dimensional-case-part-flow}
    \end{marginfigure}
    This defines an embedding $\Psi: D^{1} \setminus \{0\} \to  f^{-1}(\alpha-\epsilon)$, which splits up $D^{1}$ into two parts.
    If we add in $S^{-} := f^{-1}(\alpha-\epsilon) \cap \unstable{c}$, which in this case consists of two points, the result forms a one-dimensional manifold $Q$ (with boundary $S^{-}$).\sidenote{
        This is intuitively clear and can be verified by doing this in local coordinates, computing the flow explicitly.
        This also works in higher dimensions.
    }

    With this done, we are ready to determine the `correct direction'.
    Because of the Smale condition $\stable b \tcap \Im \Phi$ and  $\stable b \tcap S^{-}$.
    This means $\stable b \cap Q$ is a submanifold of dimension $1$ with boundary isomorphic to $\L cb$.
    In the figure, this submanifold is the right part of $Q$, with boundary the right point of $S^{-}$, a single point representing the only trajectory from $c$ to $b$, i.e. representative for $\L c b$.
    Now our problem is solved and we know the `right direction'.
    Let $c^{-}$ be the exit point of $\lambda_2$ (intersection of $\lambda_2$ and $S^{-}$). 
    Define a map $\chi:[0, \delta) \to  \stable b \cap  Q$ mapping $0$ to  $c^{-}$.\sidenote{TODO: this $\chi$ parametrizes a different manifold than before \ldots Fix this}
    Similarly as before by applying the previous lemma, this defines a map $\psi: [0, \delta) \to  \Lb ab$ with $\psi(0) = (\lambda_1, \lambda_2)$.

    For the last statement in the theorem, let $\ell_n \to  (\lambda_1, \lambda_2)$ be a sequence in $\L ab$. Let $ \ell^{\pm}_n$ denote entry and exit points. 
    Eventually, $\ell_n^{+} \in \psi(D \setminus \{0\})$, hence $ \ell_n^{-} \in Q$. This implies that $ \ell_n^{-} \in Q \cap \stable b = \Im \chi$, i.e.\ in the image of the embedding of the half open interval $[0, \delta)$.
    This proves that $ \ell_n \in \Im \psi$.

    In higher dimensions, the proof is actually almost exactly the same.
    While there are many more directions from which to enter or leave the Morse chart of $c$, the procedure outlined above works, with a few adjustments to be made.
    The dimension of $P$ is $k$, which we parametrize around $c^{+}$ by $D^{k}$ instead of $D^{1}$. Flowing this along $X$ and adding $S^{+}$ we get a $k$-dimensional manifold with boundary.\sidenote{This time, the manifold does not split into two parts, but becomes an annulus} Taking the intersection with  $\stable b$, we get a one-dimensional manifold (by counting dimensions and using that codimensions add when intersection is transverse), and the proof finishes in the same way.
    In Figure~\ref{fig:lbar-manifold-three-dimensional-case} we have illustrated the three dimensional case with $\Ind c=2$.

\begin{marginfigure}
    \centering
    \incfig{lbar-manifold-three-dimensional-case}
    \caption{The situation in three dimensions.}
    \label{fig:lbar-manifold-three-dimensional-case}
\end{marginfigure}
\end{proof}

\todo{lots of greek letters!!}
\subsection{Conclusion}

\todo{again calculate the differential}
\todo{ differential $\partial^2$ over $\Z$ more in detail}


% \filbreak
\section{Morse homology is independent of the Morse function and gradient}
\begin{theorem}
    Let $M$ be a compact manifold and $ f_0, f_1: M \to  \R$ two Morse functions.
    Let $X_0, X_1$ be pseudo-gradients adapted to $f_0$ and $ f_1$ respectively with the Smale property.
    Then there exists a morphism of complexes
    \[
        \Phi_{\bul}:
        (C_\bul(f_0), \partial_{X_0}) \to  
        (C_\bul(f_1), \partial_{X_1})
    ,\] 
    that induces an isomorphism on the level of homology.
\end{theorem}
    
\begin{proof}
The proof of this theorem is truly something to behold: it features an intricate interplay between homological algebra and differential geometry.
\paragraph{Construction of a morphism of complexes}
In order to find a relation between the two complexes, we geometrically connect $f_0$ and $f_1$ via a particular type of homotopy, namely  a `stable interpolation' by which we mean a smooth map
\[
    F: [0,1] \times M \to \R: (s, m) \mapsto F_s(m)
,\] 
such that $F_s = f_0$ for $s \in \left[0, \frac{1}{3}\right]$ and $F_s = f_1$ for $s \in \left[\frac{2}{3}, 1\right]$.
On of the reasons of looking specifically at stable interpolations is that we can concatenate them and again get a $C^{\infty}$ map that is a stable interpolation.

We can visualize an interpolation between two morse functions by embedding $[0,1] \times M$ in $\R^{n+1}$ in such a way that the height function in each slice corresponds to $F_s$.
For example, doing this for the circle and the other circle, we get Figure~\ref{fig:morse-homology-independence-cilinder}.

\begin{marginfigure}
    \centering
    \incfig{morse-homology-independence-cilinder}
    \caption{An interpolation between $f_0$ and $f_1$ can result in degenerate critical points, as shown in the figure in orange:
        a homotopy between Morse functions is not necessarily Morse for all times $s$.
    }
    \label{fig:morse-homology-independence-cilinder}
\end{marginfigure}

Seen from a Morse perspective, the result is less than desirable: the function $F$ is not a Morse function: critical points in the stationary parts of $F$ are degenerate as $\frac{\partial F}{\partial s} = 0$.
Furthermore, an interpolation of two Morse functions need not to be Morse at each point in time which gives even more degenerate critical points. We have highlighted an example of this in the figure.

\begin{marginfigure}
    \centering
    \incfig{morse-homology-independence-tube}
    \caption{Adding a function (illustrated below) in the $s$-direction creates a slide overcoming the problem of degenerate critical points.}
    \label{fig:morse-homology-independence-tube}
\end{marginfigure}

\begin{marginfigure}
    \centering
    \incfig{morse-homology-independence-g-function}
    \caption{The function $g$ used to transform the tube into a slide.}
    \label{fig:morse-homology-independence-g-function}
\end{marginfigure}

We can fix this problem by replacing the horizontal tube by a `slide', as seen in Figure~\ref{fig:morse-homology-independence-tube}.
We do this by extending $F$ to $[-\frac{1}{3}, \frac{3}{4}]$ and adding a function $g$ (illustrated in Figure~\ref{fig:morse-homology-independence-g-function}) along the $s$-direction, i.e.\ $\tilde{F}_s(p) = F_s(p) + g(s)$.\sidenote{We extend the function such that the critical values of $\tilde{F}$ do not lie on the boundary.}
Whatever kind of tube we start with, if we make the slide steep enough, we will always slide down and never have flat spots, except at the top and bottom of the slide.
This means if we choose $g$ appropriately\sidenote{
    More explicitly, we want $\frac{\partial F}{\partial s} (p, s) + g'(s) <0$ for all $p \in M, s \in (0,1)$.
}, the only critical points lie in the slices $s=0$ and  $s=1$ and correspond to critical points of $f_0$ and $f_1$ respectively.
Because $g$ is Morse, these critical points remain nondegenerate.
We conclude that $\tilde{F}$ is in fact Morse with critical points $\Crit(\tilde{F}) = \{0\} \times \Crit(f_0) \cup \{1\} \times \Crit(f_1)$.

We can also determine the index of these critical points.
Because we have created an extra downward direction at the top of the slide, the indices of these critical points have increased by $1$.
At the bottom, the indices stay the same, giving us
\[
    C_{k+1}(\tilde{F}) = C_k(f_0) \oplus C_{k+1}(f_1)
.\] 

Apart from the critical points, we are also interested in constructing a pseudo-gradient on $[0,1]\times M$, as this will give rise to a differential.
 On $[-\frac{1}{3}, \frac{1}{3}] \times M$, we set $X = X_0 - \grad g$, and on $\left[\frac{2}{3}, \frac{4}{3}\right] \times M$ we set $X = X_1 - \grad g$.\sidenote{Here, $\grad g$ is the Euclidian gradient}
A partition of unity argument then fills in the gaps.
Note that this pseudo-gradient is transverse to the boundary of $\left[-\frac{1}{3}, \frac{4}{3}\right]$.
We can slightly perturb $X$ to make it satisfy the Smale condition and we can furthermore assume that the resulting vector field, $\tilde{X}$ is transversal to $ \{s\} \times M$ for $s \in \left\{-\frac{1}{3}, \frac{1}{3}, \frac{2}{3}, \frac{4}{3}\right\}$.
We can also make this perturbation small enough such that $\partial_{X} = \partial_{\tilde{X}}$, that is to say, the number of $X$-trajectories between critical points is the same as the number of $\tilde{X}$-trajectories.

Having a Morse function $\tilde{F}$ and a pseudo-gadient $\tilde{X}$ that is adapted to $\tilde{F}$, we can consider the associated Morse complex $(C_\bul(\tilde{F}), \partial_{\tilde{X}})$.
There are two types of trajectories connecting critical points of $\tilde{F}$: ones that stay in the same section ($s = 0$ or $s = 1$)  and ones that connect critical points of $f_0$ to critical points of $f_1$, i.e. ones that `slide down the slide'.
This means we can decompose $\partial_{\tilde{X}}$ as follows:
\begin{align*}
    \partial_{\tilde{X}}: C_k(f_0) \oplus C_{k+1}(f_1) &\longrightarrow C_{k-1}(f_0) \oplus C_k(f_1) \\
    (p_0, p_1) &\longmapsto (\partial_{X_0}(p_0), \partial_{X_1}(p_1)+ \Phi^{F}(p_0))
,\end{align*}
where $\Phi^{F}$ counts the trajectories connecting critical points of $f_0$ to ones of $f_1$. We can also write this as a matrix:
\[
\partial_{\tilde{X}} = \begin{pmatrix}
    \partial_{X_0} & 0 \\
     \Phi^{F}& \partial_{X_1}
\end{pmatrix}
.\] 
Readers familiar with homological algebra will recognize this construction as the mapping cone of the map $\Phi^{F}: C_\bul(f_0) \to  C_\bul(f_1)$.

\begin{marginfigure}
    \centering
    \incfig{morse-homology-independence-partial-squared-zero}
    \caption{A visual depiction of the calculation $\partial^2(p)$.}
    \label{fig:morse-homology-independence-partial-squared-zero}
\end{marginfigure}

Let us now look at what the identity $\partial_{\tilde{X}}^2 = 0$ means in this context. Let $p \in C_k(f_0)$.
Then
\begin{align*}
    \partial_{\tilde{X}}^2 (p, 0) &= \partial_{\tilde{X}}(\partial_{0}(p),  \Phi^{F}(p))\\
                                                      &= (\partial_0^2(p), \Phi^{F}\partial_0(p)+ \partial_1 \Phi^{F}(p))\\
                                                    &= (0, \Phi^{F}\partial_0(p)+ \partial_1 \Phi^{F}(p))
.\end{align*} 
Because we are working over $\Z_2$, this means that $\Phi^{F}  \circ  \partial_0 = \partial_1  \circ  \Phi^{F}$, i.e.\ the following diagram commutes for all $k$:
\[
    \begin{tikzcd}
        C_k(f_0) \arrow[d, "\Phi^{F}"]\arrow[r, "\partial_0"] &C_{k-1}(f_0) \arrow[d, "\Phi^{F}"]\\
        C_k(f_1) \arrow[r, "\partial_1"] &C_{k-1}(f_1)\\
    \end{tikzcd}
\]
This proves that $\Phi^{F}$ is a morphism of complexes.

\paragraph{$\Phi^{F}$ induces an isomorphism on the level on homology}

We will now prove that this map induces an isomorphism on the level of homology.
Let $f_0, f_1, f_2$ be Morse functions $M \to  \R$.
Suppose $F$ interpolates between $f_0$ and $f_1$,
$G$ between $f_1$ and $f_2$ and $H$ between $f_0$ and $f_1$, i.e.\ we are in the following situation:
\[
    \begin{tikzcd}
        f_0 \arrow[rr, "H"', bend right] \arrow[r, "F"] &f_1 \arrow[r, "G"] &f_2.
    \end{tikzcd}
\]
We claim that the maps induced by $\Phi^{G} \circ \Phi^{F}$ and $\Phi^{H}$ on the level of homology coincide, or equivalently, they are chain homotopic, meaning that there exists an operator $S$ such that
 \[
\Phi^{G}  \circ  \Phi^{F} - \Phi^{H} = \partial S + S \partial
.\] 
This is sufficient to prove that $\Phi^{F}$ induces an isomorphism.
Indeed, it is easy to check that if $I$ is a constant interpolation between  $(f_0, X_0)$ and itself, 
\[
    I: [0,1] \times M \to  \R: (s, p) \mapsto f_0(m),
\] 
then $\Phi^{I} = \operatorname{Id}$.
So consider $F$ a stationary interpolation between $f_0$ and $f_1$ and $G$, the reverse interpolation from $f_1$ to $f_0$ and $H = I$. Then the induced homological maps $\Phi^{F}$ and $\Phi^{G}$ are inverses of each other.

Let us prove that $\Phi^{G}  \circ  \Phi^{F}$ and $\Phi^{H}$ are chain homotopic.
The idea of this part of the proof is very similar to the first part.
Instead creating one slide from $f_0$ to $f_1$ by adding an extra dimension, we create a two dimensional slide with as sides four slides: $f_0 \xrightarrow{F} f_1$, $ f_1 \xrightarrow{G} f_2$, $f_0 \xrightarrow{H}  f_2$ and $ f_2 \xrightarrow{I} f_2$.


More concretely, we create a map
\[
    K: \left[-\tfrac{1}{3}, \tfrac{4}{3}\right] \times \left[-\tfrac{1}{3}, \tfrac{4}{3}\right] \times M \to  \R:  (s, t, p) \mapsto K_{s, t}(p)
,\] 
with the following properties, as illustrated in Figure~\ref{fig:morse-homology-independence-two-dimensional-slide}:
\begin{itemize}
    \item $K_{s,t} = H_t$ for  $s \in \left[-\tfrac{1}{3}, \tfrac{1}{3}\right]$
    \item $K_{s,t} = G_t$ for  $s \in \left[\tfrac{2}{3}, \tfrac{4}{3}\right]$
        \item $K_{s, t} = F_s$ for $t \in \left[-\tfrac{1}{3}, \tfrac{1}{3}\right]$ 
        \item $K_{s,t} = f_2$ for $t \in \left[\tfrac{2}{3}, \tfrac{4}{3}\right]$
\end{itemize}
\begin{marginfigure}
    \centering
    \incfig{morse-homology-independence-two-dimensional-slide}
    \caption{The map $K_{s,t}$ is a two-dimensional homotopy between $f_0, f_1, f_2, f_2$.}
    \label{fig:morse-homology-independence-two-dimensional-slide}
\end{marginfigure}
Note that these properties are not contradictory because we are working with stationary interpolations.

Now, to make a slide, we modify $K$ as follows:
 \[
     \tilde{K}_{s,t}(p) = K_{s,t}(p) + g(s) + g(t)
,\] 

\begin{marginfigure}
    \centering
    \incfig{morse-homology-independence-two-dimensional-slide-three-d}
    \caption{By adding the slide function $g$ in $s$- and $t$-directions, we create a two-dimensional slide, eliminating the possibility of degenerate critical points.}
    \label{fig:morse-homology-independence-two-dimensional-slide-three-d}
\end{marginfigure}

with $g$ defined similarly as before, making $\tilde{K}$ a Morse function with critical points in the yellow regions in the figure.
The points correspond to critical points of $f_0$, $f_1$, $f_2$ and $f_2$ with indices raised by $2, 1, 1, 0$ respectively.
Also similarly as before, we can construct a pseudo-gradient vector field $X$ adapted to $\tilde{K}$, by adding $-\grad g(s)$, $-\grad g(t)$ at the appropriate regions and perturbing it in order to have the Smale property, again making sure that the perturbation is small enough such that $\partial_X = \partial_{\tilde X}$, where $\tilde{X}$ is the perturbed vector field.

While the resulting manifold $[-\frac{1}{3}, \frac{4}{3}]^2 \times M$ does not have a smooth boundary, the conditions (TODO add conditions) are still satisfied.
In summary, we have
\[
    C_{k+1}(\tilde{K}) = C_{k-1}(f_0) \oplus C_k(f_1) \oplus C_k(f_2) \oplus C_{k+1}(f_2)
,\] 
and the differential can be written as
\[
\partial_{\tilde{X}} = \begin{pmatrix}
    \partial_0 & 0 & 0 & 0\\
    \Phi^{F} &\partial_1 & 0 & 0 \\
    \Phi^{H} & 0 & \partial_2 & 0 \\
    S & \Phi^{G} & \operatorname{Id} & \partial_2
\end{pmatrix}
.\] 
Now, computing $\partial_{\tilde{X}}^2(p, 0, 0, 0)$ is like water trickling down four spillway bowls, as illustrated in Figure~\ref{fig:morse-homology-independence-partial-squared-zero-bis}. We get that
\[
\Phi^{G}  \circ  \Phi^{F} + \Phi^{H} + S \partial_0 + \partial_1 S = 0
,\] 
or as we are working over $\Z_2$,
\[
    \Phi^{G}  \circ \Phi^{F} - \Phi^{H} = S \partial_0 + \partial_1 S
,\] 
proving that $\Phi^{G}  \circ \Phi^{F}$ and $\Phi^{H}$ induce the same map on the level of homology.

\begin{figure*}
    \centering
    \sidecaption{Calculating $\partial^2(p)$.
        On the left we $\partial(p)$ is illustrated, and on the right  $\partial^2(p)$.
        Because $\partial^2 = 0$, we find that $\Phi^{G}  \circ  \Phi^{F} + \Phi^{H} + S \partial_0 + \partial_1 S = 0$.
        Considering this over $\Z_2$ implies that $\Phi^{G}  \circ \Phi^{F}$ and $\Phi^{H}$ are chain homotopic.
        \label{fig:morse-homology-independence-partial-squared-zero-bis}
    }
    \vspace*{2cm}
\fullwidthincfig{morse-homology-independence-partial-squared-zero-bis}
\end{figure*}
\end{proof}


\section{Morse homology over $\Z$}
In this section, we will defined Morse homology with coefficients in $\Z$.
As is to be expected, this homology theory will be less coarse than the one over $\Z_2$. For example, as we will show next, Morse homology over $\Z_2$ cannot distinguish a torus from a Klein bottle, while homology over $\Z$ can.

The main difficulty in defining homology over $\Z$ is keeping track of orientations.
Similarly as before, we define
    \[
        C_k(f, \Z) = \Big\{ \sum_{p \in \Crit_k f} n_p p  \mid n_p \in \Z \Big\}  = \bigoplus_{p \in \Crit_k f} \Z p
    ,\] 
    with differentials
    \begin{align*}
        \partial_{X, k}: C_k &\longrightarrow C_{k-1} \\
        p &\longmapsto \sum_{q \in \Crit_{k-1} f} \NX Xpq q
    ,\end{align*}
    where $\NX X pq$ is the \emph{signed} number of trajectories between  $p$ and $q$.
    The fact $\partial^2 = 0$ follows from a similar reasoning as before, now depending on the fact the number of signed boundary points of a 1-manifold is zero.\todo{$\partial^2 = 0$ more explicitly}
    The homology of the complex $C_\bullet$ is called the integral Morse homology, and the fact that it is independent of $f$ and $X$ also follows from the reasoning as before, taking in account the signs of the trajectories.

    The signed number of trajectories $\NX pq$ between $p$ and $q$ is defined by orienting  $\L pq$. Since  $\L pq$ is a zero-dimensional manifold if  $\Ind p - \Ind q = 1$, this corresponds to assigning a sign, defining  $\NX Xpq$.
    For this, first choose an orientation for each stable manifold $\stable c$.\sidenote{This is possible since stable manifolds are diffeomorphic to open disks.}
    Let $x \in \traj p q$.
    Then, because $\unstable p \tcap \stable q$,
    \[
    T_x \stable q = T_x \traj pq \oplus N_x \unstable p  \cong
    T_x \traj pq \oplus T_x \stable p \quad \sidenotemark    \] 
\sidenotetext{
        The orientation on $\stable p$ coorients  $T_p \unstable p$, and because contractibility of the unstable manifolds, this coorients $\unstable p$ everywhere.
    }
    which defines an orientation on $T_x \traj p q$ as follows: a basis $v_i$ of $T_x \traj pq $ is said to be positively oriented if $(v_1, \ldots, v_\ell, w_1, \ldots, w_k)$ is a positive basis of  $T_x \stable q$, where $w_i$ is a positively oriented basis for $T_x \stable p$.
    Now $\traj pq = \R \oplus \L pq$, so by choosing an orientation of $\R$, which represents time, this defines in a similar way an orientation on $\L pq$.
    \begin{eg}
        As an example, let us consider the other sphere and more specifically $\L pq$ with $p$ and  $q$ as in Figure~\ref{fig:orientation-example} below. 
        We start off with an orientation of $\stable q$ and  $\stable p$.
        To orient  $\traj pq$, we use the orientation of  $\stable p$, which coorients $\traj pq$, indicated in the figure with the thick black horizontal arrows.
        The orientation on $\traj pq$, given by the dashed arrow, is then defined by requiring that the vectors form a positively oriented basis for $\stable q$. (TODO wording)
        Finally, the orientation on $\L pq$ is defined as follows: if the arrows are going up (i.e. in the negative time direction), the sign is positive, else it is negative.
        \begin{figure}[H]
    \centering
    \incfig{orientation-example}
    \caption{
        Orienting $\L pq$ on the other sphere.
    }
    \label{fig:orientation-example}
\end{figure}
    \end{eg}

    \begin{remark}
        While we are free to choose the orientations of $\stable p$, reversing it only changes the sign of $\NX X pq$ and  $\NX X qp$ for any $q$, implying that the Morse homology is independent of the choice of orientation.
    \end{remark}

    To show that the homology over $\Z$ is -- as one would expect -- less coarse than the one over $\Z_2$, we will compute $\HMf{T^2, \Z_2}$, $\HMf{K, \Z_2}$
    $\HMf{T^2, \Z}$ and $\HMf{K, \Z}$.



\begin{eg}[Homology of $T^2$ over  $\Z_2$]
    \begin{marginfigure}
        \centering
        \incfig{tilted-torus}
        \caption{The height function on a tilted torus is a Morse function giving rise to the illustrated flow lines.
            On the right, an abstract depiction of the critical points and the signed flow lines connecting them.
        }

        \label{fig:tilted-torus}
    \end{marginfigure}
    We use the Morse function of Example~\ref{eg:tilted-torus}, where the torus is slightly tilted, as illustrated in Figure~\ref{fig:tilted-torus}, where we also added the graph of the critical points.
    Remember that each edge represents a trajectory connecting two critical points of consecutive index.
    The complex is given by $ \Z_2 \to  \Z_2^2 \to \Z_2$, and because each critical point is connected twice to any other critical point of consecutive index, $\partial_k = 0$ for all $k$. 
    This means that we have
    \[
        \HMf[0]{T^2, \Z_2} = \Z_2 \qquad
        \HMf[1]{T^2, \Z_2} = \Z_2^2 \qquad
        \HMf[2]{T^2, \Z_2} = \Z_2
    .\] 
\end{eg}
\begin{eg}[Homology of the Klein bottle over $\Z_2$]
    \begin{marginfigure}
        \centering
        \incfig{tilted-klein-bottle-over-z}
        \caption{The height function of a tilted Klein bottle immersed in $\R^3$ is a Morse function.
            We have illustrated the flow lines connecting critical points.
            Not considering signs, we get the same complex as for the torus. If we do consider signs, we can distinguish one from another.
        }
        \label{fig:tilted-klein-bottle-over-z}
    \end{marginfigure}
    The Klein bottle $K$ is a non-orientable surface. We cannot embed it in $\R^3$, but the strong version Withney's theorem shows that we are able to immerse it, which is what we have done in Figure~\ref{fig:tilted-klein-bottle-over-z}.
    If we tilt the bottle, the height function $h$ is Morse\sidenote{If we did not, we could have a circle of critical points near the bottom of the bottle.}, and the gradient induced by the standard metric on $\R^3$ is adapted to $h$ and satisfies the Smale condition.
    We have also included the graph of the critical points.
    Disregarding sign differences (which will be important when discussing the complex over $\Z$), this graph is identical to the one we obtained for $T^2$. We conclude that the Morse homology of $K$ and  $T^2$ over $\Z_2$ are identical:
    \[
        \HMf[0]{K, \Z_2} = \Z_2 \qquad
        \HMf[1]{K, \Z_2} = \Z_2^2 \qquad
        \HMf[2]{K, \Z_2} = \Z_2
    .\] 
\end{eg}

\begin{eg}[Homology of $T^2$ over $\Z$]
    We use the Morse function of Example~\ref{eg:tilted-torus}, where the torus is slightly tilted, as illustrated in Figure~\ref{fig:tilted-torus}.
    We have indicated the chosen orientations as before and have assigned a positive orientation to the single point $\stable d = \{d\}$.
    For each trajectory, the sign has been added, both on the figure and on the graph.

    The complex is given by $\Z \xrightarrow{\partial} \Z^2 \xrightarrow{\partial}  \Z$, and because the signs of the trajectories cancel out, $\NX Xpq = 0$ for all $p$ and $q$, so each differential is zero.
    In the end, we have
    \[
        \HMf[0]{T^2, \Z} = \Z \qquad
        \HMf[1]{T^2, \Z} = \Z^2 \qquad
        \HMf[2]{T^2, \Z} = \Z
    .\] 
\end{eg}
\begin{eg}
    We have done the same for the Klein bottle, but notice that this time, the signs do not cancel out.
    It is easy to check that $\partial_1 = 0$ and that  $\partial_2$ is defined by  $\partial_2(\ell d) = -2\ell b$, where $a,b,c,d$ are the critical points of $h$ with increasing height.
    This means that $\Im \partial_2 = 2 \Z$. Summarizing, we have
    \[
        \HMf[0]{K, \Z} = \Z \qquad
        \HMf[1]{K, \Z} = \Z \oplus \Z_2 \qquad
        \HMf[2]{K, \Z} = 0
    .\] 
    The conclusion of this series of examples is that integral homology is less coarse than homology over $\Z_2$.
\end{eg}

\section{Morse homology is singular homology}

In the final section of this Chapter, we prove that Morse homology is isomorphic to singular homology.
There are many ways to go about this.
Some authors\sidecite[p.110]{audin} prove it by proving that Morse homology is isomorphic to cellular homology, which is in itself isomorphic to singular homology.
This approach consists of two steps: first proving that unstable manifolds form a cellular decomposition of $M$, and second: proving that the corresponding map is a chain map isomorphism.
While the first step is intuitive, the proof of it is actually very technical.
The second step on the other hand, is the easy part of the proof.

To mitigate these technical difficulties, many other authors\sidenote{
    \fullcite[p. 195]{banyaga2013lectures}\\[0.3em]
    \fullcite[p. 13]{hutchings2002lecture}\\[0.3em]
\fullcite[p. 66]{abbondandolo2006lectures}} follow a different approach, based on cellular filtrations, currents and other techniques.
Here we opt for the proof by Hutchings based on currents.

For a thorough introduction on currents, see `Differentiable manifolds' by de Rham.\sidecite{de2012differentiable}

\begin{definition}[Current]
    A current is a continuous linear functional on the space of compactly supported $k$-forms on a manifold $M$.
\end{definition}
\begin{remark}
    One should think of currents like distributions on manifolds. For example, the Dirac delta distribution is a current acting on $0$-forms on $\R$ as follows: $f \mapsto  f(0)$, where $f$ is a $0$-form, i.e. a function.
\end{remark}
\begin{eg}
    Any compact submanifold (with boundary) $M$ of dimension $n$ defines a current $[M]$ on $n$-forms in the following way:
    \[
        [M](\omega) := \int_M \omega
    .\] 
    For two disjoint submanifolds $M, N$ we have
    \[
        [M \sqcup N] = [M] + [N]
    .\] 
    Notice that by Stokes' theorem we have
    \[
        [\partial M](\omega) = \int_{\partial M} \omega = \int_M d \omega = [M](d \omega).
    \] 
    This motivates the following definition:
\end{eg}
\begin{definition}
    We can define a differential on the space of currents by defining
    \[
        (\partial T) (\omega) := T(d \omega)
    .\] 
    When $T = [M]$, this can be written as $\partial [M] = [\partial M]$. 
\end{definition}
This differential clearly squares to zero, and hence defines a complex.
It turns out that its homology is actually isomorphic to singular homology.\sidenote{
    See \fullcite[p.~89, Theorem~16]{de2012differentiable}, or \fullcite[p.~582, Theorem~2]{giaquinta1998cartesian}
}
With this set up, we are ready to prove the following theorem:

\begin{theorem}
    Let $M$ be a closed manifold. Then
     \[
         \HMf[\bul]{M, \Z} = H_{\bul}(M, \Z)
    ,\] 
    where $H_{\bul}(M, \Z)$ denotes the singular homology of $M$.
\end{theorem}

\begin{proof}
    As stated before, we follow Hutchings\sidecite{hutchings2002lecture}.
    \paragraph{Idea of the proof}
    
    We define two chain maps:
    \begin{alignat*}{5}
        D&: C_\bul^{\text{Morse}}(f) &&\to  C_\bul(M)&&: \text{critical point} &&\mapsto &&\ [\text{compactification of $\unstable{c}$}]\\
        A&: C_\bul(M) &&\to  C_\bul^{\text{Morse}}(f)&&: \text{generic simplex} &&\mapsto &&\ \parbox[t][][t]{5cm}{sum of critical points on which the simplex hangs on by flowing via $X$.}
    \end{alignat*}
\begin{marginfigure}
    \centering
    \incfig{idea-of-the-proof-maps-a-and-d}
    \caption{The map $D$ is defined by mapping a critical point to the current of a compactification of $\unstable c$. The map $A$ maps a generic simplex to the critical points it `hangs on'.}
    \label{fig:idea-of-the-proof-maps-a-and-d}
\end{marginfigure}

    Here we define $C_i(M)$ as the subspace of all  $i$-dimensional currents on $M$ generated by \emph{generic} $i$-simplices, by which we mean simplices that are smooth and whose faces are transverse to the stable manifolds of all critical points.


    Then $A  \circ D$ is the identity, and while $D  \circ  A$ is not, it is chain homotopy to the identity.
    The chain homotopy sends a singular chain to its entire forward orbit under the flow of $X$.
    This proves the theorem.

    In summary, we will prove the theorem in $7$ steps:
    \begin{enumerate}
        \item Compactification of $\unstable c$
        \item Definition of $D$
        \item $D$ is a chain map
        \item Definition of $A$
        \item $A$ is a chain map
        \item $A  \circ D = \operatorname{Id}$
        \item  $D  \circ A \cong \operatorname{Id}$
    \end{enumerate}

    \paragraph{1. Compactification of $\bm{\unstable c}$}
    We can compactify $\unstable c$ into a manifold with corners as follows:
     \[
         \unstableb c = \unstable c \cup  \bigcup_{d \neq c} \Lb cd \times  \unstable d
    .\] 
    We have given some examples in Figure~\ref{fig:compactification-of-unstable-manifolds} which make it clear that this compactification possibly differs from the reader's expectations. 
    For example, on top the compactification becomes a closed interval, and not a circle, even though the two end points map to the same point in $M$.
    \begin{marginfigure}
        \centering
        \incfig{compactification-of-unstable-manifolds}
        \caption{
            Two examples of compactifications of unstable manifolds in the other sphere.
            On top we consider the index 1 critical point and on the bottom the index 2 critical point.
        Note that the compactifications are subtle  and in particular are not diffeomorphic to $S^{1}$ and $B^{2}$ resp.}
        
        \label{fig:compactification-of-unstable-manifolds}
    \end{marginfigure}
    
    As an oriented manifold, its codimension-$1$ stratum\sidenote{
        The codimension $k$ stratum of a manifold with corners $M$ is the set of points $p$ in  $M$ such that there exists a chart  $f : U(p) \to \R^{n-k} \times [0, \infty)^{k}$ such that at least one of the last $k$ coordinates of $p$ is zero.
        The codimension $0$ stratum is the interior of $M$, the codimension $1$ stratum is its boundary, without the `higher order' corners, etc.
    } is given by
    \[
        \partial \unstableb c = \bigcup_{d \neq c}  (-1)^{\Ind d + \Ind c + 1} \Lb cd \times  \unstable d.
    \]

    \paragraph{2. Defining $\bm{D}$}
    Let $e: \unstableb c \to  M$ be the inclusion extending the inclusion of $\unstable c \to  M$.\sidenote{Note that by previous remarks, this extension is not necessarily injective}
    Then we define the current $D(c) := e_* \left[\unstableb c\right]$, i.e.\ integration over $\unstableb c$:
    \[
        D(c)(\omega) = \int_{\unstableb c} e^{*} \omega
    .\] 
    This current is an element of $C_{\bul}(M)$ because of the Smale condition.

    \paragraph{3. The map $\bm{D}$ is a chain map:  $\bm{\partial D = D \partial^{\text{Morse}}}$ }
    We have

    \[
        \partial \unstableb c = \bigcup_{d \neq c}  (-1)^{\Ind d + \Ind c + 1} \Lb cd \times  \unstable d,
    \]
    \begin{marginfigure}
        \centering
        \incfig{d-is-a-chain-map}
        \caption{An example illustrating that $D$ is a chain map.}
        \label{fig:d-is-a-chain-map}
    \end{marginfigure}
    which using the fact that $[M \sqcup N] = [M] + [N]$ implies that
    \[
        \partial D(c) = \sum_{d \neq c} (-1)^{\Ind d + \Ind c + 1} e_*\left[\Lb cd \times  \unstable d\right]
    .\] 
    We have three cases to consider based on the index of $d$:
    \begin{description}
        \item[$\Ind d > \Ind c-1$]: Then $\Lb c d = \O$ by the Smale condition, so these terms vanish.
        \item[$\Ind d = \Ind c-1$]: We get a term of the form \[
            e_* \left[\Lb cd \times \unstableb d\right] = \# \L cd \cdot e_*[\unstable d],\]
            because $\L cd = \Lb cd$ is  $0$-dimensional.
        \item[$\Ind d < \Ind -1$]: In this case, the term corresponds to a current of dimension less than $\Ind c - 2$, hence zero in $C_{\Ind c - 1}(M)$.
    \end{description}
    Summarizing, we have
    \begin{align*}
        \partial D(c) &= \sum_{d \in \Crit_{\Ind c - 1} f} \# \L cd \cdot e_* \left[ \unstableb c \right] \\
                      &= D(\partial^{\text{Morse}}(c))
    .\end{align*} 

    \paragraph{4. Defining $\bm{A}$}

    The map $A$ is defined by mapping a generic simplex $\sigma$ to the sum of critical points on which the simplex hangs on by flowing via $X$.
    Some examples of this vague definition are illustrated in Figure~\ref{fig:morse-homology-is-singular-homology-defining-map-a}.
    \begin{marginfigure}
        \centering
        \incfig{morse-homology-is-singular-homology-defining-map-a}
        \caption{
            Examples illustrating the definition of $A$.
            From left to right, a $2$, $1$ and  $0$ simplex $\sigma$ and the resulting critical point $A(\sigma)$ indicated in orange.
        }
        \label{fig:morse-homology-is-singular-homology-defining-map-a}
    \end{marginfigure}
    More rigorously, we denote with $\L \sigma d$ the moduli space of gradient flow lines starting in the $i$-simplex $\sigma$ and ending in $d$.
    This space has a natural orientation and further more a compactification $\Lb \sigma d$, such that
    \[
    \partial \Lb \sigma d = \Lb {\partial \sigma} d \cup \bigcup_{c \neq d}  (-1)^{i + \Ind d} \L \sigma c \times \L c d
    .\] 
    The dimension of this manifold is $i - \Ind d$, which means that if  $\Ind d = i$,  $\Lb \sigma p = \L \sigma p$ consists of a finite number of points (with signs if we are considering orientation).
    The map $A$ is then defined as
    \[
        A(\sigma) = \sum_{p \in \Crit_i f} \# \L \sigma p p
    ,\] 
    which corresponds to the intuitive definition given earlier.


    \paragraph{5. The map  $\bm A$ is a chain map:  $\bm{A \partial = \partial^{\text{Morse}} A}$ }
    We will show that $A \partial = \partial^{\text{Morse}} A$ component by component. To this purpose, we introduce the inner product on the space of critical points, defined on a basis as follows:
    \[
        \left<c, d \right> = \begin{cases}
            1 & \text{if $c = d$}\\
            0 & \text{else.}
        \end{cases}
    \] 
\begin{marginfigure}
    \centering
    \incfig{a-is-a-chain-map}
    \caption{An example where $\sigma$ is a $1$-simplex illustrating that  $A$ is a chain map.}
    \label{fig:a-is-a-chain-map}
\end{marginfigure}

    Let $\sigma$ be a  $i$-simplex and $d \in \Crit_{i-1} f$.
    Then $\dim \Lb \sigma d = 1$, so  $\# \partial \Lb{\sigma}{d} = 0$, with signs taken into account.
    This implies that
    \begin{align*}
        \left<A(\partial \sigma), d \right> - \left<\partial^{\text{Morse}} A (\sigma) ,d\right> &= \# \Lb {\partial \sigma} d - \# \bigcup_{c \in  \Crit_i f} \L \sigma c \times  \L c d\\
                                                                                                 &= \# \partial \Lb \sigma d = 0
    ,\end{align*} 
    showing that $A \partial = \partial^{\text{Morse}} A$.


\paragraph{6. The map $\bm{A}$ is left inverse to $\bm{D}$: $\bm{A  \circ D = \text{Id}}$}
    This immediately follows from the definitions of $A$ and $D$.
    $\L {D(c)} c = \{c\}$ and $\L {D(c)} d = \O$ for any other critical point $d$.
\begin{marginfigure}
    \centering
    \incfig{a-is-left-inverse-to-d}
    \caption{The map $A$ forms a left inverse to $D$}
    \label{fig:a-is-left-inverse-to-d}
\end{marginfigure}


\paragraph{7. The composition $\bm{D  \circ A}$ is chain homotopic to Id}
While clearly $D  \circ  A \neq \operatorname{Id}$, these two maps are chain map homotopic, meaning that there exists a map $F: C_i(M) \to  C_{i+1}(X)$ such that $\partial F + F \partial = D  \circ  A - \operatorname{Id}$.
\begin{marginfigure}
    \centering
    \incfig{d-na-a-is-not-identity}
    \caption{$D  \circ  A \neq  \operatorname{Id}$}
    \label{fig:d-na-a-is-not-identity}
\end{marginfigure}
\begin{marginfigure}
    \centering
    \incfig{chain-homotopy-between-d-na-a-and-id}
    \caption{A compactification of the forward orbit of a chain $\sigma$ forms a chain homotopy between $\sigma$ and  $(D  \circ  A)(\sigma)$.}
    \label{fig:chain-homotopy-between-d-na-a-and-id}
\end{marginfigure}
Intuitively, a chain homotopy is a chain of one dimension higher that connects $(D  \circ  A)(\sigma)$ and $\operatorname{Id} (\sigma) = \sigma$.
The intuitive choice in this case would be the so-called forward orbit of $\sigma$:
\[
    G(\sigma) := [0, \infty) \times \sigma
,\] with the map $e: G(\sigma) \to  M: (s, x) \mapsto \psi_s (\sigma(x))$.
This set can be compactified to a smooth manifold with corners with boundary
\[
    \partial \overline{G(\sigma)} = - \sigma \cup - G(\partial \sigma) \cup \bigcup_{d}  \L \sigma d \times \unstable d
.\] 
We can extend $e$ to a smooth map and define 
\begin{align*}
    F: C_i(M) &\longrightarrow C_{i+1}(X) \\
    \sigma &\longmapsto e_* \left[ \overline{G(\sigma)} \right] 
.\end{align*}
While intuitively, this is a chain map between $\sigma$ and  $(D  \circ A)(\sigma)$, we can similarly as before, we should check that $\partial F + F \partial = D  \circ  A - \operatorname{Id}$.
Calculating $\partial F$ and  $F \partial$, we have
\begin{align*}
    (\partial F)(\sigma) &= -e_*[\sigma] - e_*[G(\partial \sigma)] + \sum_{d} e_*\left[\L \sigma d \times \unstable d\right]\\
    (F \partial)(\sigma) &= e_*\left[\overline{G(\partial \sigma)}\right]
    .\end{align*}
    When we add this up, $[G (\partial \sigma)]$ and $\left[ \overline{G(\partial \sigma)} \right]$ cancel as currents, leaving us with

    \[
        (\partial F + F \partial)(\sigma) = -e_* [\sigma] +\sum_{d} e_*\left[\L \sigma d \times \unstable d\right].
    \]
    Now, calculating $(D  \circ A - \operatorname{Id})(\sigma)$, we get
    \begin{align*}
        (D  \circ A - \operatorname{Id})(\sigma) &=
        D\left(\sum_d \# \L \sigma d d\right) - e_*[\sigma]\\
        &= \sum_d \# \L \sigma d \cdot D(d) - e_*[\sigma]\\
        &= \sum_d \# \L \sigma d \cdot e_*[\unstableb d] - e_*[\sigma]
    .\end{align*} 
    Keeping in mind that as currents $[\unstableb p] = [\unstable p]$, we find that $D  \circ  A - \operatorname{Id} = \partial F + F \partial$.

    \begin{itemize}
        \item \todo{Specify summations more clearly. Sometimes only over specific index!}
        \item Fix definition of Morse / partial k + 1
    \end{itemize}
\end{proof}

