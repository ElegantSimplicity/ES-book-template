\chapter{Stable and unstable manifolds}
\label{chap:stable-and-unstable-manifolds}

% \startcontents[chapters]
% \printcontents[chapters]{}{1}{}


In this chapter, we address two natural questions that came up when discussing examples of handle decompositions: `When can we reorder handles in a handle decomposition?' and `Under what conditions can we cancel two critical points?'
To answer these questions, we introduce the concept of stable and unstable manifolds.
While we previously viewed handles as a local phenomenon, these new concepts will allow us to understand how multiple handles can interact on a more global scale.
Moreover, stable and unstable manifolds will give rise to trajectory spaces between critical points which will play a key role in the upcoming chapters.

\section{Definition of stable and unstable manifolds}
Stable and unstable manifolds associated to a critical point $p$ consist of points that under the flow of the (negative) gradient reach $p$ in the limit.
\begin{marginfigure}
    \centering
    \incfig{stable-and-unstable-manifolds-are-manifiolds}
    \caption{Locally in a Morse chart, stable and unstable manifolds are given by the vertical and horizontal axis, i.e.\ $ x_1= \cdots= x_k = 0$ and $x_{k+1}= \cdots= x_n = 0$.}
    \label{fig:stable-and-unstable-manifolds-are-manifiolds}
\end{marginfigure}
\begin{definition}
    Let $p$ be a critical point of a Morse function $f$.
    Denote by $\psi^{t}$ the flow of a pseudo-gradient.
    Then the unstable manifold is defined as
    \[
        \unstable{p} = \big\{x \in M  \mid  \lim_{t \to -\infty} \psi^{t} (x)  = p\big\} 
    ,\] 
    and its stable manifold is defined as
    \[
        \stable{p} = \big\{x \in M  \mid  \lim_{t \to \infty} \psi^{t} (x)  = p\big\} 
    ,\] 
\end{definition}
Just like their names imply, these sets are indeed manifolds.
Around a critical point $p$, the unstable manifold $\unstable{p}$ is in a Morse chart $U$ given by $x_{k+1} = \cdots = x_n = 0$, so it is diffeomorphic to an open disk $\odisk{k}$.
The points in $\unstable{p}$ that lie outside $U$ can then be obtained by flowing the boundary of $\odisk{k}$ along the pseudo-gradient, which is diffeomorphic to $\sphere{k-1} \times \R$. All in all when gluing this to $\odisk{k}$, we end up with something that is indeed a manifold diffeomorphic to $\odisk{k}$.
A similar reasoning for the stable manifold shows that $\stable{p} \cong \odisk{n - k}$.
In summary, we have obtained the following result:

\begin{prop}
    Stable and unstable manifolds of a critical points are submanifolds diffeomorphic to open disks. Moreover,
    \[
        \dim \unstable{p} = \codim \stable{p} = \Ind p
    .\] 
\end{prop}

\begin{eg}
    Let us consider $T^{2}$ embedded in $\R^3$ in the standard way and consider the height function. 
    This function has $4$ critical points and is clearly Morse.
    Let $X = -\grad f$ be the negative gradient of $f$ w.r.t.\ the standard metric on $\R^{3}$.
    Then the stable and unstable manifolds of the critical points of $f$ are illustrated in Figure~\ref{fig:torus-height-function-stable-and-unstable-manifolds}.
    \label{eg:torus-stable-unstable-manifolds-standard-gradient}
    We have similarly done so for an embedding of $S^2$ in Figure~\ref{fig:other-sphere-definition-of-mathcal-m}.
\end{eg}

    \begin{figure}[H]
        \centering
        \incfig{torus-height-function-stable-and-unstable-manifolds}
        \caption{
        Stable and unstable manifolds for all critical points of the height function on the torus with the standard gradient on $\R^3$.
    All of them are diffeomorphic to either $\odisk{2}$, $\odisk{1}$ or  $\odisk{0}$.
    Note that $\stable{b}$ and  $\unstable{c}$ do not intersect transversely.
}
        \label{fig:torus-height-function-stable-and-unstable-manifolds}
    \end{figure}

    \begin{figure}[H]
        \centering
        \incfig{other-sphere-definition-of-mathcal-m}
        \caption{
            Stable and unstable manifolds of the `other sphere'. Note that all of them intersect transversely.
        }
        \label{fig:other-sphere-definition-of-mathcal-m}
    \end{figure}


    \section{Intersections of (un)stable manifolds}

    \begin{marginfigure}
        \centering
        \incfig{when-can-we-attach-multiple-handles-at-the-same-time}
        \caption{
            A cobordism from $S^{1}$ to $S^{1} \sqcup S^1 \sqcup S^1$.
            Stable and unstable manifolds do not intersect, which implies we can reorder the critical points $p$ and $q$.
        }
        \label{fig:when-can-we-attach-multiple-handles-at-the-same-time}
    \end{marginfigure}

\begin{marginfigure}
    \centering
    \incfig{when-can-we-not-attach-multiple-handles-at-the-same-time}
    \caption{The intersection of stable and unstable manifolds is not empty, indicating the dependence of the $2$-handle on the $1$-handle.}
    \label{fig:when-can-we-not-attach-multiple-handles-at-the-same-time}
\end{marginfigure}
    Stable and unstable manifolds give us information about interaction of handles.
    For example, consider the situation in Figure~\ref{fig:when-can-we-attach-multiple-handles-at-the-same-time}.
    None of the stable and unstable manifolds intersect,
    indicating the independence of the two handles.
    This allows us attach both handles at the same time or attach them in a different order.
    In terms of the Morse function,
    this means we can modify $f$ such that  $f(p) > f(q)$.

    In Figure~\ref{fig:when-can-we-not-attach-multiple-handles-at-the-same-time}, the situation is different. The intersection of $\stable p$ and  $\unstable q$ is not empty, indicating the dependence of the  $2$-handle associated to $q$ on the $1$-handle associated to $p$. We cannot first attach the $2$-handle first and then attach the $1$-handle.

    This motivates us to in investigate necessary and sufficient conditions for the stable and unstable manifolds to intersect.
    Suppose $p$ and $q$ are two critical points of a Morse function $f: M \to  \R$ and let us consider $\stable{p} \cap \unstable{q}$.
    If $f(p) > f(q)$, then  $\stable{p} \cap \unstable{q} = \O$, because $f(\stable{p}) > f(\unstable{q})$.
    In the other case when $f(p) \le  f(q)$, the intersection $\stable p \tcap \unstable q$ can be non-empty.
    Moreover, if we assume that the intersection is transverse, we can say something about its dimension.
    Indeed, Proposition~\ref{prop:transverse-codimensions-add} implies that
    \[
        \codim(\stable{p} \tcap  \unstable{q}) = \codim \stable p + \codim \unstable q
    ,\] 
    so we have
    \[
        \dim(\stable p \tcap \unstable q) = \Ind q - \Ind p
    .\]
    In particular, we have that if $\Ind q < \Ind p$,  $\stable{p} \tcap \unstable{q} = \O$, which in other words means that lower index handles do not depend on higher index handles, i.e.\ we can always attach lower index handles before higher index ones.


    This transversality assumption has a name:
    \begin{definition}[Smale condition]
        A pseudo-gradient field addapted to a Morse function $f$ is said to satisfy the \emph{Smale condition} if for all pairs of critical points $ \{p, q\}  \subset \Crit f$, we have that $\stable{p}$ intersects  $\unstable{q}$ transversely, i.e. 
        \[
            \stable{q} \tcap  \unstable{p} \text{ for all $p, q \in \Crit f$}
        .\] 
    \end{definition}
    It turns out that this condition is not at all restricting: we can always perturb the pseudo-gradient field such that it satisfies the Smale condition.
    \begin{theorem}
        Any Morse function $f:M \to  \R$ admits a pseudo-gradient field that satisfies the Smale condition
    \end{theorem}
    This should be intuitively clear as transverse intersections are generic and stable.
    This means that a small perturbation of $\stable q$ and  $\unstable p$ is enough to make their intersection transverse, and we can accomplish this perturbation by perturbing $X$.
    A rigorous proof of this is rather technical and we refer the reader to Audin and Damian.\sidecite{audin}

    Almost all of the previous examples we have given satisfy the Smale condition with the one exception being the torus.
    \begin{noneg}
        The gradient vector field in Example~\ref{eg:torus-stable-unstable-manifolds-standard-gradient} does not satisfy the Smale condition: the intersection of $\stable{b}$ and  $\unstable{c}$ is not transverse.
        Even more, if the condition were satisfied, there would be no trajectories connecting $b$ and $c$, because both have index $1$, so $\dim (\stable{b} \tcap \unstable{c}) = 0$. However, as illustrated in the figure, we have two such paths.
    \end{noneg}
    \begin{eg}
        Instead of considering the embedding as in the previous example, we can embed the torus at a slight angle.
        Then the gradient of the height function (by using the standard metric on $\R^3$) does satisfy the Smale condition.
        The intersection of $\stable{b}$ and  $\unstable{c}$ is no longer tangent: indeed, they do not even intersect at all. We have illustrated this below.
        \label{eg:tilted-torus}
    \end{eg}
    \begin{figure}[H]
        \centering
        \sidecaption{
            When embedding the torus in $\R^3$ tilted, the Smale condition is satisfied: all stable and unstable manifolds intersect transversely.
            Indeed, stable and unstable manifolds of $c$ and $b$ don not intersect at all.
        \label{fig:torus-tilted-height-function-stable-and-unstable-manifolds}
        }
        \incfig{torus-tilted-height-function-stable-and-unstable-manifolds}
    \end{figure}
    \begin{eg}
        As illustrated in Figure~\ref{fig:other-sphere-definition-of-mathcal-m}, the `other sphere' with gradient of the height function w.r.t the metric of $\R^3$ does satisfy Smale condition.  \end{eg}

The Smale condition also has another interesting consequence, which we have not touched upon.
If stable and unstable manifolds intersect transversely, we know that the intersection is again a submanifold.
So $\stable{q} \tcap \unstable{p}$ is a manifold for all critical points $p, q \in \Crit f$.
This submanifold consists of all points on the trajectories connecting $p$ to $q$.
 \begin{definition}
    Let $f: M \to  \R$ be a Morse function and $\psi^{t}$ the flow of a pseudo-gradient that satisfies the Smale condition.
    Then we define
    \begin{align*}
        \traj{p}{q} &= \stable{q} \tcap \unstable{p}\\
                    &= \big\{
            x \in M 
            \mid 
            \lim_{t \to -\infty} \psi^{t}(x) = p, \ 
            \lim_{t \to \infty} \psi^{t}(x) = q
        \big\} 
    ,\end{align*} 
    which is a submanifold of dimension $\Ind p - \Ind q$.
\end{definition}

\begin{eg}
    Consider the `other sphere'. 
    We have illustrated $\traj{p}{q}$ below for some of the critical points of the height function.
    Here we see that these type of submanifolds do not need to be connected. For example, $\traj{b}{a}$ is diffeomorphic to the disjoint union of two open intervals.
\end{eg}
\begin{figure}[H]
    \centering
    \sidecaption{Illustration of $\traj p q$ for critical points of the `other sphere'.
    \label{fig:mathcal-m-trajectories-other-sphere-only-m}
    }
    \incfig{mathcal-m-trajectories-other-sphere-only-m}
\end{figure}

Instead of considering a manifold consisting of of points lying on all trajectories from $p$ to $q$, $\traj{p}{q}$, it is often more interesting to construct a manifold each point corresponds to exactly one trajectory, that is a so-called moduli space of trajectories.
We can do this by modding out $\traj{p}{q}$ by $\R$-action of translations in time. We denote the resulting space with $\L{p}{q}$.
More explicitly, we have the following:

\begin{prop}
    Let $f: M \to  \R$ be a Morse function and $\psi^{t}$ the flow of a pseudo-gradient field satisfying the Smale condition.
    Then the group $(\R, +)$ of time translations acts on $\traj{p}{q}$ by  $t \cdot x = \psi^{t}(x)$.  If $p \neq q$ then the action is free and we can define $ \L{p}{q} = \traj{p}{q} / \R $. The dimension of $\L{p}{q}$ is $\Ind p - \Ind q - 1$.
\end{prop}
\begin{myproof}
    It is clear that $\R$ acts on $\traj{p}{q}$ by time translations. 
    If $p \neq q$,  $\traj{p}{q}$ does not contain any critical point, so flowing along a pseudo-gradient field, the value of $f$ is strictly decreasing. This proves freeness.
\end{myproof}

\begin{remark}
    If the index of two points only differs by one, say $\Ind p = \Ind q + 1$, then the dimension of $\L{p}{q}$ is $0$, so it is a discrete set.
    This proves that the number of trajectories from $p$ to $q$ is always countable.
    We will later prove that it is in fact finite.
    \label{remark:trajectories-finite}
\end{remark}
\begin{remark}
    Another way to look at $\L{p}{q}$ is to consider a regular value $a \in \R$ such that $f(p)<a<f(q)$. Then every flow line from $p$ to $q$  intersects $f^{-1}(a)$ exactly once (this is because the value of $f$ is decreasing along the way), so we can identify  $\L{p}{q}$ with  $\traj{p}{q} \cap f^{-1}(a)$.
\end{remark}


\begin{eg}
    Below, we have illustrated the moduli space of trajectories for the `other sphere'.
    When there is a single trajectory for example between $c$ and  $b$, $\L cb$ consist of a single point.
    If the indices of critical points differ by two, as is the case for $d$ and  $a$,  $\L da$ is a one-dimensional manifold describing a one-parameter family of flow lines between  $d$ and  $a$.
\end{eg}
\begin{figure}[H]
    \centering
    \sidecaption{
        Illustration of $\traj{p}{q}$ and $\L{p}{q}$ for critical points of the `other sphere'.
    \label{fig:mathcal-m-trajectories-other-sphere}
    }
    \incfig{mathcal-m-trajectories-other-sphere}
\end{figure}




\filbreak
\section{Reordering critical points}
\mbox{}\\[-3em]

Now that we have introduced the (un)stable manifolds, we are ready to prove a first reordering theorem. The statement and proof can be found in notes of `Lectures on the $h$-Cobordism Theorem' by Milnor.\sidecite{hcobord}
\begin{theorem}
    Let $f: M \to  \R$ be a Morse function on a cobordism $M$ from $M_0$ to $M_1$ with two critical points $p$ and  $p'$.
    Suppose that for some choice of pseudo-gradient field $X$, the stable and unstable manifolds do not intersect.
    Let $a, a' \in (0,1)$ be arbitrary.
    Then there exists a new Morse function $g$ such that
    \begin{enumerate}[(a)]
        \item $X$ is a gradient-like vector field for  $g$
        \item The critical points of  $g$ are still $p, p'$ and $g(p) = a$,  $g(p') = a'$.
        \item $g$ agrees with  $f$ near $M_0 \sqcup M_1$ and equals $f$ plus a constant in some neighbourhood of  $p$ and some neighbourhood of  $p'$.
    \end{enumerate}
    \label{thm:reordening}
\end{theorem}
\begin{marginfigure}
    \centering
    \incfig{reordening-theorem-milnor-h-cobordism}
    \caption{
        Construction of $\overline{\mu}$ and $\pi$ in the proof on reordening critical points.
    }
    \label{fig:reordening-theorem-milnor-h-cobordism}
\end{marginfigure}
\begin{myproof}
    We want to mask out the area around the stable and unstable manifolds of one of the critical points.
    Let $\mu: M_0 \to  [0,1]$ be a smooth map that vanishes around  $ M_0 \cap \unstable p$ and $1$ around $M_0 \cap \unstable{p'}$, as illustrated in Figure~\ref{fig:reordening-theorem-milnor-h-cobordism}.
    Then we can smoothly extend this to a function on the whole manifold $M$ as follows.
    Define $\pi: M \to  M_0$ by flowing along the pseudo-gradient field until we reach $M_0$.
    Then we can extend $\mu$ uniquely to a smooth function that is constant on each trajectory by defining
    \begin{align*}
        \overline{\mu}: M &\longrightarrow [0,1] \\
        x &\longmapsto \begin{cases}
            0 & \text{if $x$ in stable or unstable manifold of $p$}\\
            1 & \text{if $x$ in stable or unstable manifold of $p'$}\\ 
            \mu(\pi(x)) & \text{else}
        \end{cases}
    .\end{align*}
    We have illustrated this in the figure by
    indicating the value of $\overline{\mu}$ in red.

    Define a new Morse function $g: M \to  [0,1]$ by $g(q) = G_{\overline{\mu}(q)}(f(q))$, where $G_{s}(x)$ is a smooth family of smooth functions $G_s: [0,1] \to  [0,1]$ with $s \in [0,1]$ that has the following properties, also indicated in Figure~\ref{fig:proof-reordening-properties-of-g}.
    \begin{marginfigure}
        \centering
        \incfig{proof-reordening-properties-of-g}
        \caption{Necessary properties of $G$ in the proof on reordering critical points are indicated in yellow.}
        \label{fig:proof-reordening-properties-of-g}
    \end{marginfigure}
    \begin{enumerate}[(1)]
        \item For all $s$, $G_s' > 0$ and $G_s(0) = 0, G_s(1) = 1$
        \item  $G_0(f(p)) = a$\\ $G_1(f(p')) = a'$
        \item  $G_s(x) = x$ for $x$ near  $0$ or  $1$ and for all $s$ 
        \item  $G_0'(x) = 1$ for $x$  in a neighbourhood of $f(p)$\\
        $G_1'(x) = 1$ for $x$  in a neighbourhood of $f(p')$
    \end{enumerate}
    Claim (b) and (c) in the statement of the theorem are clear: they follow immediately from (2), (3) and (4).
    For (a), consider
    \[
        dG = \frac{\partial G}{\partial \overline{\mu}}  d\overline{\mu} + \frac{\partial G}{\partial f}  df
    .\] 
    Plugging in $X$, we have
    \begin{align*}
        dg(X) &= \frac{\partial G}{\partial \overline{\mu}}  d\overline{\mu}(X) + \frac{\partial G}{\partial f}  df(X)\\
              &= \frac{\partial G}{\partial f}  df(X) < 0 \text{, except at critical points of $f$}
    ,\end{align*} 
    where we used that $d\overline{\mu}(X) = 0$ by construction of $\overline{\mu}$, $\frac{\partial G}{\partial f} > 0$ by (1) and $df(X) < 0$ everywhere except at critical points of $f$ by definition of pseudo-gradient field.
    Because of (4), this implies that $g$ and $f$ share the same behaviour around their critical points, proving that $g$ is also Morse.
\end{myproof}

\begin{remark}
    This theorem can be extended to a more general setting. Suppose we have a set of points $\mathbf{p} = \{p_1, \ldots, p_k\}$ and $\mathbf{p}' = \{p_1', \ldots, p_\ell'\}$, with all $p_i$ at the same level and all $p_i'$ at a single level.
    Then the theorem remains valid, with exactly the same proof.
\end{remark}

Applying the previous theorem repeatedly and using the fact that if $\Ind p \le  \Ind q$, the intersection of $\unstable{p}$ and  $\stable{q}$ is empty, we find the following result:

\begin{theorem}
    Any closed manifold admits a Morse function such that for all critical points, $\Ind p < \Ind q \implies f(p) < f(q)$.
    In particular, it admits a Morse function which satisfies $ \Ind p  = f(p) $ for all critical points $p \in \Crit f$.
\end{theorem}
In other words, this asserts that lower index handles can always be attached before higher index ones.
Morse functions as described in the second part of the theorem have a name:
\begin{marginfigure}
    \centering
    \incfig{self-indexing-morse-function-torus-tilted}
    \caption{When tilting the torus to the right angle, the height function becomes self-indexing.}
    \label{fig:self-indexing-morse-function-torus-tilted}
\end{marginfigure}
\begin{definition}[Self-indexing Morse function]
    A Morse function $f: M \to  \R$ is \emph{self-indexing} if $\Ind p = f(p)$ for all critical points  $p$ of  $f$.
\end{definition}
\begin{eg}
    Consider $T^2 \subset \R^3$ embedded at an angle. We have illustrated the side view in Figure~\ref{fig:self-indexing-morse-function-torus-tilted}.
    Then the height function is a self-indexing Morse function.
\end{eg}

\section{Heegaard splittings}
\begin{marginfigure}
    \centering
    \incfig{heegaard-splittings-schemattically}
    \caption{Schematic visualization of a self-indexing Morse function on a $3$-manifold $M$.
        The manifold $S = f^{-1}(\frac{3}{2})$ is called the splitting surface of $M$.
    }
    \label{fig:heegaard-splittings-schemattically}
\end{marginfigure}
In the three-dimensional setting, the existence of self-indexing Morse functions give rise so-called Heegaard splittings.
Heegaard splittings form an essential tool for a low-dimensional topologist (studying manifolds of dimension $\le 4$) and give a way to understand $3$-manifolds.
As Heegaard splittings will not play an important role in the upcoming chapters, we will only discuss them briefly.



A self-indexing Morse function gives rise to a handle decomposition schematically shown in Figure~\ref{fig:heegaard-splittings-schemattically}. When splitting the manifold along $f^{-1}(\frac{3}{2})$, we decompose $M$ in two parts: a part that consists of $0$- and $1$-handles and a part consisting only of $2$- and  $3$-handles.
To build $M$ we glue these two parts along their boundary.
The part consisting of $2$- and $3$-handles can also be seen as being constructed of $0$- and $1$-handles, simply by building $M$ from top to bottom by considering the Morse function $-f$ instead of $f$. This then interchanges $k$- and $n-k$-handles, which in the three-dimensional case results in $2 \leftrightarrow 1$ and  $3 \leftrightarrow 0$.
All things considered, we have a decomposition of $M$ in two so-called handlebodies:
\begin{definition}[Genus $k$ handlebody\sidenotemark]
    A genus $k$ handlebody is a compact connected orientable $3$-manifold with boundary that possesses a handle decomposition consisting of $0$-handles and $1$-handles such that its boundary is a surface of genus $k$.
\end{definition}
A Heegaard splitting is then defined as follows.
\begin{definition}[Heegaard splitting\sidenotemark]
    A \emph{Heegaard splitting} of a closed $3$-manifold $M$ is a decomposition $M = V \cup _S W$ such that $V$ and $W$ are genus $k$ handlebodies and  $S = \partial V = \partial W$. Here  $S$ is called the \emph{splitting surface of $M$}.
    Two Heegaard splittings are considered \emph{equivalent} if their splitting surfaces are isotopic. The \emph{genus} of a Heegaard splitting is the genus of $S$.
\end{definition}
\sidenotetext[][-5cm]{\fullcite{schultens2014introduction}}

With these definitions, we can summarize our findings as follows:
\begin{theorem}
    Every closed orientable $3$-manifold admits a Heegaard splitting.
\end{theorem}
\begin{myproof}
    Let $f$ be a self-indexing Morse function on $M$.
    Then $M = f^{-1}\left[0, \tfrac{3}{2}\right] \cup f^{-1}\left[\tfrac{3}{2}, 3\right]$ is a Heegaard splitting of $M$ by duality of $k$- and $n-k$-handles.
\end{myproof}

Let us give some examples.
\begin{marginfigure}
    \centering
    \incfig{genus-four-heegaard-splitting-of-s3}
    \caption{A genus four Heegaard splitting of $S^{3}$, seen as the one point compactification of $\R^3$.
        This way, we can obtain a Heegaard splitting of $S^{3}$ of any genus.
    }
    \label{fig:genus-four-heegaard-splitting-of-s3}
\end{marginfigure}
\begin{eg}
    As seen earlier, the `height' function on $S^{3}$ gives rise to a handle decomposition $S^{3} = \cdisk{2} \cup \cdisk{2}$. This is a Heegaard splitting of genus $0$ with splitting surface $S^{2}$.
\end{eg}
\begin{eg}
    The previous example is not the only way to decompose $S^{3}$.
    Indeed, we can arbitrarily add cancelable $1$- and $2$-handles (as in Example~\ref{eg:cancel-three}) to increase the genus. 
    For example a genus four Heegaard splitting of $S^{3}$ is illustrated in Figure~\ref{fig:genus-four-heegaard-splitting-of-s3}.
    Here, we visualized $S^{3}$ as the one point compactification of $\R^3$.
    \label{eg:s3-heegaard-four-genus}
\end{eg}

\begin{eg}
    If we split the handle decomposition of $T^{3}$ given in Example~\ref{eg:handle-decomposition-three-torus} between index $1$ and $2$ critical points, then we find a splitting surface that has genus three, as illustrated in Figure~\ref{fig:three-torus-handle-decomposition}.
\end{eg}

% \subsection{Stabilization and destabilization}
\begin{marginfigure}
    \centering
    \incfig{stabilization}
    \caption{
        The act of stabilization is replacing a ball near the boundary as illustrated, increasing the genus of the splitting surface by one.}
    \label{fig:stabilization}
\end{marginfigure}
\begin{remark}
    The act of increasing the genus, as done in Example~\ref{eg:s3-heegaard-four-genus} is called stabilization.
    It can be done in as in Figure~\ref{fig:stabilization}.
    The reverse procedure, called destabilization cannot be done in general.
\end{remark}

\section{Cancellation of critical points}

Let us now answer the second question that came up when discussing examples in the previous chapter: `When can we cancel a pair of critical points $p, q$?'
It turns out that the existence of a unique trajectory connecting $p$ to  $q$ is a sufficient condition, as the following theorem states.


\begin{theorem}[Cancellation theorem\sidenotemark]
    Let $M$ be a cobordism from $M_0$ to $M_1$.
    Let $f: M \to  [0,1]$ be a Morse function with exactly two critical points $p, q$ of index  $k+1$ and  $k$ and let $X$ be a pseudo-gradient adapted to $M$.

    If $\# \L p q = 1$, that is if there is a single trajectory $\ell = \traj p q$ connecting $p$ and  $q$, then we can cancel $p$ and  $q$.
    More specifically, we can alter $X$ around  $\ell$ and $f$ away from  $M_0, M_1$ such that $f$ has no critical points.
    \label{firstcancellation}
\end{theorem}
\sidenotetext[][-3cm]{\fullcite{hcobord}}

Below, we have illustrated the alteration of the pseudo-gradient $X$:
\begin{figure}[H]
    \centering
    \sidecaption{The pseudo-gradient before and after the alteration.
        Initially, $X$ vanishes twice (once at  $p$ and once at $q$).
        After the alteration, $X'$ does vanishes nowhere.
    \label{fig:alteration-of-pseudogradient-vector-field}}
    \incfig{alteration-of-pseudogradient-vector-field}
\end{figure}

Before proving this theorem, let us give an example showing the importance of the conditions in the theorem.
\begin{marginfigure}
    \centering
    \incfig{other-sphere-cancelation-of-critical-points}
    \caption{The `other sphere' with trajectories between critical points whose index differ by exactly one. We can only cancel $b$ and $c$ or $d$ and $b$. Cancelling $b$ and $a$ is impossible.}
    \label{fig:other-sphere-cancelation-of-critical-points}
\end{marginfigure}
\begin{eg}
    Consider the `other sphere' as in Figure~\ref{fig:other-sphere-cancelation-of-critical-points}, where we have also drawn trajectories between critical points of consecutive index. 

    For the pair $(d, b)$, let $N = f^{-1}([f(b) - \epsilon, f(d) + \epsilon])$ be a cobordism containing only $b$ and $d$ as critical points.
    There is a unique flow line connected $d$ to $b$, and the theorem allows us to cancel these two critical points.

    At first glance, the theorem does not allow us to cancel $c$ and  $b$: while there is a unique trajectory connecting $c$ to  $b$, we cannot cut out a part of $M$ only containing $c$ and  $b$ as critical points.
    This however turns out not be a problem, because we can lower $c$ and raise $d$ using methods discussed before such that $f(d) > f(c)$. Once that is done, we can consider $N = f^{-1}([f(b) - \epsilon, f(c) + \epsilon])$ and apply the theorem.

    The last pair of critical points of adjacent index  is $(b, a)$.
    In this case, there are two flow lines connecting  $b$ to $a$ and  we cannot cancel the critical points.
\end{eg}



\begin{myproof}
    We will follow the proof given by Milnor which is based on the original proof by Morse.\sidecite{cairns2015differential,huebsch1964bowl}
    For a proof focussing more on handles than on critical points, we refer the reader to `Differential Geometry' by Kosinski.\sidecite{kosinski2013differential}
    There is also a nice proof by Laudenbach\sidecite{laudenbach2013proof} reducing the general case to dimension one, where the problem is easy to solve, as we have seen in Example~\ref{eg:cube}.

We prove the statement in a local model.
Let $U_\ell$ be an open neighbourhood of $\ell$. 
We may assume that there are coordinates such that:
\begin{itemize}
    \item  The critical points are given by $p = (0, \ldots, 0)$ and $q = (1, 0, \ldots, 0)$
    \item The pseudo-gradient is given by $X = (v(x_1), x_2, \ldots, x_k, -x_{k+1}, \ldots, -x_n)$ where $v(x_1)$ is a smooth function of $x_1$, such that  $v$ is positive on  $(0, 1)$, vanishes at $0$ and $1$ and is negative elsewhere.
        Moreover, we assume that  $v'(x_1) = 1$ near $0$ and $1$.
\end{itemize}
We have illustrated these properties in Figure~\ref{fig:local-model} and formal proof of this fact can be found in the notes by Milnor.
\begin{marginfigure}
    \centering
    \incfig{local-model}
    \caption{Model of the pseudo-gradient vector field $X$. In local coordinates it is given by $X = (v(x_1), x_2, \ldots, x_k, -x_{k+1}, \ldots, -x_n)$, where $v$ is as illustrated above.
    }
    \label{fig:local-model}
\end{marginfigure}

\renewcommand{\qedsymbol}{\ensuremath{\blacksquare}}

\paragraph{Assertion 1.}
Given an open neighbourhood $U$ of  $ \ell$, we can find a smaller neighbourhood $U'$ such that no trajectory exiting $U'$ enters $U'$ again.

Note that this can be false when there are multiple trajectories connecting $p$ to $q$, as is the case with $p = b, q = a$ in Figure~\ref{fig:other-sphere-cancelation-of-critical-points}.
\begin{myproof}
    Suppose this was not the case.
    Then there would exists a sequence of trajectories $\ell_k$ that pass through points $r_k, s_k, t_k$ with $s_k \not\in U$ and $r_k$ and $t_k$ approaching $\ell$.
    We may assume that $s_k \to s$ because $M \setminus U$ is compact.
    The trajectory through $s$, call it $ \ell_\infty$ comes from  $ M_0$ or reaches $ M_1$ (or both), as it would otherwise be a trajectory connecting $p$ and $q$ not equal to $\ell$.
    Suppose without loss of generality that it comes from $M_1$.
    Then the tail of the sequence of trajectories $\ell_{k}$ passing through points $s_k$ near $s$ also originate in $M_1$.
    This means that the minimal distance between points on $\ell_{k}$ and points on $\ell$ is bounded from below by a positive number for $k$ big enough.
    Now, because $r_k \in \ell_k$, the points $r_k$ cannot approach  $\ell$. This is a contradiction.
\end{myproof}

Let $U$ and $U'$ be neighbourhoods of $\ell$ such that $\ell \subset U' \subset U \subset \overline{U} \subset U_\ell$ such that $U'$ satisfies the conditions of the previous assertion.

\paragraph{Assertion 2.}
We can alter $X$ on a compact subset of $U'$ in such a way that any point in $U$ will exit $U$ when flowing both backwards and forwards in time.
In other words, any trajectory that contains a point in $U$ has entered $U$  and will exit $U$.
Note that the altered vector field is not longer adapted to $f$.

\begin{myproof}
    \begin{marginfigure}
        \centering
        \incfig{alteration-of-vector-field}
        \caption{Alteration of the pseudo-gradient. In particular, notice that the alteration vanishes nowhere.}
        \label{fig:alteration-of-vector-field}
    \end{marginfigure}
    Replace $X$ by  $(w(x_1, \rho), x_2, \ldots, -x_n)$.
    Here, $\rho = \sqrt{x_2^2 + \cdots + x_n^2}$ is the norm of the coordinates normal to $\ell$ and $w(x_1, \rho)$ satisfies the following properties, also illustrated in Figure~\ref{fig:alteration-of-vector-field}.
    \begin{itemize}
        \item The vector field $w(x_1, \rho(x))$ is equal to $v(x_1)$ outside a compact neighbourhood of  $ \ell$ in $U'$.
        \item For $\rho = 0$, the vector field $w(x_1, 0)$ is everywhere negative.
    \end{itemize}
    To prove the claim, let $x^0 = (x_1^{0}, \ldots, x_n^{0}) \in U$ and let $x(t)$ the unique trajectory such that  $x(0) = x^{0}$.
    \begin{enumerate}[(a)]
        \item If one of $x_{k+2}^{0}, \ldots, x_n^{0}$ is non-zero, say $x_m$. Then $|x_m(t)|$ increases exponentially, so the trajectory leaves $U$.
        \item If all of $x_{k+2}^{0}, \ldots, x_n^{0}$ are zero, then $\rho$ decreases exponentially, so it gets closer to the $x_1$-axis.
            Since $w(x_1, \rho(x))$ is strictly negative on the $ x_1$-axis, it is also negative on a compact set $K = \{ x \in U  \mid  \rho(x) \le  \delta\} $ for some small $\delta$.
            Therefore, $ w(x_1, \rho(x))$ has a negative upper bound $-\alpha < 0$ on $K$.
            As $\rho$ decreases exponentially, eventually  $x \in K$ and $x_1'(t) < -\alpha$ from then onwards.
            As $U$ is bounded, we conclude that $x$ leaves $U$. \qedhere
    \end{enumerate}
\end{myproof}
\paragraph{Assertion 3.}
Every trajectory of $X'$ goes from $M_0$ to $M_1$.
\begin{myproof}
    Integral curves that do not enter $U$ follow $X$ and hence always go from  $M_1$ to $M_0$.
    Any integral curve through a point in $U$ eventually exits $U$ by Assertion 2. Moreover, it cannot re-enter $U$ by Assertion $1$, and hence once exited, it follows $X$.
\end{myproof}

\paragraph{Assertion 4.}
The vector field $X'$ determines a diffeomorphism $\phi: [0, 1] \times M_0 \to M$ that maps $0 \times M_0$ to $M_0$ and $1 \times M_0$ to $ M_1$.
\begin{myproof}
    \begin{marginfigure}
        \centering
        \incfig{assertion-4}
        \caption{The nowhere vanishing vector field $X'$ determines a diffeomorphism between  $[0,1] \times M_0$ and the manifold $M$.}
        \label{fig:assertion-4}
    \end{marginfigure}
    We will reverse $X'$ and normalize it such that flowing along it for time $1$, it it sends a point in  $M_0$ to a point in $M_1$.
    Let $\tau_0(q)$ be the time needed to flow $q$ back to $M_0$ along $-X'$ and similarly define $\tau_1(q)$. The maps $\tau_0$ and $\tau_1$ depend smoothly on $q$.
    Let $\pi_0$ be the projection  $M \to M_0$ by flowing along $-X'$ and similarly define $\pi_1$.
    Then the vector field $Y(q) = -\tau_1(\pi_0(q)) X'(q)$ has flow lines that go from $ M_0$ to $M_1$ in unit time.
    This defines a diffeomorphism in the following way:
    \begin{align*}
        \phi: [0, 1] \times M_0 &\longrightarrow M \\
        (t, q) &\longmapsto \psi^{t}_Y (q)
    ,\end{align*}
    where $\psi^{t}_Y$ is the flow of $Y$.
    The inverse of $\phi$ is $q \mapsto (\tau_0(q), \pi_0(q))$.
\end{myproof}

\paragraph{Assertion 5.}
The vector field $X'$ is a pseudo-gradient vector field for some Morse function  $g$ on $M$ that agrees with $f$ near  $M_0$ and $M_1$.
Moreover, $g$ has no critical points.
\begin{myproof}
    By using the previous assertion,
    it suffices to define a Morse function $g$ on $[0, 1] \times M_0$ such that $\frac{\partial g}{\partial t} > 0$ and $g$ corresponds to  $ f_1 = f  \circ  \phi$ around $0 \times M_1$ and $1 \times M$.
    To make sure that $\frac{\partial g}{\partial t} > 0$, we will define $g$ as the integral of a positive function,
    and to make sure it corresponds to $f_1$ around the boundary of $[0,1] \times M_0$, we will define $g$ in the following way:
     \[
         g(u, q) = \int_0^{u} \lambda(t) \frac{\partial f_1}{\partial t} + (1-\lambda(t))\: k(q) \: dt
    .\] 
    Here, $\lambda: [0,1] \to  [0,1]$ is a bump function supported in the interior of $[0, \delta) \cup (1-\delta, 1]$, with $\delta$ small enough such that $\frac{\partial f_1}{\partial t}<0$ in $[0, \delta) \cup (1-\delta, 1]$ and $k: M_0 \to [0,1]$ is some yet unknown function that only depends on $q$.
    We have illustrated the situation below.
    \begin{figure}[H]
        \centering
        \sidecaption{Construction of a Morse function that is adapted to $X'$ and equals $f$ in a neighbourhood of the boundary of $[0,1] \times M_0$.
            Here $\phi$ is the diffeomorphism obtained in Assertion 4.
        \label{fig:assertion-5}
        }
        \incfig{assertion-5}
    \end{figure}

    The idea is to integrate the derivative of the original Morse function near the bottom and the top, and in the middle we will integrate an appropriately large function $k(q)$ such that  $g(1, q) = 1$ for all $q$, i.e.\ the speed is adjusted so that at $t=1$, we reach the top of $[0,1] \times M_0$.
    Expanding this condition and solving for $k(q)$, we get
     \begin{align*}
         g(1, q) = \int_{0}^{1} \lambda(t) \frac{\partial f_1}{\partial t}  + \int_{0}^{1}  (1-\lambda(t)) k(q) dt &= 1\\
          \int_{0}^{1}  (1-\lambda(t))  dt \  k(q)&= 1 - \int_{0}^{1} \lambda(t) \frac{\partial f_1}{\partial t}\\
          k(q)&= \frac{1 - \int_{0}^{1} \lambda(t) \frac{\partial f_1}{\partial t}}{\int_{0}^{1}  (1-\lambda(t))  dt}
    .\end{align*} 

    Taking $\delta$ small enough ensures that  $k(q) > 0$, hence $\frac{\partial g}{\partial t}>0$, concluding the proof of the last assertion.
\end{myproof}

This finishes the proof of the first cancellation theorem.
\renewcommand{\qedsymbol}{\ensuremath{\square}}
\end{myproof}

