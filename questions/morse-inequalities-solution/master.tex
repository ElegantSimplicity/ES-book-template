\documentclass[a4paper]{article}

\usepackage[shortlabels]{enumitem}
\usepackage{float}
\usepackage[utf8]{inputenc}
\usepackage[T1]{fontenc}
\usepackage{textcomp}
\usepackage[dutch]{babel}
\usepackage{amsmath, amssymb, amsthm}
\usepackage{url}


% figure support
\usepackage{import}
\usepackage{xifthen}
\pdfminorversion=7
\usepackage{pdfpages}
\usepackage{transparent}
\newcommand{\incfig}[1]{%
    \def\svgwidth{\columnwidth}
    \import{./figures/}{#1.pdf_tex}
}

\newcommand\R{\mathbb R}
\newcommand\Z{\mathbb Z}
\newcommand\Q{\mathbb Q}
\newcommand\F{\mathbb F}
\pdfsuppresswarningpagegroup=1
\DeclareMathOperator{\Crit}{Crit}

\DeclareMathOperator{\Ker}{ker}
\renewcommand{\Im}{\operatorname{Im}}
\DeclareMathOperator{\rank}{rank}

\begin{document}

\section{Ranks of Abelian groups}
We know that for free $\Z$-modules, the rank-nullity theorem holds, e.g. tensor $H_n(M, \Z)$ by $\Q$ to only filter out the free part.
This immediately allows us to prove the Morse inequalities for $H_k(M, \Z)$, taking only the rank of the free part of $H_k(M,\Z)$ into account.

From now on, denote this notion of  rank of an abelian group $A$ by  $r_0(A)$.  $r_0(A)$ is the cardinality of a maximal set of free independent elements in $A$, i.e. independent elements of infinite order.

Denote by $r_p(A)$ the cardinality of a maximal set of independent elements of order  $p$ in  $A$ ($p$ prime)
, by $r(A)$ the cardinality of a maximal set of independent elements (of any order).
$r(A)$,  $ r_0(A)$, $r_p(A)$ are all invariants of  $A$ and  $\tau(A) = r_0(A) + \sum_{p} r_p(A)$.
We also write $r_t(A) = \sum_p r_p(A)$ and we call this the torsion rank.


\section{Morse inequalities}
Strong Morse inequalities over $\Z$ in the usual sense. Let $0 \le  m \le  n = \dim M$.
\[
    \sum_{k=0}^{m} (-1)^{k+m} \# \Crit_k f \ge  \sum_{k=0}^{m} (-1)^{k+1} r_0(H_k(M, \Z))
.\] 
In fact, we precisely get
\[
    \sum_{k=0}^{m} (-1)^{k+1} r_0(H_k(M, \Z)) = \sum_{k=0}^{m} (-1)^{k+m} \# \Crit_k f - r_0 (\operatorname{Im} \partial_{m+1})
.\] 
where $\partial_k: C_k \to  C_{k-1}$. (Do this over $\Z$ if you haven't yet)

In the Encyclopedia of Mathematics, we additionally find the following inequalities for critical points of a particular index:
\[
    \# \Crit_k (f) \ge  \underbrace{r_0(H_k(M, \Z) + r_t(H_k(M, \Z))}_{r(H_k(M, \Z))} + r_t(H_{k-1}(M, \Z))
,\] 
for all $k$.

\begin{proof}
    Here, it is important to note that the $C_k$ are all free abelian groups, as are  $\ker(\partial_k)$ and  $\Im(\partial_k)$. Only $H_k(M, \Z)$ are on occasion not free. We thus have 
    \begin{align*}
        \# \Crit_k(f) = r(C_k) = r_0(C_k)
        &= r_0(\ker \partial_k) + r_0(\Im \partial_k))\\
        &= r(\ker \partial_k) + r(\Im \partial_k) \tag{*}
    .\end{align*}
    Note that
    \[
        r(H_k(M, \Z)) = r\left( \frac{\ker (\partial_k)}{\Im (\partial_k+1)} \right) \le  r(\ker(\partial_k)) = r_0(\ker \partial_k)
    .\] 
    Also note that 
    \[
        r(H_k(M, \Z)) = (r_0 + r_t)(H_k(M, \Z))
    ,\] 
    and $H_{k-1}(M, \Z) = \frac{\Ker \partial_{k-1}}{\Im \partial_k}$.

    Since $\ker \partial_{k-1}$ is a free abelian group, all torsion in $H_{k-1}(M, \Z)$ has to come from quotienting $\Im \partial_k$.
    Clearly $r_t(H_{k-1}(M, \Z)) \le  r(\Im(\partial_k))$.
    We thus have
    \[
        r(C_k) \ge  (r_0 + r_t)(H_k(M, \Z)) + r_t(H_{k-1}(M, \Z)),
    \] 
    completing the proof.
\end{proof}

Lastly, this allows us to get a stronger version of the weak Morse inequalities
\begin{align*}
    \# \Crit_f &= \sum_{k=0}^{n} r(C_k) \ge  \sum_{k=0}^{n} r_0(H_k(M, \Z)) + \sum_{k=0}^{n}r_t(H_k(M, \Z)) + r_t(H_{k-1}(M, \Z))\\
               &= \sum_{k=0}^{n} r_0 (H_k(M, \Z)) + 2 \sum_{k=0}^{n-1} r_t(H_k(M, \Z))
,\end{align*} 
where we used that $H_n(M, \Z) = \ker \partial_n$. There is no $\partial_{n+1}$ thus $r_t(H_n(M, \Z)) = 0$.


\section{Comparing homology over $\Z$ and $\Z / p \Z$}

Denote by $C_k(\Z)$ the $k$-th Morse Chain group with coefficients in $\Z$, by $C_n(\frac{\Z}{n\Z})$ the $k$-th chain group with coefficients in  $\Z / n \Z$.
We have the short exact sequence of chain complexes:
\[
    0 \to  C_k(\Z) \xrightarrow{\cdot n} C_k(\Z) \xrightarrow{p} C_k(\Z / n \Z) \to  0
.\] 
This gives rise to the long exact sequence in homology
\[
    \cdots \to  H_{i+1}(M, \Z / n \Z) \xrightarrow{\partial}  H_i(M, \Z) \xrightarrow{n} H_i(M, \Z) \xrightarrow{p_*} H_i(M, \Z / n \Z) \xrightarrow{\partial}  H_{i-1}(M, \Z) \to  \cdots
.\] 
Long exact sequences can always be truncated to short exact sequences. In this case, we want to truncate as follows:
\[
    0 \to  \Ker \partial \to  H_i(M, \Z / n \Z) \xrightarrow{\partial}  \Im (\partial) \times H_{i-1}(M, \Z) \to  0
.\] 
We can descripte the groups on the left and the right more precisely.
\begin{description}
    \item[$\ker \partial$ ]: The long sequence, exact, so $\Ker \partial = \Im(p_*) \subset  H_1(M, \Z / n\Z)$. Now,
        \[
            p_*: H_i(M, \Z) \to  H_i(M, \Z n \Z)
        \] 
        simply takes coefficients mod $n$, so the image is  $\Im(p_*) = H_i(M, \Z) \otimes \Z / n \Z$
    \item $[\Im \partial$] Again, because of exactness of the long exact sequence,  $\Im \partial = \ker(n \cdot)$,
        \[
            n \cdot: H_{i-1}(M, \Z) \to  H_{i-1}(M, \Z)
        .\] 
        By definition, $\Ker(n \cdot) = H_{i-1}(M, \Z)_{\text{$n$-torsion}}$.
\end{description}
This means we obtain a short exact sequence
\[
    0 \to  H_i(M, \Z) \otimes \Z / n \Z \to  H_i(M, \Z / n \Z) \to  H_{i-1}(M, \Z)_{\text{$n$-torsion}} \to  0
.\] 
which allows us to compute $\Z / n \Z$ valued cohomology for all $n$.

Furthermore, for primes $p$,  $\Z$-Morse inequalities immediately imply $\Z / p \Z$ morse inequalities for finitely many $p$.

Get Morse inequalities for all  $p$ by direct calculation.

\section{StackOverflow answer}

Given a non-negative integer $n$ and a space $X$, there is a long exact sequence
 \[
     \cdots \to  H_{i+1}(X, \Z / n) \to  H_i(X, \Z) \xrightarrow{n \cdot}  H_i(X, \Z) \to  H_i(X, \Z / n) \to  H_{i-1}(X, \Z) \to  \cdots
\] 
and thus a short exact sequence
\[
    0 \to  H_i(X, \Z) \otimes \Z / n \to  H_i(X, \Z / n) \to  H_{i-1}(X, \Z)_\text{$n$-tors}  \to  0
,\] 
(where $M_{\text{$n$-torsion}}$ denotes the $n$-torsion elements in an abelian group $M$).
If $X$ has the homotopy type of a finite CW complex, then the $H_i(X, \Z)$'s are trivial for all but finitely many indices $i$ and they are all of finite type. If  $p$ is a prime number, in the case where neither  $H_i(X, \Z)$ or $H_{i-1}(X, \Z)$ have $p$-torsion elements (which is the case for all but finitely many $p$), we have
\[
    \dim_{\Q} H_i(X, \Z) = \rank H_i(X, \Z) = \dim_{\F_p }H_i(X, \F_p)
.\] 
There fore, controlling ranks with coefficients in finite fields $\F_p$ for all but finitely many prime numbers  $p$ is the same as controlling the dimension with coefficients in  $\Q$ (or the rank with coefficients in $\Z$)
The fact that these inequalities hold for all $p$ is thus stronger than having them only for all but finitely many of them, or just for  $p = 2$.

The proof of Morse inequalities (weak or strong) just goes through for  $\F_p$-linear coefficients for all  $p$. There is no reason to focus on  $2$-torsion.

If you work with $R$-linear coefficients for another commutative ring $R$, the same reasoning apply: you will be interested in fact by ranks with coefficients in  $K$ where  $K$ runs over all residu fields of  $R$. But for a field  $K$, you will have  $H_i(X, K) \cong H_i(X, \F) \otimes_\F K$ where $\F \subset K$ is the prime field contained in $K$. Therefore, knowing Morse inequalities for all  $p$ is the same as Morse inequalities for all coefficient rings for which the notion of rank makes sense.

Finally, the original work of Morse and Smale is with  $\Z$-linear coefficients

\end{document}
