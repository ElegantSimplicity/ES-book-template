\chapter{Stable and unstable manifolds}
\label{chap:stable-and-unstable-manifolds}

% \startcontents[chapters]
% \printcontents[chapters]{}{1}{}


In this chapter, we address two natural questions that came up when discussing examples of handle decompositions: `When can we reorder handles in a handle decomposition?' and `Under what conditions can we cancel two critical points via an isotopy?'
To answer these questions, we introduce the concept of stable and unstable manifolds.
While we previously viewed handles as a local phenomenon, these new concepts will allow us to understand how multiple handles can interact on a more global scale.

\section{Definition of stable and unstable manifolds}
Let us get straight to the point and give the definition of (un)stable manifolds.
\begin{marginfigure}
    \centering
    \incfig{stable-and-unstable-manifolds-are-manifiolds}
    \caption{Locally in a Morse chart, stable and unstable manifolds are given by the vertical and horizontal axis, i.e.\ $ x_1= \cdots= x_k = 0$ and $x_{k+1}= \cdots= x_n = 0$.}
    \label{fig:stable-and-unstable-manifolds-are-manifiolds}
\end{marginfigure}
\begin{definition}
    Let $p$ be a critical point of a Morse function $f$.
    Denote by $\psi^{t}$ the flow of a pseudo-gradient.
    Then the unstable manifold is defined as
    \[
        \unstable{p} = \big\{x \in M  \mid  \lim_{t \to -\infty} \psi^{t} (x)  = p\big\} 
    ,\] 
    and its stable manifold is defined as
    \[
        \stable{p} = \big\{x \in M  \mid  \lim_{t \to \infty} \psi^{t} (x)  = p\big\} 
    ,\] 
\end{definition}
Just like their names imply, these sets are indeed manifolds.
Around a critical point $p$, the unstable manifold $\unstable{p}$ is in a Morse chart $U$ given by $x_{k+1} = \cdots = x_n = 0$, so it is diffeomorphic to an open disk $\odisk{k}$.
The points in $\unstable{p}$ that lie outside $U$ can then be obtained by flowing the boundary of $\odisk{k}$ along the pseudo-gradient, which is diffeomorphic to $\sphere{k-1} \times \R$. All in all when gluing this to $\odisk{k}$, we end up with something that is indeed a manifold that is diffeomorphic to $\odisk{k}$.
A similar reasoning for the stable manifold shows that $\stable{p} \cong \odisk{n - k}$.
In summary, we have obtained the following result:

\begin{prop}
    Stable and unstable manifolds of a critical points are submanifolds diffeomorphic to open disks. Moreover,
    \[
        \dim \unstable{p} = \codim \stable{p} = \Ind p
    .\] 
\end{prop}

\begin{eg}
    Let us consider $T^{2}$ embedded in $\R^3$ in the standard way and consider the height function. 
    This function has $4$ critical points and is clearly Morse.
    Let $X = \grad f$ be the gradient of $f$ w.r.t.\ the standard metric on $\R^{3}$.
    Then the stable and unstable manifolds of $X$ are illustrated in Figure~\ref{fig:torus-height-function-stable-and-unstable-manifolds}.
    \label{eg:torus-stable-unstable-manifolds-standard-gradient}
    We have similarly done so for an embedding of $S^2$ in Figure~\ref{fig:other-sphere-definition-of-mathcal-m}.
\end{eg}

    \begin{figure}[H]
        \centering
        \incfig{torus-height-function-stable-and-unstable-manifolds}
        \caption{
        Stable and unstable manifolds for all critical points of the height function on the torus with the standard gradient on $\R^3$.
    All of them are diffeomorphic to either $\odisk{2}$, $\odisk{1}$ or  $\odisk{0}$.
    Note that $\stable{b}$ and  $\unstable{c}$ do not intersect transversely.
}
        \label{fig:torus-height-function-stable-and-unstable-manifolds}
    \end{figure}

    \begin{figure}[H]
        \centering
        \incfig{other-sphere-definition-of-mathcal-m}
        \caption{
            Stable and unstable manifolds of the `other sphere' with standard gradient on $\R^3$.  Note that all of them intersect transversely.
        }
        \label{fig:other-sphere-definition-of-mathcal-m}
    \end{figure}


    \begin{marginfigure}
        \centering
        \incfig{when-can-we-attach-multiple-handles-at-the-same-time}
        \caption{
            A cobordism from $S^{1}$ to $S^{1} \sqcup S^1 \sqcup S^1$.
            Stable and unstable manifolds do not intersect, which implies we can reorder the critical points $p$ and $q$.
        }
        \label{fig:when-can-we-attach-multiple-handles-at-the-same-time}
    \end{marginfigure}
    The stable and unstable manifolds give us information about interaction of handles.
    For example, consider the situation in Figure~\ref{fig:when-can-we-attach-multiple-handles-at-the-same-time}.
    Here, we can flow the two handle structures to the bottom without intersecting, which implies we can attach the two handles at the same time, i.e.\ one is not dependent on the other.
    Here, `flowing of handle structures' means flowing along unstable manifolds.
    An obtrusion to attaching two handles at the same time would then be that they intersect before reaching the bottom. We will make this more precise later, but it gives an idea as for why we might be interested in intersections of (un)stable manifolds.
    \section{Intersections of stable and unstable manifolds}
    %TODO: figure.
    %TODO: check for use of descending/ascending.
    Let us now investigate the condition under which stable and unstable manifolds do not intersect.
    Suppose $p$ and $q$ are two critical points of a Morse function $f: M \to  \R$.
    Let's consider $\stable{p} \cap \unstable{q}$.
    If $f(p) > f(q)$, then  $\stable{p} \cap \unstable{q} = \O$, because $f(\stable{p}) > f(\unstable{q})$.
    What can we say if $f(p) \le  f(q)$?
    If we assume that the intersection is transverse, Proposition~\ref{prop:transverse-codimensions-add} implies that the codimensions add in the following sense:
    \[
        \codim(\stable{p} \tcap  \unstable{q}) = \codim \stable p + \codim \unstable q
    ,\] 
    so we have
    \[
        \dim(\stable p \tcap \unstable q) = \Ind q - \Ind p
    .\]
    In particular, we have that if $\Ind q < \Ind p$,  $\stable{p} \tcap \unstable{q} = \O$, which will turn out to mean that we can always attach lower index handles before higher index ones.


    This transversality condition has a name:
    \begin{definition}[Smale condition]
        A pseudo-gradient field addapted to a Morse function $f$ is said to satisfy the \emph{Smale condition} if for all pairs of critical points $ \{p, q\}  \subset \Crit f$, we have that $\stable{p}$ intersects  $\unstable{q}$ transversely, i.e. 
        \[
            \stable{q} \tcap  \unstable{p} \text{ for all $p, q \in \Crit f$}
        .\] 
    \end{definition}
    It turns out that this condition is not at all restricting: we can always perturb the pseudo-gradient field such that it satisfies the Smale condition. The proof of this is rather technical and we refer the reader to Audin and Damian\sidecite{audin}.
    \begin{theorem}
        Any Morse function $f:M \to  \R$ admits a pseudo-gradient field that satisfies the Smale condition
    \end{theorem}

    Almost all of the previous examples we have given satisfy the Smale condition with the one exception being the torus.
    \begin{noneg}
        The gradient vector field in Example~\ref{eg:torus-stable-unstable-manifolds-standard-gradient} does not satisfy the Smale condition: the intersection of $\stable{b}$ and  $\unstable{c}$ is not transverse.
        Even more, if the condition were satisfied, there would be no trajectories connecting $b$ and $c$, because both have index $1$, so $\dim (\stable{b} \tcap \unstable{c}) = 0$. However, as illustrated in the figure, we have two such paths.
    \end{noneg}
    \begin{eg}
        Instead of considering the embedding as in the previous example, we can embed the torus at a slight angle.
        Then the gradient of the height function (by using the standard metric on $\R^3$) does satisfy the Smale condition.
        The intersection of $\stable{b}$ and  $\unstable{c}$ is not tangent any more: indeed, they do not even intersect at all. We have illustrated this below in Figure~\ref{fig:torus-tilted-height-function-stable-and-unstable-manifolds}.
        \label{eg:tilted-torus}
    \end{eg}
    \begin{figure}[H]
        \centering
        \sidecaption{
            When embedding the torus in $\R^3$ tilted, the Smale condition is satisfied: all stable and unstable manifolds intersect transversely.
            Indeed, stable and unstable manifolds of $c$ and $b$ don not intersect at all.
        \label{fig:torus-tilted-height-function-stable-and-unstable-manifolds}
        }
        \incfig{torus-tilted-height-function-stable-and-unstable-manifolds}
    \end{figure}
    \begin{eg}
        As illustrated in Figure~\ref{fig:other-sphere-definition-of-mathcal-m}, the `other sphere' with gradient of the height function w.r.t the metric of $\R^3$ does satisfy Smale condition.  \end{eg}

The Smale condition also has another interesting consequence, which we have not touched upon.
If stable and unstable manifolds intersect transversely, we know that the intersection is again a submanifold.
So $\stable{q} \tcap \unstable{p}$ is a manifold for all critical points $p, q \in \Crit f$.
This submanifold consists of all points on the trajectories connecting $p$ to $q$.
 \begin{definition}
    Let $f: M \to  \R$ be a Morse function and $\psi^{t}$ the flow of a pseudo-gradient that satisfies the Smale condition.
    Then we define
    \begin{align*}
        \traj{p}{q} &= \stable{q} \tcap \unstable{p}\\
                    &= \big\{
            x \in M 
            \mid 
            \lim_{t \to -\infty} \psi^{t}(x) = p, \ 
            \lim_{t \to \infty} \psi^{t}(x) = q
        \big\} 
    ,\end{align*} 
    which is a submanifold of dimension $\Ind p - \Ind q$.
\end{definition}

\begin{eg}
    Consider the `other sphere'. 
    We have illustrated $\traj{p}{q}$ below for some of the critical points of the height function.
    Here we see that these type of submanifolds do not need to be connected. For example, $\traj{b}{a}$ is diffeomorphic to the disjoint union of two open intervals.
\end{eg}

Instead of considering the set of points lying on all trajectories from $p$ to $q$, $\traj{p}{q}$, we can also construct a set where each point corresponds to exactly one trajectory.
We do this by modding out $\traj{p}{q}$ by $\R$-action of translations in time. We denote the resulting space with $\L{p}{q}$.
More explicitly, we have the following:

\begin{prop}
    Let $f: M \to  \R$ be a Morse function and $\psi^{t}$ the flow of a pseudo-gradient field satisfying the Smale condition.
    Then the group $(\R, +)$ of time translations acts on $\traj{p}{q}$ by  $t \cdot x = \psi^{t}(x)$.  If $p \neq q$ then the action is free and we can define $ \L{p}{q} = \traj{p}{q} / \R $. The dimension of $\L{p}{q}$ is $\Ind p - \Ind q - 1$.
\end{prop}
\begin{proof}
    It is clear that $\R$ acts on $\traj{p}{q}$ by time translations. 
    If $p \neq q$,  $\traj{p}{q}$ does not contain any critical point, so flowing along a pseudo-gradient field, the value of $f$ is strictly decreasing. This proves freeness.
\end{proof}

\begin{remark}
    If the index of two points only differs by one, say $\Ind p = \Ind q + 1$, then the dimension of $\L{p}{q}$ is $0$, so it is a discrete set.
    This proves that the number of trajectories from $p$ to $q$ is always countable.
    We will later prove that it is in fact finite.
    \label{remark:trajectories-finite}
\end{remark}
\begin{remark}
    Another way to look at $\L{p}{q}$ is to consider an $a \in \R$ such that $f(p)<a<f(q)$. Then every flow line from $p$ to $q$  intersects $f^{-1}(a)$ exactly once (the value of $f$ is decreasing along the way), so we can identify  $\L{p}{q}$ with  $\traj{p}{q} \cap f^{-1}(a)$.
\end{remark}




\begin{figure}[H]
    \centering
    \incfig{mathcal-m-trajectories-other-sphere}
    \caption{
        Illustration of $\traj{p}{q}$ and $\L{p}{q}$ for critical points of the `other sphere'.
    }
    \label{fig:mathcal-m-trajectories-other-sphere}
\end{figure}



% \todo{Mention that handles are thickened up descending manifolds.}


\filbreak
\section{Reordering critical points}
Now that we have introduced the (un)stable manifolds, we are ready to prove a first reordering theorem. The statement and proof can be found in notes of `Lectures on the $h$-Cobordism Theorem' by Milnor\sidecite{hcobord}.
\begin{theorem}
    Let $f: M \to  \R$ be a Morse function on a cobordism $M$ from $M_0$ to $M_1$ such that $ M_0 = f^{-1}(0)$ and $ M_1 = f^{-1}(1)$ with two critical points $p$ and  $p'$.
    Suppose that for some choice of pseudo-gradient field $X$, the stable and unstable manifolds do not intersect.
    Let $a, a' \in (0,1)$ be arbitrary.
    Then there exists a new Morse function $g$ such that
    \begin{enumerate}[(a)]
        \item $X$ is a gradient-like vector field for  $g$
        \item The critical points of  $g$ are still $p, p'$ and $g(p) = a$,  $g(p') = a'$.
        \item $g$ agrees with  $f$ near $M_0 \sqcup M_1$ and equals $f$ plus a constant in some neighbourhood of  $p$ and some neighbourhood of  $p'$.
    \end{enumerate}
    \label{thm:reordening}
\end{theorem}
\begin{marginfigure}
    \centering
    \incfig{reordening-theorem-milnor-h-cobordism}
    \caption{
        Construction of $\overline{\mu}$ and $\pi$ in the proof on reordening critical points.
    }
    \label{fig:reordening-theorem-milnor-h-cobordism}
\end{marginfigure}
\begin{proof}
    We want to mask out the area around the stable and unstable manifolds of one of the critical points.
    Let $\mu: M_0 \to  [0,1]$ be a smooth map that is $0$ around  $ M_0 \cap \unstable p$ and $1$ around $M_0 \cap \unstable{p'}$, as illustrated in Figure~\ref{fig:reordening-theorem-milnor-h-cobordism}.
    Then we can smoothly extend this to a function on the whole manifold $M$ as follows.
    Define $\pi: M \to  M_0$ by flowing along the pseudo-gradient field until we reach $M_0$.
    Then we can extend $\mu$ uniquely to a smooth function that is constant on each trajectory by defining
    \begin{align*}
        \overline{\mu}: M &\longrightarrow [0,1] \\
        x &\longmapsto \begin{cases}
            0 & \text{if $x$ in stable or unstable manifold of $p$}\\
            1 & \text{if $x$ in stable or unstable manifold of $p'$}\\ 
            \mu(\pi(x)) & \text{else}
        \end{cases}
    .\end{align*}
    We have illustrated this in the figure by colouring regions where $\mu$ is large red.

    Define a new Morse function $g: M \to  [0,1]$ by $g(q) = G_{\overline{\mu}(q)}(f(q))$, where $G_{s}(x)$ is a smooth family of smooth functions $G_s: [0,1] \to  [0,1]$ with $s \in [0,1]$ that has the following properties, also indicated in Figure~\ref{fig:proof-reordening-properties-of-g}.
    \begin{marginfigure}
        \centering
        \incfig{proof-reordening-properties-of-g}
        \caption{Necessary properties of $G$ in the proof on reordering critical points are indicated in yellow.}
        \label{fig:proof-reordening-properties-of-g}
    \end{marginfigure}
    \begin{enumerate}[(1)]
        \item For all $s$, $G_s' > 0$ and $G_s$ increases from $0$ to  $1$ as $x$ increases from  $0$ to  $1$
        \item  $G_0(f(p)) = a$\\ $G_1(f(p')) = a'$
        \item  $G_s(x) = x$ for $x$ near  $0$ or  $1$ and for all $s$ 
        \item  $G_0'(x) = 1$ for $x$  in a neighbourhood of $f(p)$\\
        $G_1'(x) = 1$ for $x$  in a neighbourhood of $f(p')$
    \end{enumerate}
    Property (b) and (c) in the statement of the theorem are clear: they follow immediately from (2), (3) and (4).
    For (a), consider
    \[
        dG = \frac{\partial G}{\partial \overline{\mu}}  d\overline{\mu} + \frac{\partial G}{\partial f}  df
    .\] 
    Plugging in $X$, we have
    \begin{align*}
        dg(X) &= \frac{\partial G}{\partial \overline{\mu}}  d\overline{\mu}(X) + \frac{\partial G}{\partial f}  df(X)\\
              &= \frac{\partial G}{\partial f}  df(X) < 0 \text{, except at critical points of $f$}
    ,\end{align*} 
    where we used that $d\overline{\mu}(X) = 0$ by construction of $\overline{\mu}$, $\frac{\partial G}{\partial f} > 0$ by (1) and $df(X) < 0$ everywhere except at critical points of $f$ by definition of pseudo-gradient field.
    Because of (4), this implies that $g$ and $f$ share the same behaviour around their critical points, proving that $g$ is also Morse.
\end{proof}

\begin{remark}
    This theorem can be extended to a more general setting. Suppose we have a set of points $\mathbf{p} = \{p_1, \ldots, p_k\}$ and $\mathbf{p}' = \{p_1', \ldots, p_\ell'\}$, with all $p_i$ at the same level and all $p_i'$ at a single level.
    Then the theorem remains valid, with exactly the same proof.
\end{remark}

If the pseudo-gradient vector field $X$ satisfies the Smale condition, then we know the intersection of stable and unstable manifolds very well.
For example, if $\Ind p \le  \Ind q$, the intersection of $\unstable{p}$ and  $\stable{q}$ is empty.
Applying this reordering theorem multiple times, we have the following

\begin{theorem}
    Any closed manifold admits a Morse function such that for all critical points, $\Ind p < \Ind q \implies f(p) < f(q)$.
    In particular, it admits a Morse function, which satisfies $ \Ind p  = f(p) $ for all critical points $p \in \Crit f$.
\end{theorem}
In other words, this asserts that lower index handles can always be attached before higher index ones.
Morse functions as described in the second part of the theorem have a name:
\begin{marginfigure}
    \centering
    \incfig{self-indexing-morse-function-torus-tilted}
    \caption{When tilting the torus to the right angle, the height function becomes self-indexing.}
    \label{fig:self-indexing-morse-function-torus-tilted}
\end{marginfigure}
\begin{definition}[Self-indexing Morse function]
    A Morse function $f: M \to  \R$ is \emph{self-indexing} if $\Ind p = f(p)$ for all critical points  $p$ of  $f$.
\end{definition}
\begin{eg}
    Consider $T^2 \subset \R^3$ embedded at an angle. We have illustrated the side view in Figure~\ref{fig:self-indexing-morse-function-torus-tilted}.
    Then the height function is a self-indexing Morse function.
\end{eg}


% \section{Smale condition is generic (proof)}
\section{Cancellation of critical points}
The second question that arose, was `Under what conditions can we cancel a pair of critical points'.
The answer to this question is more complicated than the previous one.
We will first state the theorem and then investigate the conditions it imposes.
\begin{marginfigure}
    \centering
    \incfig{canellation-of-critical-points-situation}
    \caption{A situation where we can cancel two critical points. $\ell$ is the unique gradient line between $p$ and $q$ and any other gradient line starting around $p$ reaches $f^{-1}(f(q) - \epsilon)$.}
    \label{fig:canellation-of-critical-points-situation}
\end{marginfigure}
\begin{theorem}[Francois Laudenbach\sidenotemark]
    Let $f: M \to  \R$ be a Morse function and $X$ be a pseudo-gradient field.
    Let $p$ and $q$ be two critical points such that
    \begin{enumerate}[(i)]
        \item $p$ and $q$ are connected by a unique gradient line  $\ell := \traj{p}{q}$.
        \item For some $\epsilon>0$, each gradient flow line starting in $\unstable{p}$ distinct from $ \ell$ crosses the level set $f^{-1}(f(q) - \epsilon)$
    \end{enumerate}
    Then these two points are cancelable.
More precisely, there exists an isotopy of functions $f_t$ starting at $f_0=f$ supported in an open neighbourhood of the closure of  $\unstable{p} \cap  \{f \ge  f(q) - \epsilon\} $ such that
\begin{itemize}
    \item $f_t|_U$ is Morse with two critical points when  $0 \le t\le \frac{1}{2}$;
    \item $f_t|_U$ has a cubic singularity when $t=\frac{1}{2}$
    \item $f_t|_U$ has no critical points when $\frac{1}{2}<t\le 1$.
\end{itemize}
\end{theorem}
\sidenotetext[][0.0cm]{\fullcite{laudenbach2013proof}}

\begin{remark}
    The first condition implies $\dim \traj{p}{q} = 1$, so $\Ind p = \Ind q + 1$.  This means we can only cancel critical points if their indices differ by one.
\end{remark}
The proof of this theorem is somewhat technical but nonetheless it is quite transparent and reduces the general case to dimension one, where the situation is simple.
Let us show in this one dimensional context why the two conditions of the theorem are important.

Suppose we have two critical points, one of index $0$ and one of index $1$ and have a look at the figure below.
If the conditions of the theorem are satisfied, we are in the situation on the left. There is a unique flow line and other flow lines leaving $p$ eventually going lower than $q$. It is quite intuitive that we can construct a cancellation, as illustrated.

If we have a unique flow line, but other flow lines leaving $p$ do not reach below $q$, then we cannot cancel $p$ and $q$ while fixing the end segments of the curve.
This is the second situation below.
Here, the reason that other flow lines than $\ell$ cannot reach below $q$ is the existence of an additional critical point $r$ nearby.
We are required to fix the end segments, because otherwise we would change the behaviour at $r$, and while doing so, we simply cannot cancel $p$ and $q$.
Note that in this specific situation, cancelling $r$ and $p$ does work.

Lastly, if we do not have a unique flow line, the third situation in the figure could occur. It is clear that we cannot cancel $p$ and $q$.

\begin{figure}[H]
    \centering
    \sidecaption[][-1cm]{Three situations.
        The first one satisfies the conditions of the theorem and we can cancel the critical points.
        The second only satisfies (i), the third only (ii) and cancellation is impossible.  \label{fig:one-dimensional-reduction}}
    \incfig{one-dimensional-reduction}
\end{figure}

\begin{marginfigure}
    \centering
    \incfig{other-sphere-cancelation-of-critical-points}
    \caption{The `other sphere' with trajectories between critical points whose index differ by exactly one. We can only cancel $b$ and $c$ or $d$ and $b$. Cancelling $b$ and $a$ is impossible.}
    \label{fig:other-sphere-cancelation-of-critical-points}
\end{marginfigure}
\begin{eg}
    As an example, let us consider the `other sphere' as in Figure~\ref{fig:other-sphere-cancelation-of-critical-points}.
   We also added flow lines between critical points whose index differs by exactly one.
   It is clear that the pairs $(c,b)$ and $(d, b)$ satisfy the conditions of the theorem, and so we can cancel them.
   The pair $(b,a)$ does not, because there are two flow lines from $b$ to $a$. Intersecting the `other sphere' with a plane containing $\traj{b}{a}$, we see that we are exactly in the third situation discussed earlier.
\end{eg}

\section{Heegaard splittings}
\begin{marginfigure}
    \centering
    \incfig{heegaard-splittings-schemattically}
    \caption{Schematic visualization of a self-indexing Morse function on a $3$-manifold $M$.
        The manifold $S = f^{-1}(\frac{3}{2})$ is called the splitting surface of $M$.
    }
    \label{fig:heegaard-splittings-schemattically}
\end{marginfigure}
In the three dimensional setting, an important consequence of the existence of self-indexing Morse functions are so-called Heegaard splittings.
Such a Morse function gives rise to a handle decomposition schematically shown in Figure~\ref{fig:heegaard-splittings-schemattically}. When splitting the manifold along $f^{-1}(\frac{3}{2})$, we decompose $M$ in two parts: a part that consists of $0$- and $1$-handles glued along its boundary to a part consisting only of $2$- and  $3$-handles.
However the latter can also be seen as being constructed of $0$- and $1$-handles, simply by building $M$ from top to bottom by considering the Morse function $-f$ instead of $f$. This then interchanges $k$- and $n-k$-handles, or in the case of $n=3$, $2$-handles  become $1$-handles and $3$-handles become $0$-handles.
All things considered, we have a decomposition of $M$ in two so-called handlebodies:
\begin{definition}[Genus $k$ handlebody\sidenotemark]
    A genus $k$ handlebody is a compact connected orientable $3$-manifold with boundary that possesses a handle decomposition consisting of $0$-handles and $1$-handles such that its boundary is a surface of genus $k$.
\end{definition}
A Heegaard splitting is then defined in the following way:
\begin{definition}[Heegaard splitting\sidenotemark]
    A \emph{Heegaard splitting} of a closed $3$-manifold $M$ is a decomposition $M = V \cup _S W$ such that $V$ and $W$ are handlebodies and  $S = \partial V = \partial W$. Here  $S$ is called the \emph{splitting surface of $M$}.
    Two Heegaard splittings are considered \emph{equivalent} if their splitting surfaces are isotopic. The \emph{genus} of a Heegaard splitting is the genus of $S$.
\end{definition}
\sidenotetext[][-5cm]{\fullcite{schultens2014introduction}}

With these definitions, we can summarize our findings as follows:
\begin{theorem}[Moise]
    Every closed orientable $3$-manifold admits a Heegaard splitting.
\end{theorem}
\begin{proof}
    Let $f$ be a self-indexing Morse function on $M$.
    Then $M = f^{-1}\left[0, \tfrac{3}{2}\right] \cup f^{-1}\left[\tfrac{3}{2}, 3\right]$ is a Heegaard splitting of $M$ by duality of $k$- and $n-k$-handles.
\end{proof}

Let us start off with some examples.
\begin{marginfigure}
    \centering
    \incfig{genus-four-heegaard-splitting-of-s3}
    \caption{A genus four Heegaard splitting of $S^{3}$, seen as the one point compactification of $\R^3$.
        This way, we can obtain a Heegaard splitting of $S^{3}$ of any genus.
    }
    \label{fig:genus-four-heegaard-splitting-of-s3}
\end{marginfigure}
\begin{eg}
    As seen earlier, the `height' function on $S^{3}$ gives rise to a handle decomposition $S^{3} = \cdisk{2} \cup \cdisk{2}$. This is a Heegaard splitting of genus $0$ with splitting surface $S^{2}$.
\end{eg}
\begin{eg}
    The previous example is not the only way to decompose $S^{3}$.
    Indeed, we can arbitrarily add cancelable $1$- and $2$-handles to increase the genus. 
    For example a genus four Heegaard splitting of $S^{3}$ is illustrated in Figure~\ref{fig:genus-four-heegaard-splitting-of-s3}.
    Here, we visualized $S^{3}$ as the one point compactification of $\R^3$.
    \label{eg:s3-heegaard-four-genus}
\end{eg}

\begin{eg}
    If we split the handle decomposition of $T^{3}$ given in Example~\ref{eg:handle-decomposition-three-torus} between index $1$ and $2$ critical points, then we find a splitting surface that has genus three.
\end{eg}

% \subsection{Stabilization and destabilization}
\begin{marginfigure}
    \centering
    \incfig{stabilization}
    \caption{
        The act of stabilization is replacing a ball near the boundary as illustrated, increasing the genus of the splitting surface by one.}
    \label{fig:stabilization}
\end{marginfigure}
\begin{remark}
    The act of increasing the genus, as done in Example~\ref{eg:s3-heegaard-four-genus} is called stabilization.
    It can be done in as in Figure~\ref{fig:stabilization}.
    The reverse procedure, called destabilization cannot be done in general.
\end{remark}

% \begin{definition}
%     The Heegaard genus of a 3-manifold $M$ is the smallest possible genus of a Heegaard splitting.
%     We then say the Heegaard slitting is \emph{minimal}.
% \end{definition}


