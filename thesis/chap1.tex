\chapter{Morse theory}

\begin{definition}[Critical point]
    Let $M$ and $N$ be a manifolds and  $f$ a map from $M$ to $N$.
    The critical points of $f$ are
    \[
    \Crit f = \{p \in M  \mid \text{$df_p$ is not surjective}\} 
    .\] 
    In particular, if $N = \R$ we have that
    \[
    \Crit f = \{ p \in M  \mid  df_p = 0\} 
    .\] 
\end{definition}

\begin{definition}[Morse function]
    Let $M$ be a manifold. A function $f: M \to  \R$ a \emph{Morse function} if for all critical points $p$, there exists a chart centred around $p$ such that $f$ is locally given by
    \[
        f(x) = f(p) -x_1^2 - \cdots - x_k^2 + x_{k+1}^2 + \cdots + x_n^2
    .\] 
    We call $k$ the index of $p$, which we also denote with $\Ind p$.
    
\end{definition}
Intuitively, the index of a critical point $p$ is ``the number of downward directions''.
Let us first give some examples.
\begin{marginfigure}
    \centering
    \incfig{examples-of-morse-functions}
    \caption{Example of Morse function on the torus. }
    \label{fig:examples-of-morse-functions}
\end{marginfigure}
\begin{eg}
    Let $M$ be the torus  $T^2$ embedded in $\R^3$ as illustrated in \cref{fig:examples-of-morse-functions}.
    Then the height function $h: T^2 \to  \R$ which is the projection on the $z$-axis is a smooth map with four critical points:
    \begin{itemize}
        \item A minimum $m$, where $h(x) = h(m) + x_1^2 + x_2^2$.
        \item Two saddle points $s_1, s_2$ where $h(x) = h(s_i) + x_1^2 - x_2^2$.
        \item A maximum $p$ where $h(x) = h(p) - x_1^2 - x_2^2$.
    \end{itemize}
\end{eg}

\begin{eg}
    Let $M = \R^2$ and $f: \R^2 \to  \R: (x, y) \mapsto  x^2$.
    Then all points $(0, y)$ for  $y \in \R$ are critical points of this function.
    In particular, $(0, 0)$ is a critical point, but clearly $f$ is not Morse.
\end{eg}
\begin{marginfigure}
    \centering
    \incfig{non-example-of-morse-function}
    \caption{}
    \label{fig:non-example-of-morse-function}
\end{marginfigure}
\begin{marginfigure}
    \centering
    \incfig{non-examples-of-morse-functions}
    \caption{An example of a function that is not Morse: $f: \R \to  \R: x \mapsto  x^3$.
        Small perturbations of $f$ are Morse.
    }
    \label{fig:non-examples-of-morse-functions}
\end{marginfigure}
\begin{eg}
    Let $M = \R$ and $f : \R \to  \R: x \mapsto x^3$.
    Then $x = 0$ is a critical point, but $f$ is not Morse.
    Note however that if we add a small perturbation to $f$, say $g_t: x\mapsto x^3+ tx$, then for small non-zero $t$, $g$ is Morse. For $t < 0$, $g_t$  has two critical points: one of index $1$ and one of index $0$.
    If $t > 0$,  $g_t$ has no critical points.
\end{eg}

Note that this last case where $f$ has no critical points cannot happen if $M$ is compact.
Indeed, any function attains it maximum and minimum on a compact manifold, so we have at least two critical points.
On the other hand, the number of critical points is at most finite.
Indeed, the definition of a Morse function implies that critical points are isolated, which on a compact manifold implies that their number is finite.
This also immediately rules out the situation we had in the other example, where the set of critical points was a straight line.

\section{Handle decompositions}

\begin{figure}[H]
    \centering
    \incfig{morse-chart}
    \caption{morse chart}
    \label{fig:morse-chart}
\end{figure}


We can also define Morse functions on manifolds with boundary

% \begin{prop}
%     Let $M$ be a compact manifold and $f: M \to  \R$ a Morse function.
%     Then  $f$ has a finite number of critical points.
% \end{prop}


\section{Coordinate free definition of a Morse function}


The attentive reader will have noticed that the definition of the index of a critical point could possibly not be well defined.
To show that it is and that $\Ind p$ does not depend on the choice of coordinates, we will will give a coordinate free definition. For this, we first need to define the Hessian at a critical point.

\begin{definition}
    Let $M$ be a manifold and $f: M \to  \R$ a function.
    Let $p$ be a critical point of $f$.
    Then we define the Hessian $H_p$ to be the bilinear form
    \begin{align*}
        H_p: T_pM \times T_pM &\longrightarrow  \R\\
        (X, Y) &\longmapsto X (\tilde{Y} f)|_p
    ,\end{align*} 
    where $\tilde{Y}$ is a local extension of $Y$.
\end{definition}
This is a well defined symmetric bilinear form.\sidenote{The difference between $H_p(X, Y)$ and  $H_p(Y, X)$ is
\begin{align*}
    H_p(X, Y) - H_p(Y, X) &= X(\tilde{Y} f)|_p - Y (\tilde{X} f)|_p\\
                          &= [\tilde{X}, \tilde{Y}] f |_p\\
                          &= df_p [\tilde{X}, \tilde{Y}]|_p = 0
.\end{align*}
The value of $H_p$ also does not depend on the extension of the vector field.
Indeed, suppose $\tilde{Y}$ and $\overline{Y}$ are two different extensions of $Y$. Then by symmetry of $H_p$, we have
\[
    X(\tilde{Y} f)|_p = Y(\tilde{X} f)|_p = X(\overline{Y}f )|_p
.\] 
}
In case of a Morse function given locally by $f(x) = f(p) - x_1^2 - \cdots - x_k^2 + x_{k+1}^2 + \cdots + x_n^2$, the Hessian at $p$ is 
\[
    H_p = - dx_1^2 - \cdots - dx_k^2 + dx_{k+1}^2  + \cdots + dx_n^2
.\] 
Note in particular that $H_p$ is non-degenerate and its signature is $(i_-, i_{+}) = (k, n-k)$.
As the signature of a symmetric bilinear form is coordinate independent, this shows that the index of a critical point is as well.


Interestingly, the reverse is also true: if $H_p$ is non-degenerate for all critical points  $p$ of $f$, then  $f$ is a Morse function.
Many authors take this to be the definition of a Morse function.

\begin{lemma}[Morse Lemma]
    Let $M$ be a manifold and $f: M \to  \R$ a map.
    If for all $p \in \Crit f$, the Hessian $H_p$ is non-degenerate, then $f$ is Morse.
\end{lemma}
\begin{proof}
    We follow the proof of Milnor\sidecite{milnor}.
    We may assume that $M = \R^{n}$, $p$ is the origin and $f(p) = 0$.
    Then by a version of Taylor's theorem, we can write
    \begin{align*}
        f(x)  &= f(p) + \sum_{i=1}^{n} (x_i - p_i) g_i (x)\\
              &= \sum_{i=1}^{n} x_i g_i(x)
    ,\end{align*} 
    where $g_i$ are smooth functions. 
    Now, as $g_i(0) = \partial_i f (0) = 0$, we can repeat this for each  $g_i$, giving us the following:
    \[
        f(x) = \sum_{i, j= 1}^{n} x_i x_j h_{ij}(x)
    .\] 
    Because this sum is symmetric in $i$ and  $j$, we may assume that  $h_{ij}$ is symmetric as well.\sidenote{If $h_{ij}$ is not symmetric, we can define $h_{(ij)} = \frac{1}{2}(h_{ij} + h_{ji})$. Then $h_{(ij)}$ is symmetric and we still have $\sum x_{i}x_{j}h_{ij} = \sum x_{i}x_{j}h_{(ij)}$.}
    Note that
    \[
        h_{(ij)}(0) = \frac{1}{2} \partial_{ij} f(0)
    ,\]
    which is non-degenerate.

    Now we imitate the proof of diagonalization of a non-degenerate quadratic form.
    We do this by induction.
    Suppose we have coordinates $u_1, \cdots, u_n$ in a neighbourhood of $0$ such that
    \[
        f = \pm u_1^2 \pm \cdots \pm u_{r-1}^2 + \sum_{i,j\ge r} u_i u_j H_{ij}(u)
    ,\] 
    where $H_{ij}$ is a symmetric matrix.
    After a linear change, we may assume that $H_{rr} \neq 0$.
    Then define new coordinates $ v_1, \ldots, v_r$ as follows:
    \[
        v_i = \begin{cases}
            u_i & \text{if $i \neq r$}\\
            \sqrt{|H_{rr}|} (u_r + \sum_{i > r} u_i H_{ir} / H_{rr}) & \text{if $i = r$.}
        \end{cases}
    \] 
    Note that we may need to restrict the neighbourhood such that $\sqrt{H_rr} \neq 0$.
    Then we have that
    \[
        f = \sum_{i\le r} \pm v_i^2 + \sum_{i,j > r} v_i v_j H_{ij}'(v)
    .\] 
    for some symmetric matrix $H_{ij}'$.
    Indeed,
    \begin{align*}
        v_r^2 &= |H_{rr}| \left(u_r^2 + \sum_{i>r} 2 u_r u_i H_{ir} / H_{rr} + \left(\sum_{i>r} u_i H_{ir} / H_{rr}\right)^2\right)\\
              &= TODO
    .\end{align*} 

\end{proof}

\section{Existence and abundance}

Let $M$ be a compact manifold.
Then we can embed $M$ in $\R^{n}$ for some big enough $n$.

\begin{marginfigure}
    \centering
    \incfig{level-sets-of-distance-function-torus}
    \caption{An embedding of the torus $T^2$ in $\R^3$. The level sets of $f_p$ are spheres. We see that $f_p$ has four critical points: a maximum, a minimum and two saddle points.}
    \label{fig:level-sets-of-distance-function-torus}
\end{marginfigure}

\begin{prop}
    Let $V \subset \R^{n}$ be a submanifold.
    Then for almost every point $p \in \R^{n}$, we have that
    \[
    f_p : V \to \R: x \mapsto  \|x - p\|^2
    \] 
    is a Morse function.
\end{prop}

First, let us give an example of when $f_p$ is \emph{not} a Morse function.

\begin{marginfigure}
    \centering
    \incfig{example-when-fp-is-not-a-morse-function}
    \caption{When $p$ is the center of a circle, $f_p$ is not a Morse function}
    \label{fig:example-when-fp-is-not-a-morse-function}
\end{marginfigure}
\begin{eg}
    Let $V = S^1 = \{\|x\|^2 = 1  \mid  x \in \R^2\} $.
    Then $f_{(0, 0)}$ is not a Morse function.
    Indeed, $f_{(0, 0)} \equiv 1$, so in particular the second derivative is non-degenerate (it vanishes everywhere).
\end{eg}


More generally, $f_p$ is not a Morse function when infinitesimally close normals intersect in $p$, because then $f_p$ is constant (until ??) second order (in a certain direction \ldots ).
To make this exact, we will show that $H_p$ is non-degenerate if $p$ is a critical value of the map
\[
    E: NV \to  \R^{n}: (x, v) \mapsto x + v
,\]
where $N$ is the normal bundle of $V$.
Then Sard's theorem\sidecite{todo} will immediately imply that $f_p$ is Morse for almost all $p$.

\begin{figure}[H]
    \centering
    \sidecaption{TODO: existence of morse functions normal bundle map}
    \incfig{existence-of-morse-functions-normal-bundle-map}
    \label{fig:existence-of-morse-functions-normal-bundle-map}
\end{figure}

\begin{proof}
    Given our realisation, the proof now reduces to a straightforward calculation.
    First note that $x$ is a critical point of $f_p$ only if $x-p \perp T_x V$.
    Indeed:
    \[
        d f_p = \sum d(x_i - p_i)^2 = 2 \sum (x_i-p_i) dx_i = 0 \qquad \text{if $x-p \perp T_x V$}
    .\] 
    Let $d = \dim V$ and $(u_1, \ldots, u_d) \mapsto x(u_1, \ldots, u_d)$ be a local parametrization of $V$.
    Then we have
    \[
        \partial_i f_p = 2(x-p) \cdot  \partial_i  x
    \] 
    and for the hessian we have
    \[
    H_p = \partial_{ij} f_p = 2 \left(\partial_j x \cdot \partial_i x  + (x-p) \partial_{ij} x \right)
    ,\] 
    where we denoted $\partial_i = \frac{\partial }{\partial u_i}$.
    We will show that $H_p$ is not of full rank if and only if $p$ is a critical value of  
    \[
        E: NV \to  \R^n: (x, v) \mapsto x + v
    ,\] 
    where $NV$ is the normal bundle to $V$ w.r.t.\ the Euclidean metric on $\R^n$.

    First we define a local parametrization of $NV$:
    \[
        (u_1, \ldots, u_d, t_1, \ldots, t_{n-d}) \mapsto \Big(x(u_1, \ldots, u_d), \sum_{i=1}^{n-d} t_i v_i(u_1, \ldots, u_{d})\Big)
    ,\] 
    where the $v_i$ form a local orthonormal basis at each point, normal to $TV$.

\begin{marginfigure}
    \centering
    \incfig{abundance-of-morse-functions-parametrization-of-the-normal-bundle}
    \caption{TODO: abundance of morse functions parametrization of the normal bundle}
    \label{fig:abundance-of-morse-functions-parametrization-of-the-normal-bundle}
\end{marginfigure}

    \begin{TODO}
        In Audin and Damien $ v_i(u_1, \ldots, u_{n-d})?$
        orthonormal basis of $(T_xV)^{\perp}$? Orientable?
    \end{TODO}

    Then in these coordinates,
    \[
        \partial_i E = \partial_i x  + \sum_{k=1}^{n-d} t_k \partial_i v_k 
        \qquad
        \qquad
        \partial_{t_j} E  = v_j
    .\]
    To see whether these vectors are independent, we compute the inner products with the $n$ independent vectors
    \[
    \partial_1 x , \ldots, \partial_d x , v_1, \ldots, v_{n-d}
    .\] 
    This gives the following matrix with the same rank as $E_*$:
     \[
    \begin{pmatrix}
        (\partial_i x\cdot \partial_j x + \sum_k t_k \partial_iv_{k} \cdot \partial_j x) &  \sum_k \partial_i v_{k} \cdot  v_\ell\\
        0 & \operatorname{Id}
    \end{pmatrix}
    .\] 
    Therefore,\todo{This is wrong? Should be something along the lines of $+ \Rank I$?}
    \[
    \Rank E_* = \Rank \left(\partial_i x\cdot \partial_j x + \sum_k t_k \partial_iv_{k} \cdot \partial_j x\right)  
    .\] 
    Now in the second term, we can move $\partial_i$ from $v_k$ to $x$ and get a minus sign in return:
    \begin{align*}
        \partial_iv_{k} \cdot \partial_j x &= \partial_i (v_k \cdot \partial_j x)  - v_k \cdot  \partial_i \partial_j x\\
                               &= - v_k \cdot  \partial_i \partial_j x
    ,\end{align*} 
    as $v_k \perp \partial_j x$.
    So finally, we have
    \begin{align*}
    \Rank {E_*}_{(x, v)}
        &= \Rank \left(\partial_i x\cdot \partial_j x - \sum_k t_k v_{k} \cdot \partial_i \partial_j x\right)  \\
        &= \Rank \left(\partial_i x\cdot \partial_j x - v \cdot \partial_i \partial_j x\right)  \\
        &= \Rank \left(\partial_i x\cdot \partial_j x + (x-p) \cdot \partial_i \partial_j x\right)  \\
        &= \Rank H_p
    .\end{align*} 
    This concludes the proof.
\end{proof}




\begin{prop}
    Let $V$ be a manifold that can be embedded as a submanifold into a Euclidean space.
    Let $f: V \to  \R$ be a function.
    Let $k$ be an integer.
    Then $f$ and all its derivatives of order $\le k$ can be uniformly
    approximated by Morse functions on every compact subset.
\end{prop}

The idea of the proof goes as follows.
We choose an embedding of $V$ where $f$ is the first coordinate on $V$, so we can think of $f$ as a simple projection: $x \mapsto x_1$.
Then this function can be approximated in the following way:
\[
    x_1 \approx \frac{(x_1+c)^2 - c^2}{2c} \qquad \text{as $c \to  \infty$,}
\]
and even when considering the other dimensions, the approximation still works:
\[
    x_1 \approx \frac{\|x - p\|^2 - c^2}{2c} \qquad \text{with $p = (-c, 0, \ldots, 0)$ and $c\to  \infty$}
.\] 
Note that the right hand side is almost always a Morse function.

\begin{proof}
    Embed $V$ in $\R^{n}$ for $n$ sufficiently large such that $f$ is the first coordinate:
    \[
        h(x) = (f(x), h_2(x), \ldots, h_n(x))
    .\] 
    Let $c \in \R$. For almost every point $p = (-c + \epsilon_1, \epsilon_2, \ldots, \epsilon_n)$, the function
    \[
        g_c(x) = \frac{\|x - p\|^2 - c^2}{2c} 
    \] 
    is Morse.
    Then
    \begin{align*}
        g_c(x) &= \frac{1}{2c}  \sum (x_i - p_i)^2 - c^2\\
             &= \frac{1}{2c} \left((f(x) + c - \epsilon_1 )^2 + (h_2(x) - \epsilon_2)^2 + \cdots + (h_n(x) - \epsilon_n)^2\right)\\
             &= f(x) +  \frac{f(x)^2 + \sum h_i(x)^2}{2c} - \frac{\epsilon_1 f(x)  + \sum \epsilon_i h_i(x)}{c}  + \sum \epsilon_i^2 - \epsilon_1
    .\end{align*} 

    This concludes the proof.
    Indeed, let $K$ be a compact subset of $V$.
    The functions $\frac{d^{j}}{dx^{j}} (f(x)^2 + \sum h_i(x)^2)$ for $j = 1, \ldots,k$ all attain their maximum on $K$, so by choosing $c$ big enough, we can make them simultaneously arbitrarily small in a uniform way. Similarly for the third term.
    Lastly, we can also $\epsilon_i$ arbitrarily small while still retaining that $g$ is a Morse function.
\end{proof}

\begin{figure}
    \centering
    \sidecaption{
        We can approximate any smooth function with a Morse function.
        On the left, we plotted the level sets of $f$ itself.
        Because the coordinates are $f$ and $h_2$, these level sets are vertical planes.
        The two right plots show level sets of $g_c$ for $c=10$ and $c=100$, which are circles.
        We see that $g_c$ approximates $f$ if $c \to  \infty$.
    }
    \incfig{approximate-morse-functions}
    \label{fig:approximate-morse-functions}
\end{figure}
