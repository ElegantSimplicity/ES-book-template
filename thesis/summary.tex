\chapter*{Summary}
\label{ch:summary}
\addcontentsline{toc}{chapter}{\nameref{ch:summary}}

Morse theory is the study of smooth functions from manifolds to $\R$ that satisfy a non-degeneracy condition. While these functions are some of the simplest things that live on manifolds, it turns out that they can provide great insight in the structure of manifolds.

\bigskip
In \textbf{Chapter \ref{chap:morse-theory}}, we talk about the basics of Morse theory.
We give several equivalent definitions of a Morse function and give some examples.
We also show that Morse functions give rise to so-called handle decompositions: there is a one to one correspondence between critical points of a Morse function and so-called handles, building blocks from which any manifold can be built.
We give some very concrete examples of handle decompositions in dimensions one, two and three.
We end the chapter by showing that any manifold admits (infinitely many) Morse functions.
Even more, we show that they are generic and stable, meaning that any function can be approximated by a Morse function and if we perturb a Morse function, it stays Morse.

\bigskip

\textbf{Chapter \ref{chap:stable-and-unstable-manifolds}} concerns the concept of stable and unstable manifolds.
They form a way of understanding interaction of handles.
For example, two handles are `independent' if the intersection of the associated stable and unstable manifolds is empty, allowing us to reorder them, as we prove in Theorem~\ref{thm:reordening}.
This idea leads to the existence of self-indexing Morse functions, asserting that we can always build a manifold by first attaching $0$-handles, then $1$-handles, $2$-handles, etcetera, in that order.
We end the chapter by proving a first cancellation theorem, stating that under certain circumstances, we can cancel pairs of critical points.

\bigskip
In \textbf{Chapter \ref{chap:morse-homology}}, we introduce Morse homology.
We first define the Morse complex $C_k(f)$ and the Morse differential $\partial_X$, which is based on counting the number of trajectories along the (pseudo-)gradient vector field $X$ between critical points of a Morse function $f$.
Morse homology $\HM f X$ is the homology associated to this complex.
As an illustration, we compute some examples in two and three dimensions.
The rest of the chapter covers three important theorems in Morse theory.
We prove that the Morse complex is actually a complex ($\partial_X^2 = 0$), that Morse homology is independent of the Morse function and gradient, and lastly that Morse homology is actually isomorphic to singular homology.
The proofs of these theorems are very geometrical in nature and their ideas have inspired many other theories.

\bigskip
\textbf{Chapter~\ref{chap:app-morse-homology}} discusses some applications of Morse homology.
While we now know that it is isomorphic to singular homology, and hence it enjoys all the properties of singular homology, it still can be illuminating to derive these facts directly from Morse homology.
For example, we can prove Poincaré duality by simply changing the Morse function $f \leadsto -f$, i.e.\ turning the manifold upside down.  This has the effect of $k$-handles becoming $n-k$-handles and flow lines reversing direction, from which the desired result follows rapidly.
We also discuss the Morse inequalities giving a lower bound for the number of critical points of a Morse function in terms of the singular homology of $M$:
 \[
     \# \Crit_k f \ge \operatorname{rank} H_k(M, \Z)
.\] 
We end the chapter by proving some related facts including a stronger version of the Morse inequalities.

\bigskip
In \textbf{Chapter~\ref{chap:h-cobord}}, we prove that under certain conditions, the Morse inequalities can be attained by some Morse function.
In other words, there always exists a Morse function such that $\# \Crit_k f = \operatorname{rank} H_k(M, \Z)$.
We prove this by considering an arbitrary Morse function,
and then cancelling pairs of critical points until the Morse inequalities have been reached.
To this purpose, this chapter mostly contains stronger cancellation results.
In Section~\ref{sec:minimality}, all these cancellation theorems come together and we prove the minimality of the Morse inequalities.
This has as an immediate corollary two of the most important theorems in differential topology: the $h$-cobordism theorem and the generalized Poincaré conjecture in dimension $n \ge 5$, stating that a homotopy sphere is a topological sphere.
To end this thesis, we also discuss some of the historical aspects of these theorems.
