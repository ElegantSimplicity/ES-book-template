\chapter*{Conclusion}
\label{ch:conclusion}
\addcontentsline{toc}{chapter}{\nameref{ch:conclusion}}

We have started this thesis with an introduction to Morse functions.
While simple, they provide great insight in the structure of differential manifolds, as we have seen.
Morse functions give rise to handlebody decompositions, the Morse complex and eventually Morse homology.
We have shown that Morse homology does not depend on the Morse function nor on the pseudo-gradient vector field and that it is in fact isomorphic to singular homology.
Morse homology gives rise to the Morse inequalities, providing a lower bound for the number of critical points of a Morse function.

In the last part of the thesis, we have witnessed the power of ideas of Morse.
We have proven multiple cancellation theorems eventually leading to the proof of the minimality of the Morse inequalities, which was originally due to Smale.
This in turn has the $h$-cobordism and the generalized higher dimensional Poincaré conjecture as an immediate corollary, forming the pinnacle of this~thesis.

As of today, Morse theory remains an important subject in differential geometry.
For example, handlebody decompositions and Heegaard splittings are used extensively for studying $3$- and $4$-manifolds.
Moreover, the ideas of Morse homology have been extended to infinite dimensions by Andreas Floer, resulting in proofs of various versions of the Arnold conjecture.
