\setcounter{chapter}{-1}
\chapter{Preliminaries}

In this thesis, we assume the reader is familiar with basic concepts of differential geometry such as smooth manifolds, vector fields, flows, bundles, differential forms, et cetera.
In this chapter, we discuss some concepts that might be unfamiliar.

\section*{Differential Geometry}
The first concept we want to introduce is transversality.

\todo{Interesting source appendix of Schultens\sidecite{schultens2014introduction}}


% TODO: spell check in brackets in vim!
\begin{definition}[Transversality]
    Let $M$ be a manifold and $N_1, N_2$ be two submanifolds.
    Then $N_1$ intersects $N_2$ \emph{transversely} if and only if for all points of intersection $p \in N_1 \cap N_2$, we have
    $T_pM = T_pN_1 + T_p N_2$.
    We denote this by $ N_1 \tcap N_2$.\sidenotemark[0]
\end{definition}
\sidenotetext[0]{The notation is suggestive: the line $\cap$ intersects $|$ transversely $\tcap$.}
Note in particular that this depends on the ambient manifold $M$, and that if $N_1$ and $N_2$ do not intersect, their intersection is vacuously transverse.
We give some examples below.
\begin{figure}[H]
    \centering
    \sidecaption{
        Examples and non-examples of transverse intersections.
        Multiple configurations of two circles are shown: thrice with ambient manifold $\R^2$, and once embedded in $\R^3$.
    \label{fig:transversality-examples}
    }
    \incfig{transversality-examples}
\end{figure}
If the intersection of two manifolds $ N_1$ and $N_2$ is transverse, much more can be said about their intersection than in the general case.
In particular, as the following proposition states, we are guaranteed that $N_1 \tcap N_2$ is again a manifold, which as Figure~\ref{fig:intersection-of-manifolds-is-not-always-a-manifold} shows is not always true.
\begin{marginfigure}
    \centering
    \incfig{intersection-of-manifolds-is-not-always-a-manifold}
    \caption{Let $M = \R^2$ and let $N_1$ and $N_2$ be submanifolds as in the figure. Then $N_1$ and $N_2$ do not intersect transversely and their intersection is not a manifold: it is the union of a point and a closed interval.}
    \label{fig:intersection-of-manifolds-is-not-always-a-manifold}
\end{marginfigure}
\begin{prop}
    Let $M$ be a manifold and $N_1, N_2$ be two submanifolds. If $ N_1 \tcap N_2$, then $ N_1 \tcap N_2$ is again a manifold and 
    \[
        \codim (N_1 \tcap N_2) = \codim N_1 + \codim N_2
    ,\] 
    where the codimensions are to be taken with respect to $M$, by which we mean $\codim N_i = \dim M - \dim N_i$.
    Moreover, $T(N_1\tcap N_2) = TN_1 \tcap TN_2$.
\label{prop:transverse-codimensions-add}
\end{prop}
\begin{proof}
    As this is a local statement, we prove it for $M = \R^{m}$.
    We may assume that $N_1 = f_1^{-1}(0)$ and $ N_2 = f_2^{-1}(0)$ where $f_i$ are submersions from $\R^{m}$ to $\R^{n_i}$, where $n_i$ is the codimension of $N_i$.
    % i.e. $(f_i)_*$ has full rank everywhere.
    Then we can also consider $F= (f_1, f_2): \R^{m} \to  \R^{n_1} \times \R^{n_2}$.
    Notice that $ N_1 \cap N_2 = F^{-1}(0)$.
    Then
    \begin{align*}
        \dim N_1 \cap  N_2 &= \dim \Ker d_0F\\
                           &= \dim(\Ker d_0 f_1 \cap \Ker d_0 f_2)\\
                           &= \dim \Ker d_0 f_1 + \dim \Ker d_0 f_2 - \dim (\Ker d_0 f_1 + \Ker d_0 f_2)\\
                           &= (m - n_1) + (m - n_2) - m\\
                           &= m - (n_1 + n_2)
    .\end{align*} 
    This also shows that $T_p(N_1 \tcap N_2) = T_pN_1 \tcap T_pN_2$ and that $ N_1 \tcap N_2$ is a manifold, which is not always the case when the intersection is not transverse.
\end{proof}
\begin{marginfigure}[-5cm]
    \centering
    \incfig{codimensions-transverse}
    \caption{When $N_1 \tcap N_2$, the codimension of the intersection is the sum of their codimensions. By using the implicit function theorem, we straighten the situation.}
    \label{fig:codimensions-transverse}
\end{marginfigure}

An interesting property of transversality is that it is both \emph{generic} and \emph{stable}.
By generic we mean that any two manifolds intersecting can be made intersecting transversely by perturbing one of the manifolds.
Stability on the other hand means that when we perturbate a transverse intersection, it stays transverse.
At the core of the proof of stability and genericness of transversality lies the theorem of Sard:
% In morse rigorous terms, we have the following:
% \begin{definition}[Stable]
%     A property $P$ is stable for a class $C$ of maps $f:M \to  N$ if for all $f \in C$ that satisfy $P$, we have that there any homotopy  $f_t$ with  $f_0 = f$ there exists an $\epsilon>0$ such that  $f_t$ satisfies $P$ for each  $t < \epsilon$.
% \end{definition}
% \begin{definition}[Generic]
%     A property $P$ is generic for a class  $C$ of maps $f: M \to  N$ if the set of 
% \end{definition}
\begin{theorem}[Sard's theorem]
    Let $f: M \to  N$ be a $\Cinfty$ map. Then the set of critical values of  $f$ has measure zero in $N$.
\end{theorem}
Note in particular that it is the set of critical \emph{values} that has measure zero, not the set of critical points. 

% \begin{definition}[Isotopy]
%     Two embeddings $f_0, f_1: M \to  N$ are \emph{isotopic} if there is a continuous map $H: M \times  [0, 1] \to  N$ such that $H(x, 0) = f_0$ and $H(x, 1) = f_1(x)$ for all $x \in M$ and such that for all $t \in [0, 1]$, the map $f_t$ defined by $H(\cdot , t)$ is an embedding. The map $H$ is called an \emph{isotopy between} $f_0$ and $f_1$.

%     Two submanifolds $S_0, S_1$ of $M$ are \emph{isotopic} if their inclusion maps are isotopic.
% \end{definition}
% \sidenotetext{\fullcite[23]{schultens2014introduction}}

\section*{Homological Algebra}
In this section, we will assume that the reader is familiar with modules over a ring $R$. If not, they can mentally substitute each occurence of `module' by `vector space'.

\begin{definition}[Chain complex]
    A chain complex of $R$-modules is a sequence $M_\bul$ of the form
    \[
    \cdots \to  M_2 \xrightarrow{d_2}  M_1 \xrightarrow{d_1} M_0 \xrightarrow{d_0} M_{-1} \xrightarrow{d_{-1}}   M_{-2} \to  \cdots
    \] 
    where each term $M_i$ is an $R$-module and $d_i: M_i \to  M_{i-1}$ is an $R$-module homomorphism such that $d_{i-1}  \circ  d_i = 0$  for all $i \in \Z$.
\end{definition}
We often suppress the indices of the maps $d_i$ and the last condition becomes then  $d^2 = 0$.

\begin{marginfigure}
    \centering
    \incfig{homology-definition}
    \caption{Homology measure exactness of a chain complex.}
    \label{fig:homology-definition}
\end{marginfigure}

\begin{definition}[Homology]
    Let $M_\bul$ be a chain complex of  $R$-modules. The $i$-th homology of $M_\bul$ is
     \[
         H_i(M_\bul) = \frac{\Ker d_i}{\Im d_{i+1}}
    .\] 
    This is well defined because $\Im(d_{i+1}) \subset \Ker (d_i)$.
\end{definition}
It is clear that this measure the exactness of the sequence. Indeed, if the sequence is exact, i.e. $\Im d_{i+1} = \Ker d_i$, then $H_i(M_\bul) = 0$.


\begin{definition}[Chain map]
    A chain map $f_\bul: M_\bul \to  N_\bul$ is a collection of $R$-module homomorphisms which makes the following diagram commute:
    \[
        \begin{tikzcd}
            \cdots  \arrow[r, ""]&
            M_{i+1} \arrow[r, "d_{i+1}^{M}"] \arrow[d, "f_{i+1}"]&
            M_{i} \arrow[r, "d_{i}^{M}"] \arrow[d, "f_i"]&
            M_{i-1} \arrow[r, ""] \arrow[d, "f_{i-1}"]&
            \cdots \\
            \cdots  \arrow[r, ""]&
            N_{i+1} \arrow[r, "d_{i+1}^{N}"] &
            N_{i} \arrow[r, "d_{i}^{N}"] &
            N_{i-1} \arrow[r, ""] &
            \cdots \\
        \end{tikzcd}
    \]
    In other words, suppressing indices, $f  \circ  d^{M} = d^{N}  \circ f$.
\end{definition}

It is easy to check that chain maps induce a map on the level on homology which we denote by $H_i(f_\bul): H_i(M_\bul) \to  H_i(N_\bul)$, proving that $H_i$ is a functor. 

The following is a useful criterion for determining whether two chain maps induce the same maps on the level on homology.
\begin{definition}[Chain homotopic]
    Let $f_\bul, g_\bul$ be chain maps between  $M_\bul$ and $N_\bul$.
    A chain homotopy from  $f_\bul$ to  $g_\bul$ is a collection of  $R$-module homomorphisms $h_i: M_i \to  N_{i+1}$ such that $g_i - f_i = d^{N}_{i+1}  \circ  h_i + h_{i-1}  \circ  d_i^{M}$ for all $i$.
    If such a map exists, we say that $f_\bul$ and  $g_\bul$ are chain homotopic.
    \[
        \begin{tikzcd}[column sep=3em, row sep=3em]
            \cdots  \arrow[r, ""]&
            M_{i+1} \arrow[r, "d_{i+1}^{M}"] \arrow[d, "f_{i+1}"', shift right=2pt] \arrow[d, "g_{i+1}", shift left=2pt]&
            M_{i} \arrow[r, "d_{i}^{M}"] \arrow[d, "f_i"', shift right=2pt] \arrow[d, "g_{i}", shift left=2pt] \arrow[dl, "h_i"', dashed]&
            M_{i-1} \arrow[r, ""] \arrow[d, "f_{i-1}"', shift right=2pt]\arrow[d, "g_{i-1}", shift left=2pt] \arrow[dl, "h_{i-1}"', dashed]&
            \cdots \\
            \cdots  \arrow[r, ""]&
            N_{i+1} \arrow[r, "d_{i+1}^{N}"] &
            N_{i} \arrow[r, "d_{i}^{N}"] &
            N_{i-1} \arrow[r, ""] &
            \cdots \\
        \end{tikzcd}
    \]
    Suppressing indices, this becomes $g - f = dh + hd$.
\end{definition}
\begin{prop}
    Let $f_\bul, g_\bul$ be two chain homotopic chain maps from $M_\bul$ to $N_\bul$. 
    Then $H_i(f_\bul) = H_i(f_\bul)$, i.e.\ the maps induced on the level of homology are identical.
\end{prop}
\begin{proof}
    Let $h_i$ be a chain homotopy between  $f_\bul$ and  $g_\bul$.
    Let $x \in \Ker(d_i^{M})$.
    Then, suppressing indices,
    \begin{align*}
        g(x) + \Im(d^{N}) &= f(x) + (h  \circ  d^{M})(x) + (d^{N}  \circ h)(x)  + \Im(d^{N})\\
                          &= f(x) + h(0) + \Im(d^{N})\\
                        &= f(x) + \Im(d^{N}).
    \end{align*} 
\end{proof}
